
\chapter{Partitionen und Träger}

Unsere Konstruktion der logischen Formel aus einer Schaltkreisfamilie
wird voraussetzen, dass jedes Gate $g$ unter allen Bijektionen $\pi:A\rightleftarrows U$
ausgewertet wird. Wir stehen zunächst vor dem Problem, dass es $\left|\mathrm{Bij}\left(A,U\right)\right|=n!\in\Omega\left(2^{n}\right)$
solche Bijektionen gibt. Um sie für jedes Gate auf eine $\mathrm{poly}\left(n\right)$-beschränkte
Zahl zu reduzieren, möchten wir gerne nachweisen, dass die $n!$ Bijektionen
bezüglich jedem Gate in $\frac{n!}{\left(n-k\right)!}$ verschiedene
Äquivalenzklassen der Größe $\left(n-k\right)!$ (mit $k\in\mathcal{O}\left(1\right)$)
fallen (und einen effizienten Algorithmus finden, der die Repräsentanten
dieser Äquivalenzklassen erzeugt).

In dem Ergebnis von Martin Otto in 1997\cite{Otto1997} setzt er eine
lokale polynomielle Beschränkung der Orbits jedes Gates voraus, und
leitet für diese Schaltkreisfamilien eine Charakterisierung durch
das variablen-beschränkte Fragment der infinitären Logik ($L_{\infty\omega}^{\omega}$)
her.

Anderson und Dawar betrachten in \cite{AD2014} symmetrische Schaltkreisfamilien
mit polynomieller Größe (und daher automatisch polynomiell beschränkten
Orbits). Hier wird für jedes Gate nachgewiesen, dass sein Wert nur
von $\mathcal{O}\left(1\right)$ vielen Elementen des Universums $U$
abhängt, und daraus eine Charakterisierung durch die Fixpunktlogik
abgeleitet.

Hierfür werden zunächst die Begriffe der Partition, des Stabilisators
und des Trägers eingeführt.

\section{Partitionen einer Menge}

Sei $U$ ein beliebiges Universum. Wir führen die \textbf{Partition
}als Zerlegung von $U$ in disjunkte Teilmengen ein.
\begin{defn}
Sei $\mathcal{P}\coloneqq\left\{ P_{1},\cdots,P_{k}\right\} $ eine
Menge von disjunkten nicht-leeren Mengen. Wir nennen $\mathcal{P}$
eine Partition von $U$, wenn $\biguplus_{i=1}^{k}P_{i}=U$. Sei $\mathrm{Part}_{U}$
die Menge aller Partitionen von $U$. Sei $\sim_{\mathcal{P}}$ eine
Äquivalenzrelation auf $U$, deren Äquivalenzklassen von $\mathcal{P}$
repräsentiert werden.
\end{defn}
Die Permutationen $\mathrm{Sym}_{U}$ werden auf natürliche Weise
auf $\mathcal{P}\in\mathrm{Part}_{U}$ erweitert:
\begin{eqnarray*}
\pi\mathcal{P} & \coloneqq & \left\{ \pi P_{i}\mid P_{i}\in\mathcal{P}\right\} \in\mathrm{Part}_{U}
\end{eqnarray*}
Als nächstes wird die \textbf{Feinheit} als Relation auf $\mathrm{Part}_{U}$
eingeführt.
\begin{defn}
\label{def:feinheit}Sei $\mathcal{P}\preceq\mathcal{P}'$ (,,$\mathcal{P}$
ist mindestens so fein wie $\mathcal{P}'$``) genau dann wenn jedes
$P_{i}\in\mathcal{P}$ eine Teilmenge eines $P'_{j}\in\mathcal{P}'$
ist.

Dies ist äquivalent zu der Teilmengenbeziehung von $\sim_{\mathcal{P}}$
und $\sim_{\mathcal{P}'}$: 
\begin{eqnarray*}
\mathcal{P}\preceq\mathcal{P}' & \Leftrightarrow & \mathrm{f.a.}\,P\in\mathcal{P}\,\mathrm{ex.\,}P'\in\mathcal{P}'\,\mathrm{s.d.}\,P\subseteq P'\\
 & \Leftrightarrow & \mathrm{f.a.}\,u,u'\in U\,\mathrm{gilt}\,u\sim_{\mathcal{P}}u'\,\Rightarrow u\sim_{\mathcal{P}'}u'\\
 & \Leftrightarrow & \left(\sim_{\mathcal{P}}\right)\subseteq\left(\sim_{\mathcal{P}'}\right)
\end{eqnarray*}

Genau wie die Teilmengenbeziehung der Äquivalenzrelation bildet $\preceq$
eine Halbordnung auf $\mathrm{Part}_{U}$.

Daher bildet $\mathrm{Part}_{U}$ einen vollständigen Verband (siehe
\cite{Klein-Barmen1935}) mit den folgenden Infimum- und Supremum-Operationen
$\sqcap$ und $\sqcup$:
\begin{eqnarray*}
\mathcal{A}\sqcap\mathcal{B} & \coloneqq & \max_{\preceq}\left\{ \mathcal{P}\in\mathrm{Part}_{U}\mid\mathcal{P}\preceq\mathcal{A}\,\mathrm{und}\,\mathcal{P}\preceq\mathcal{B}\right\} \\
\mathcal{A}\sqcup\mathcal{B} & \coloneqq & \min_{\preceq}\left\{ \mathcal{P}\in\mathrm{Part}_{U}\mid\mathcal{A}\preceq\mathcal{P}\,\mathrm{und}\,\mathcal{B}\preceq\mathcal{P}\right\} 
\end{eqnarray*}

($\mathcal{A}\sqcap\mathcal{B}$ sei die gröbste feinere Partition
als $\mathcal{A}$ und $\mathcal{B}$, und $\mathcal{A}\sqcup\mathcal{B}$
die feinste gröbere Partition als $\mathcal{A}$ und $\mathcal{B}$.)
\end{defn}
\begin{prop}
\label{prop:part-eq}Wenn $\left[x\right]_{\sim}$ die Äquivalenzklasse
$\left\{ y\mid x\sim y\right\} $ ist, und $\mathcal{P}_{1},\mathcal{P}_{2}\in\mathrm{Part}_{U}$
wie folgt sind, 
\begin{eqnarray*}
\mathcal{P}_{1} & \coloneqq & \left\{ \left[x\right]_{\sim_{\mathcal{A}}\cap\sim_{\mathcal{B}}}\mid x\in U\right\} \\
\mathcal{P}_{2} & \coloneqq & \left\{ \left[x\right]_{\sim^{*}}\mid x\in U\right\} \\
 &  & \mathrm{wobei}\,\sim\coloneqq\left(\sim_{\mathcal{A}}\right)\cup\left(\sim_{\mathcal{B}}\right)
\end{eqnarray*}
dann ist $\mathcal{P}_{1}=\mathcal{A}\sqcap\mathcal{B}$ und $\mathcal{P}_{2}=\mathcal{A}\sqcup\mathcal{B}$.
\end{prop}
\begin{proof}
Die Eigenschaft $\mathcal{P}_{1}\preceq\mathcal{A},\mathcal{B}\preceq\mathcal{P}_{2}$
folgt offensichtlich aus $\left(\sim_{\mathcal{A}}\cap\sim_{\mathcal{B}}\right)\subseteq\sim_{\mathcal{A}},\sim_{\mathcal{B}}\subseteq\left(\sim_{\mathcal{A}}\cup\sim_{\mathcal{B}}\right)^{*}$.
Ferner muss jede Relation $\sim_{\mathcal{P}}$ mit $\left(\sim_{\mathcal{P}}\right)\subseteq\left(\sim_{\mathcal{A}}\right)$
und $\left(\sim_{\mathcal{A}}\right)\subseteq\left(\sim_{\mathcal{B}}\right)$
auch in $\left(\sim_{\mathcal{A}}\right)\cap\left(\sim_{\mathcal{B}}\right)$
enthalten sein, und jede mit $\left(\sim_{\mathcal{P}}\right)\supseteq\left(\sim_{\mathcal{A}}\right)$,
$\left(\sim_{\mathcal{P}}\right)\supseteq\left(\sim_{\mathcal{B}}\right)$
muss auch $\left(\sim_{\mathcal{A}}\right)\cup\left(\sim_{\mathcal{B}}\right)$
(und dessen Abschluss) enthalten.
\end{proof}
\begin{defn}
Die feinste und gröbste Partition von $U$ seien $\mathcal{P}_{\min}\left(U\right)\coloneqq\left\{ \left\{ u_{1}\right\} ,\cdots,\left\{ u_{n}\right\} \right\} $
und $\mathcal{P}_{\max}\left(U\right)\coloneqq\left\{ U\right\} $.
\end{defn}

\section{Stabilisatoren einer Partition}

Wir führen die Stabilisatoren von Elementen, Teilmengen und Partitionen
eines Universums $U$ ein.
\begin{defn}
Der Stabilisator eines Elements $u\in U$ in $U$ sei die Untergruppe
$\mathrm{Stab}_{U}\left(u\right)\subseteq\mathrm{Sym}_{U}$ aller
Permutationen, die $u$ fixieren: 
\[
\mathrm{Stab}_{U}\left(u\right)\coloneqq\left\{ \pi\in\mathrm{Sym}_{U}\mid\pi u=u\right\} 
\]

Der Punktstabilisator\textbf{ }einer Teilmenge $X\subseteq U$ in
$U$ sei die Untergruppe $\mathrm{Stab}_{U}\left(X\right)\subseteq\mathrm{Sym}_{U}$
der Permutationen, die jedes Element $x\in X$ fixieren:
\begin{eqnarray*}
\mathrm{Stab}_{U}\left(X\right) & \coloneqq & \left\{ \pi\in\mathrm{Sym}_{U}\mid\pi x=x\,\,\mathrm{f.a.}\,\,x\in X\right\} \\
 & = & \bigcap_{x\in X}\mathrm{Stab}_{U}\left(x\right)
\end{eqnarray*}

Der Mengenstabilisator von $X$ in $U$ sei die Untergruppe $\mathrm{Stab}_{U}\left\{ X\right\} \subseteq\mathrm{Sym}_{U}$
der Permutationen, die die Menge $X$ als ganzes fixieren.
\begin{eqnarray*}
\mathrm{Stab}_{U}\left\{ X\right\}  & \coloneqq & \left\{ \pi\in\mathrm{Sym}_{U}\mid\pi X=X\right\} \\
 & = & \bigcup_{\pi\in\mathrm{Sym}_{X}}\left(\pi\cup\mathbf{id}_{U\backslash X}\right)\left(\mathrm{Stab}_{U}\left(X\right)\right)
\end{eqnarray*}
\end{defn}

\begin{defn}
Die obige Definition wird auf Partitionen $\mathcal{P}\in\mathrm{Part}_{U}$
erweitert:

Der Punktstabilisator\textbf{ }von $\mathcal{P}$ in $U$ sei die
Untergruppe $\mathrm{Stab}_{U}\left(\mathcal{P}\right)\subseteq\mathrm{Sym}_{U}$
aller Permutationen, die jede Menge $P_{i}\in\mathcal{P}$ fixieren:
\begin{eqnarray*}
\mathrm{Stab}_{U}\left(\mathcal{P}\right) & \coloneqq & \left\{ \pi\in\mathrm{Sym}_{U}\mid\pi P_{i}=P_{i}\,\mathrm{f\ddot{u}r}\,\mathrm{alle}\,P_{i}\in\mathcal{P}\right\} \\
 & = & \bigcap_{P_{i}\in\mathcal{P}}\mathrm{Stab}_{U}\left\{ P_{i}\right\} 
\end{eqnarray*}

Der Mengenstabilisator von $\mathcal{P}$ in $U$ sei die Untergruppe
$\mathrm{Stab}_{U}\left\{ \mathcal{P}\right\} \subseteq\mathrm{Sym}_{U}$
aller Permutationen, die die Partition als ganzes fixieren: 
\begin{eqnarray*}
\mathrm{Stab}_{U}\left\{ \mathcal{P}\right\}  & \coloneqq & \left\{ \pi\in\mathrm{Sym}_{U}\mid\pi\mathcal{P}=\mathcal{P}\right\} 
\end{eqnarray*}
Diese Gruppe wird durch $\mathrm{Stab}_{U}\left(\mathcal{P}\right)$
und alle Permutationen von gleich-mächtigen Elementen von $\mathcal{P}$
erzeugt: Sei $\mathcal{P}_{\mid i}\coloneqq\left\{ P_{j}\in\mathcal{P}\mid\left|P_{j}\right|=i\right\} $
für $i\in\left[1,\left|U\right|\right]$, dann gilt: 
\[
\mathrm{Stab}_{U}\left\{ \mathcal{P}\right\} =\bigcup_{\begin{subarray}{c}
i\in\left[1,\left|U\right|\right]\\
\pi\in\mathrm{Sym}_{\mathcal{P}_{\mid i}}
\end{subarray}}\pi\mathrm{Stab}_{U}\left(\mathcal{P}\right)
\]

Die Feinheit von Partitionen ist äquivalent zu der Teilmengenbeziehung
ihrer Stabilisatoren.
\end{defn}
\begin{prop}
\label{prop:feinheit-stab}Für zwei Partitionen $\mathcal{P},\mathcal{P}'\in\mathrm{Part}_{U}$
gilt $\mathcal{P}\preceq\mathcal{P}'$ genau dann wenn $\mathrm{Stab}_{U}\left(\mathcal{P}\right)\subseteq\mathrm{Stab}_{U}\left(\mathcal{P}'\right)$.
\end{prop}
\begin{proof}
Sei $\pi\in\mathrm{Stab}_{U}\left(\mathcal{P}\right)$ beliebig, so
besteht $\pi$ aus einer Folge von Transpositionen:
\[
\pi=\left(u_{1}v_{1}\right)\circ\cdots\circ\left(u_{k}v_{k}\right)
\]

Jede Transposition $\left(u_{i}v_{i}\right)$ vertauscht Elemente
einer Menge $P_{i}\in\mathcal{P}$; daher gilt $u_{i}\sim_{\mathcal{P}}v_{i}$.
Per Definition \ref{def:feinheit} gilt $\left(\sim_{\mathcal{P}}\right)\subseteq\left(\sim_{\mathcal{P}'}\right)$,
und daher $u_{i}\sim_{\mathcal{P}'}v_{i}$ und $\left(u_{i}v_{i}\right)\in\mathrm{Stab}_{U}\left(\mathcal{P}'\right)$
für alle $i\in\left[1,k\right]$. Aus der Abgeschlossenheit des Stabilisators
folgt $\pi\in\mathrm{Stab}_{U}\left(\mathcal{P}'\right)$.

Umgekehrt impliziert auch $\mathrm{Stab}_{U}\left(\mathcal{P}\right)\subseteq\mathrm{Stab}_{U}\left(\mathcal{P}'\right)$,
dass für jedes Paar $u\sim_{\mathcal{P}}v$ die Transposition $\left(uv\right)\in\mathrm{Stab}_{U}\left(\mathcal{P}\right)$
auch in $\mathrm{Stab}_{U}\left(\mathcal{P}'\right)$ enthalten ist,
und daher $u\sim_{\mathcal{P}'}v$ gilt. Aus $\left(\sim_{\mathcal{P}}\right)\subseteq\left(\sim_{\mathcal{P}'}\right)$
folgt $\mathcal{P}\preceq\mathcal{P}'$.
\end{proof}

\section{Träger}

\subsection{Trägerpartitionen einer Permutationsgruppe}
\begin{defn}
\label{def:traeger}Sei $G\subseteq\mathrm{\mathrm{Sym}}_{U}$ eine
Untergruppe und $\mathcal{P}$ eine Partition von $U$. $\mathcal{P}$
sei eine \textbf{Trägerpartition} von\textbf{ }$G$ genau dann wenn
$\mathrm{Stab}_{U}\left(\mathcal{P}\right)\subseteq G$.
\end{defn}
Wenn die Partition $\mathcal{P}$ eine Trägerpartition von $G$ ist,
dann ist sie es auch von jeder Obermenge $G'\supseteq G$. Außerdem
sind per Satz \ref{prop:feinheit-stab} alle feineren $\mathcal{P}'\preceq\mathcal{P}$
Trägerpartitionen von $G$. Somit ist $\mathcal{P}_{\min}=\left\{ \left\{ u_{1}\right\} ,\cdots,\left\{ u_{n}\right\} \right\} $
mit $\mathrm{Stab}_{U}\left(\mathcal{P}_{\min}\right)=\left\{ \mathbf{id}\right\} $
eine triviale Trägerpartition jeder Untergruppe $G\subseteq\mathrm{Sym}_{U}$,
und alle $\mathcal{P}\in\mathrm{Part}_{U}$ sind Trägerpartitionen
der Gruppe $\mathrm{Sym}_{U}=\mathrm{Stab}_{U}\left(\mathcal{P}_{\max}\right)$.
\begin{prop}
\label{prop:traeger-abschluss}Wenn $\mathcal{A}$ und $\mathcal{B}$
Trägerpartitionen von $G$ sind, so sind es auch $\mathcal{A}\sqcap\mathcal{B}$
und $\mathcal{A}\sqcup\mathcal{B}$.
\end{prop}
\begin{proof}
Per Definition ist $\mathcal{A}\sqcap\mathcal{B}\preceq\mathcal{A}$
und $\mathcal{A}\sqcap\mathcal{B}\preceq\mathcal{B}$, und daher folgt
die Tatsache direkt aus Satz \ref{prop:feinheit-stab}.

Für $\mathcal{P}\coloneqq\mathcal{A}\sqcup\mathcal{B}$ gilt:

\begin{enumerate}
\item Jede Permutation $\pi\in\mathrm{Stab}_{U}\left(\mathcal{P}\right)$
ist eine Folge von Transpositionen $\pi=\left(u_{1}v_{1}\right)\circ\cdots\circ\left(u_{k}v_{k}\right)$,
so dass wir nur Transpositionen betrachten müssen.
\item Die Äquivalenzrelation $\sim_{\mathcal{P}}$ ist per Satz \ref{prop:part-eq}
die transitive Hülle von $\sim\coloneqq\left(\sim_{\mathcal{A}}\cup\sim_{\mathcal{B}}\right)$.
Daher existiert für alle $u\sim_{\mathcal{P}}v$ eine Folge von $\ell\leqslant\left|U\right|$
Elementen $\bar{w}\in U^{\ell}$ mit
\[
u\sim w_{1}\sim\cdots\sim w_{\ell}\sim v
\]
\end{enumerate}
Sei nun $\left(uv\right)\in\mathrm{Stab}_{U}\left(\mathcal{P}\right)$
eine beliebige Transposition. $\left(uv\right)$ lässt sich mit dem
entsprechenden $\bar{w}\in\left(U\backslash\left\{ u,v\right\} \right)^{*}$
in die folgenden Transpositionen zerlegen:
\begin{eqnarray*}
\left(uv\right) & = & \left(\begin{array}{c}
u\\
v
\end{array}\left(\begin{array}{c}
w_{i}\\
w_{i}
\end{array}\right)_{1\leqslant i\leqslant\ell}\begin{array}{c}
v\\
u
\end{array}\right)\\
 & = & \left(\begin{array}{c}
w_{1}\\
v
\end{array}\left(\begin{array}{c}
w_{i}\\
w_{i-1}
\end{array}\right)_{1<i\leqslant\ell}\begin{array}{c}
v\\
w_{\ell}
\end{array}\right)\circ\left(\begin{array}{c}
u\\
w_{1}
\end{array}\left(\begin{array}{c}
w_{i}\\
w_{i+1}
\end{array}\right)_{1\leqslant i<\ell}\begin{array}{cc}
w_{\ell} & v\\
v & u
\end{array}\right)\\
 &  & \left(vw_{\ell}\right)\left(w_{\ell}w_{\ell-1}\right)\cdots\left(w_{2}w_{1}\right)\circ\left(uw_{1}\right)\left(w_{1}w_{2}\right)\cdots\left(w_{\ell-1}w_{\ell}\right)\left(w_{\ell}v\right)
\end{eqnarray*}

Weil $\sim=\left(\sim_{\mathcal{A}}\cup\sim_{\mathcal{B}}\right)$,
ist jede der Transpositionen entweder in $\mathrm{Stab}_{U}\left(\mathcal{A}\right)$
oder in $\mathrm{Stab}_{U}\left(\mathcal{B}\right)$ enthalten, und
beide sind Teilmengen von $G$. Also ist $\left(uv\right)\in G$,
und es folgt $\mathrm{Stab}_{U}\left(\mathcal{P}\right)\subseteq G$.
\end{proof}
\begin{cor}
Jede Gruppe $G\subseteq\mathrm{Sym}_{U}$ besitzt eine eindeutige
gröbste Trägerpartition.
\end{cor}
\begin{proof}
Angenommen, $\mathcal{P}$ und $\mathcal{P}'$ seien zwei gröbste
Trägerpartitionen von $G$. Nun ist $\mathcal{P}\sqcup\mathcal{P}'$
nach Lemma \ref{prop:traeger-abschluss} ebenfalls eine Trägerpartition
von $G$, und es gilt $\mathcal{P},\mathcal{P}'\preceq\mathcal{P}\sqcup\mathcal{P}'$.

Da aber per Definition $\mathcal{P}$ und $\mathcal{P}'$ aber per
Definition gröbste Träger von $G$ sind, muss $\mathcal{P}=\mathcal{P}\sqcup\mathcal{P}'=\mathcal{P}'$
gelten.
\end{proof}
\begin{defn}
Für jede Gruppe $G\subseteq\mathrm{Sym}_{U}$ sei $\mathrm{SP}\left(G\right)$
der gröbste Träger von $G$.
\end{defn}
Wir betrachten nun die Konjugations-Operation $\pi G\pi^{-1}$ einer
Permutation $\pi$ auf einer Untergruppe $G\subseteq\mathrm{Sym}_{U}$,
und weisen nach, dass $\pi\mathrm{SP}\left(G\right)=\mathrm{SP}\left(\pi G\pi^{-1}\right)$.
\begin{prop}
\label{prop:konjugation}Wenn eine Partition $\mathcal{P}$ ein Träger
einer Gruppe $G\subseteq\mathrm{Sym}_{U}$ ist, dann ist $\pi\mathcal{P}$
ein Träger von $\pi G\pi^{-1}$ für alle $\pi\in\mathrm{Sym}_{U}$.
\end{prop}
\begin{proof}
Seien $\rho\in\mathrm{Stab}_{U}\left(\pi\mathcal{P}\right)$ und $P_{i}\in\mathcal{P}$
beliebig. $\pi^{-1}\rho\pi$ fixiert $P_{i}$:
\begin{eqnarray*}
\left(\pi^{-1}\rho\pi\right)P_{i} & = & \pi^{-1}\left(\rho\left(\pi P_{i}\right)\right)\\
 & = & \pi^{-1}\pi P_{i}\\
 & = & P_{i}
\end{eqnarray*}

Daraus folgt $\left(\pi^{-1}\rho\pi\right)\in\mathrm{Stab}_{U}\left(\mathcal{P}\right)\subseteq G$,
und schließlich gilt:
\begin{eqnarray*}
\rho & = & \left(\pi\pi^{-1}\right)\rho\left(\pi\pi^{-1}\right)\\
 & = & \pi\left(\pi^{-1}\rho\pi\right)\pi^{-1}\\
 & \in & \pi G\pi^{-1}
\end{eqnarray*}

Damit $\pi\mathcal{P}$ ein Träger der konjugierten Gruppe $\pi G\pi^{-1}$.
\end{proof}
\begin{cor}
\label{cor:sp-konjugation}Für jede Gruppe $G\subseteq\mathrm{Sym}_{U}$
und jede Permutation $\pi\in\mathrm{Sym}_{U}$ ist $\pi\mathrm{SP}\left(G\right)=\mathrm{SP}\left(\pi G\pi^{-1}\right)$.
\end{cor}
\begin{proof}
Nach Lemma \ref{prop:konjugation} ist $\pi\mathrm{SP}\left(G\right)$
eine Trägerpartition von $\pi G\pi^{-1}$, und daher gilt $\pi\mathrm{SP}\left(G\right)\preceq\mathrm{SP}\left(\pi G\pi^{-1}\right)$.

Umgekehrt ist auch $\pi^{-1}\mathrm{SP}\left(\pi G\pi^{-1}\right)$
eine Trägerpartition von $\pi^{-1}\pi G\pi\pi^{-1}=G$. Es folgt $\pi^{-1}\mathrm{SP}\left(\pi G\pi^{-1}\right)\preceq\mathrm{SP}\left(G\right)$
und daher $\mathrm{SP}\left(\pi G\pi^{-1}\right)\preceq\pi\mathrm{SP}\left(G\right)$.
\end{proof}
\begin{prop}
Jede Gruppe $G$ ist Obermenge des Punkt- und Teilmenge des Mengenstabilisators
von $\mathrm{SP}\left(G\right)$:
\[
\mathrm{Stab}_{U}\left(\mathrm{SP}\left(G\right)\right)\subseteq G\subseteq\mathrm{Stab}_{U}\left\{ \mathrm{SP}\left(G\right)\right\} 
\]
\end{prop}
\begin{proof}
Per Definition \ref{def:traeger} gilt bereits $\mathrm{Stab}_{U}\left(\mathrm{SP}\left(G\right)\right)\subseteq G$.

Sei nun $\pi\in G$ beliebig. Weil $\pi G\pi^{-1}=G$, folgt nach
Korollar \ref{cor:sp-konjugation}: 
\[
\pi\mathrm{SP}\left(G\right)=\mathrm{SP}\left(\pi G\pi^{-1}\right)=\mathrm{SP}\left(G\right)
\]

Weil $\pi$ die Partition $\mathrm{SP}\left(G\right)$ auf sich selbst
abbildet, gilt per Definition $\pi\in\mathrm{Stab}_{U}\left\{ \mathrm{SP}\left(G\right)\right\} $.
\end{proof}

\subsection{Trägermengen im Schaltkreis}

Wir erweitern die Begriffe ,,Stabilisator`` und ,,Träger`` auf
die Gates eines rigiden (Definition \ref{def:rigid}), symmetrischen
Schaltkreises. $\mathcal{C}=\left(G,W,\Sigma,\Omega,U\right)$.
\begin{defn}
Für jedes Gate $g\in G$ sei der Stabilisator von $g$ wie folgt definiert:
\[
\mathrm{Stab}_{\mathcal{C}}\left(g\right)\coloneqq\left\{ \pi\in\mathrm{Sym}_{U}\mid\hat{\pi}g=g\right\} 
\]

Eine Menge $X\subseteq U$ sei eine Trägermenge von $g$, wenn jede
Permutation, die die Elemente von $X$ fixiert, einen Automorphismus
in $\mathcal{C}$ induziert, der $g$ fixiert: 
\[
\mathrm{Stab}_{U}\left(X\right)\subseteq\mathrm{Stab}_{\mathcal{C}}\left(g\right)
\]
\end{defn}
Da für $X\subseteq X'$ offensichtlich $\mathrm{Stab}_{U}\left(X'\right)\subseteq\mathrm{Stab}_{U}\left(X\right)$
gilt (je mehr Elemente fixiert werden, um so weniger Permutationen
lassen wir zu), ist hier vor allem die kleinste Trägermenge des Gates
interessant.
\begin{prop}
Wenn $X,X'\subseteq U$ zwei Trägermengen von $g\in G$ sind, dann
ist $X\cap X'$ ebenfalls eine Trägermenge von $g$.
\end{prop}
\begin{proof}
Sei $\mathcal{P}\coloneqq\left\{ \left\{ x\right\} \mid x\in X\right\} \cup\left\{ U\backslash X\right\} $
und $\mathcal{P}'\coloneqq\left\{ \left\{ x\right\} \mid x\in X'\right\} \cup\left\{ U\backslash X'\right\} $.
Offensichtlich gilt $\mathrm{Stab}_{U}\left(\mathcal{P}\right)=\mathrm{Stab}_{U}\left(X\right)\subseteq\mathrm{Stab}_{\mathcal{C}}\left(g\right)$
und $\mathrm{Stab}_{U}\left(\mathcal{P}'\right)=\mathrm{Stab}_{U}\left(X'\right)\subseteq\mathrm{Stab}_{\mathcal{C}}\left(g\right)$.

Daher sind $\mathcal{P}$ und $\mathcal{P}'$ beide Trägerpartitionen
von $\mathrm{Stab}_{\mathcal{C}}\left(g\right)$. Per Satz \ref{prop:traeger-abschluss}
ist auch $\mathcal{P}\sqcup\mathcal{P}'=\left\{ \left\{ x\right\} \mid x\in X\cap X'\right\} \cup\left\{ U\backslash\left(X\cap X'\right)\right\} $
eine Trägerpartition, und per $\mathrm{Stab}_{U}\left(\mathcal{P}\sqcup\mathcal{P}'\right)=\mathrm{Stab}_{U}\left(X\cap X'\right)$
ist auch $X\cap X'$ eine Trägermenge von $g$.
\end{proof}
\begin{cor}
Jedes Gate $g\in G$ besitzt eine eindeutige kleinste Trägermenge.
\end{cor}
\begin{prop}
Sei $\mathcal{P}\coloneqq\mathrm{SP}\left(\mathrm{Stab}_{\mathcal{C}}\left(g\right)\right)$
die gröbste Trägerpartition des Stabilisators eines Gates $g$, und
sei $P_{\max}=\max_{\left|\cdot\right|}\mathcal{P}$. Dann ist $X\coloneqq U\backslash P_{\max}$
die kleinste Trägermenge von $g$.
\end{prop}
\begin{proof}
Sei $X'$ mit $\left|X'\right|<\left|X\right|$ eine kleinere Trägermenge
von $g$. Per Definition ist $\mathcal{P}'\coloneqq\left\{ \left\{ x\right\} \mid x\in X'\right\} \cup\left\{ U\backslash X'\right\} $
eine Trägerpartition von $\mathrm{Stab}_{\mathcal{C}}\left(g\right)$,
denn $\mathrm{Stab}_{U}\left(\mathcal{P}'\right)=\mathrm{Stab}_{U}\left(X'\right)\subseteq\mathrm{Stab}_{\mathcal{C}}\left(g\right)$.

Per Definition ist $U\backslash X=P_{\max}$ eine größte Menge in
$\mathrm{SP}\left(\mathrm{Stab}_{\mathcal{C}}\left(g\right)\right)$.
Per Annahme ist $\left|X'\right|<\left|X\right|$ und daher $\left|U\backslash X'\right|>\left|U\backslash X\right|$.
Weil aber $\mathcal{P}'\preceq\mathrm{SP}\left(\mathrm{Stab}_{\mathcal{C}}\left(g\right)\right)$
ist, muss $\mathrm{SP}\left(\mathrm{Stab}_{\mathcal{C}}\left(g\right)\right)$
eine Obermenge von $U\backslash X'$ als Element enthalten, und deren
Größe ist mindestens $\left|U\backslash X'\right|>\left|U\backslash X\right|$;
es entsteht ein Widerspruch.
\end{proof}
\begin{defn}
Sei $\mathrm{sp}\left(g\right)$ die kleinste Trägermenge von $g$,
und sei $\mathrm{sp}\left(\mathcal{C}\right)=\max_{g\in G}\left|\mathrm{sp}\left(g\right)\right|$
die maximale Größe aller kleinsten Trägermengen.
\end{defn}

\section{Obere Schranken für die Größe von Trägern}

Das Ergebnis von Anderson und Dawar beruht auf einem Theorem, das
eine konstante obere Schranke $\mathrm{S}\left(\mathcal{C}_{n}\right)\in\mathcal{O}\left(1\right)$
für jede symmetrischen Schaltkreisfamilie $\left(\mathcal{C}_{n}\right)_{n\in\mathbb{N}}$
polynomieller Größe nachweist. Diese konstante Größe führt zu einer
polynomiell beschränkten Anzahl von Permutationen $\left|\mathrm{Sym}_{\mathrm{S}\left(g\right)}\right|=\left|\mathrm{S}\left(g\right)\right|!\leqslant n^{\left|S\left(g\right)\right|}$
jeder Trägermenge eines Gates.

Wir stellen hierfür das sogenannte Support-Theorem vor:
\begin{thm}
\textbf{\label{thm:Support-Theorem}Support-Theorem} (Theorem 21 aus
\cite{AD2014})

Für $\epsilon\in\mathbb{R}_{\left[\frac{2}{3},1\right]}$ und einen
rigiden symmetrischen Schaltkreis $\mathcal{C}=\left(G,W,\Sigma,\Omega,U\right)$
mit $\left|U\right|>2^{\frac{56}{\epsilon^{2}}}$, gilt: Wenn die
maximalen Orbit-Größe mit $s\coloneqq\max_{g\in G}\mathrm{Orb}_{\mathcal{C}}\left(g\right)\leqslant2^{n^{1-\epsilon}}$
subexponentiell ist, dann ist $\mathrm{sp}\left(\mathcal{C}\right)\leqslant\frac{33}{\epsilon}\frac{\log s}{\log n}$.
\end{thm}
\begin{cor}
\label{cor:korollar-23}(Korollar 23 aus \cite{AD2014})

Für jede symmetrische, rigide $\left(\sigma,\mathbb{B}\right)$-Schaltkreisfamilie
mit $\mathrm{poly}\left(n\right)$-Größe gilt $\mathrm{sp}\left(\mathcal{C}_{n}\right)\in\mathcal{O}\left(1\right)$.
\end{cor}
\begin{proof}
Die $\mathrm{poly}\left(n\right)$-Größe des Schaltkreises $\mathcal{C}_{n}=\left(G,W,\Sigma,\Omega,U\right)$
impliziert für jedes $\epsilon<1$ und hinreichend große $n\in\mathbb{N}$
offensichtlich: 
\[
s\coloneqq\max_{g\in G}\mathrm{Orb}_{\mathcal{C}}\left(g\right)\leqslant\left|\mathcal{C}_{n}\right|\leqslant n^{c}<2^{n^{1-\epsilon}}
\]
Damit ist $\mathrm{sp}\left(\mathcal{C}\right)\leqslant\frac{33}{\epsilon}\frac{\log s}{\log n}\leqslant\frac{33}{\epsilon}\frac{k\log n}{\log n}=\frac{33k}{\epsilon}\in\mathcal{O}\left(1\right)$. 
\end{proof}


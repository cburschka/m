
\chapter{Von Formeln zu Schaltkreisfamilien}

\section{Logik erster Stufe}

Wir weisen zunächst nach, dass die Logik der ersten Stufe durch symmetrische,
$\mathrm{LOGSPACE}$-uniforme Schaltkreisfamilien konstanter Tiefe
beschrieben wird. 
\[
\mathrm{FO}\subseteq\left(\mathrm{SAC}^{0}\right)^{\mathrm{LOGSPACE}}
\]

\begin{lem}
\label{lem:fo-circuit}Jede $\mathrm{FO}\left[\sigma\right]$-Formel
$\varphi\left(\bar{x}\right)$ definiert eine $\sigma$-Anfrage $q$,
die von einer symmetrischen $\mathrm{LOGSPACE}$-uniformen Schaltkreis-Familie
$\bar{\mathcal{C}^{\varphi}}$ mit $\left\Vert \varphi\right\Vert $-Tiefe
und $\left\Vert \varphi\right\Vert n^{\mathtt{MF}\left(\varphi\right)}$-Größe
berechnet wird.
\end{lem}
\begin{proof}
Sei $\sigma$ eine relationale Signatur, und sei $\varphi\left(\bar{x}\right)$
eine $k$-stellige $\mathrm{FO}\left[\sigma\right]$-Formel.

Per induktiver Konstruktion über den Aufbau von $\varphi$ wird der
$k$-stellige Schaltkreis $\mathcal{C}_{n}^{\varphi}$ über dem Universum
$U\coloneqq\left[1,n\right]$ definiert, so dass $\left\llbracket \varphi\right\rrbracket \left(\mathfrak{A},\bar{t}\right)\Leftrightarrow\left\llbracket \mathcal{C}_{n}^{\varphi}\right\rrbracket \left(\mathfrak{A},\bar{t}\right)$
für alle $n\in\mathbb{N}$, $\mathfrak{A}\in\mathbf{FIN}^{\left[1,n\right]}\left(\sigma\right)$
und $\bar{t}\in\left[1,n\right]^{k}$ gilt. Für eine beliebige Permutation
$\pi\in\mathrm{Sym}_{U}$ wird ein Automorphismus $\hat{\pi}$ angegeben,
und damit die Symmetrie nachgewiesen.

\begin{casenv}
\item Falls $\varphi\left(\bar{x}\right)=R\bar{y}$ für $R/m\in\sigma$
und $\bar{y}\in\mathrm{frei}\left(\varphi\right)^{m}$, so besteht
$\mathcal{C}_{n}^{\varphi}$ aus $n^{k}$ Gates.

\begin{description}
\item [{Schaltkreis:}] 
\[
\mathcal{C}_{n}^{\varphi}\coloneqq\left(\left\{ g_{\bar{t}}\mid\bar{t}\in U^{k}\right\} ,\emptyset,\Sigma,\Omega,U\right)
\]
Für jedes Tupel $\bar{t}\in U^{k}$ sei $\beta_{\bar{t}}\coloneqq\left(\bar{x}\mapsto\bar{t}\right)$
die Belegung der Variablen $\bar{x}$ mit $\bar{t}$:
\begin{eqnarray*}
\Sigma\left(g_{\bar{t}}\right) & \coloneqq & R\beta_{\bar{t}}\left(\bar{y}\right)\\
\Omega\left(\bar{t}\right) & \coloneqq & g_{\bar{t}}
\end{eqnarray*}
\item [{Korrektheit:}] 
\[
\begin{array}{cccccc}
\left\llbracket R\bar{x}\right\rrbracket \left(\mathfrak{A},\bar{t}\right) & = & \left[R^{\mathfrak{A}}\right]\left(\beta_{\bar{t}}\bar{y}\right) & = & \mathcal{C}_{n}^{\varphi}\left[\mathfrak{A}\right]\left(g_{\bar{t}}\right)= & \left\llbracket \mathcal{C}_{n}^{\varphi}\right\rrbracket \left(\mathfrak{A},\bar{t}\right)\end{array}
\]
\item [{Symmetrie:}] Sei $\hat{\pi}g_{\bar{t}}\coloneqq g_{\pi\bar{t}}$
für alle Tupel $\bar{t}\in U^{k}$. Per Definition ist $\pi\beta_{\bar{t}}\left(\bar{y}\right)=\beta_{\pi\bar{t}}\left(\bar{y}\right)$
und daher 
\begin{eqnarray*}
\Sigma\left(\hat{\pi}g_{\bar{t}}\right) & = & \pi\Sigma\left(g_{\bar{t}}\right)\\
\Omega\left(\pi\bar{t}\right) & = & \hat{\pi}\Omega\left(\bar{t}\right)
\end{eqnarray*}
\item [{Größe:}] Der Schaltkreis hat die Tiefe $0$ und die Größe $n^{k}=n^{\mathtt{MF}\left(\varphi\right)}$.
\end{description}
\item Falls $\varphi\left(\bar{x}\right)=y_{1}\dot{=}y_{2}$, so besteht
$\mathcal{C}_{n}^{\varphi}$ aus $n^{k}$ isolierten Gates (hier ist
$k\in\left\{ 1,2\right\} $).

\begin{description}
\item [{Schaltkreis:}] 
\[
\mathcal{C}_{n}^{\varphi}\coloneqq\left(\left\{ g_{\bar{t}}\mid\bar{t}\in U^{k}\right\} ,\emptyset,\Sigma,\Omega,U\right)
\]
Für jedes Tupel $\bar{t}\in U^{k}$ sei $\beta_{\bar{t}}\coloneqq\left(\bar{x}\mapsto\bar{t}\right)$
die entsprechende Belegung:
\begin{eqnarray*}
\Sigma\left(g_{\bar{t}}\right) & \coloneqq & \begin{cases}
\mathbf{1} & \mathrm{falls}\,\,\beta\left(y_{1}\right)=\beta\left(y_{2}\right)\\
\mathbf{0} & \mathrm{sonst}
\end{cases}\\
\Omega\left(\bar{t}\right) & \coloneqq & g_{\bar{t}}
\end{eqnarray*}
\item [{Korrektheit:}] 
\begin{eqnarray*}
\left\llbracket y_{1}\dot{=}y_{2}\right\rrbracket \left(\mathfrak{A},\bar{t}\right)=1 & \Leftrightarrow & \left(\beta_{\bar{t}}y_{1}=\beta_{\bar{t}}y_{2}\right)\\
 & \Leftrightarrow & \mathcal{C}_{n}^{\varphi}\left[\mathfrak{A}\right]\left(g_{\bar{t}}\right)=1\\
 & \Leftrightarrow & \left\llbracket \mathcal{C}_{n}^{\varphi}\right\rrbracket \left(\mathfrak{A},\bar{t}\right)=1
\end{eqnarray*}
\item [{Symmetrie:}] Sei $\hat{\pi}\left(g_{\bar{t}}\right)\coloneqq g_{\pi\bar{t}}$.
Es gilt $\beta_{\bar{t}}\left(x\right)=\beta_{\bar{t}}\left(x'\right)$
genau dann wenn $\beta_{\pi\bar{t}}\left(x\right)=\beta_{\pi\bar{t}}\left(x\right)$,
und daher ist $\Sigma\left(\hat{\pi}g_{\bar{t}}\right)=\Sigma\left(g_{\pi\bar{t}}\right)$.
\item [{Größe:}] Der Schaltkreis hat die Tiefe $0$ und die Größe $n^{k}=n^{\mathtt{MF}\left(\varphi\right)}$.
\end{description}
\item Falls $\varphi\left(\bar{x}\right)=\varphi_{1}\left(\bar{y_{1}}\right)\wedge\cdots\wedge\varphi_{m}\left(\bar{y_{m}}\right)$
mit $\mathrm{ar}\left(\varphi_{i}\right)=k_{i}$, so besteht $\mathcal{C}_{n}^{\varphi}$
aus der disjunkten Vereinigung aller $\mathcal{C}_{n}^{\varphi_{i}}$
für $1\leqslant i\leqslant m$ mit der folgenden Erweiterung.

\begin{description}
\item [{Schaltkreis:}] 
\begin{eqnarray*}
\mathcal{C}_{n}^{\varphi_{i}} & = & \left(G_{i},W_{i},\Sigma_{i},\Omega_{i},U\right)\\
\mathcal{C}_{n}^{\varphi} & \coloneqq & \left(G,W,\Sigma,\Omega,U\right)
\end{eqnarray*}
Es werden neue Outputs für jedes $k$-Tupel aus $U$ hinzugefügt:
\begin{eqnarray*}
G & \coloneqq & \biguplus_{i=1}^{m}G_{i}\uplus\left\{ g_{\bar{t}}\mid\bar{t}\in U^{k}\right\} 
\end{eqnarray*}
Die Outputs werden entsprechend mit denen von $\mathcal{C}_{n}^{\varphi_{i}}$
verknüpft, wobei $\rho_{i}:U^{k}\rightarrow U^{k_{i}}$ ein $k$-Tupel
$\bar{t}$ wie folgt auf die in $\varphi_{i}$ frei vorkommenden Variablen
reduziere:
\begin{eqnarray*}
\mathrm{Sei}\,\,\,\bar{j} & \in & \left[1,k\right]^{k_{i}}\\
\mathrm{so}\,\mathrm{dass}\,\,\,\bar{y_{i}} & = & \left(x_{\left(j_{1}\right)},\cdots,x_{\left(j_{k_{i}}\right)}\right)
\end{eqnarray*}
\begin{eqnarray*}
\mathrm{dann}\,\,\,\rho_{i}\left(t_{1},\cdots,t_{k}\right) & \coloneqq & \left(t_{\left(j_{1}\right)},\cdots,t_{\left(j_{k_{i}}\right)}\right)
\end{eqnarray*}
\begin{eqnarray*}
W & \coloneqq & \bigcup_{i=1}^{m}W_{i}\cup W_{\mathtt{AND}}\\
W_{\mathtt{AND}} & \coloneqq & \left\{ \left(\Omega_{i}\left(\rho_{i}\bar{t}\right),g_{\bar{t}}\right)\mid1\leqslant i\leqslant m,\,\bar{t}\in U^{k}\right\} 
\end{eqnarray*}
Die Gates werden entsprechend beschriftet:
\begin{eqnarray*}
\Sigma\left(g\right) & \coloneqq & \begin{cases}
\Sigma_{i}\left(g\right) & \mathrm{f\ddot{u}r}\,\,g\in G_{i}\\
\mathtt{AND} & \mathrm{sonst}
\end{cases}\\
\Omega\left(\bar{t}\right) & \coloneqq & g_{\bar{t}}\,\,\mathrm{f\ddot{u}r\,alle\,\,}\bar{t}\in U^{k}
\end{eqnarray*}
\item [{Korrektheit:}] Es gilt für $\bar{t}\in U^{k}$: 
\begin{eqnarray*}
\left\llbracket \varphi_{1}\wedge\cdots\wedge\varphi_{m}\right\rrbracket \left(\mathfrak{A},\bar{t}\right) & = & \min_{1\leqslant i\leqslant m}\left\llbracket \varphi_{i}\right\rrbracket \left(\mathfrak{A},\bar{t}\right)\\
 & = & \min_{1\leqslant i\leqslant m}\mathcal{C}_{n}^{\varphi_{i}}\left[\mathfrak{A}\right]\left(\Omega_{i}\left(\rho_{i}\bar{t}\right)\right)\\
 & = & \mathcal{C}_{n}^{\varphi}\left[\mathfrak{A}\right]\left(g_{\bar{t}}\right)
\end{eqnarray*}
\item [{Symmetrie:}] Es existieren bereits die Automorphismen $\hat{\pi_{i}}$
für jeden Schaltkreis $\mathcal{C}_{n}^{\varphi_{i}}$. Der Automorphismus
$\hat{\pi}$ erweitert diese wie folgt:
\[
\hat{\pi}\left(g\right)\coloneqq\begin{cases}
\hat{\pi}_{i}\left(g\right) & \mathrm{f\ddot{u}r}\,g\in G_{i}\\
g_{\pi\bar{t}} & \mathrm{f\ddot{u}r}\,g=g_{\bar{t}}
\end{cases}
\]
Für die Gates und Kanten der Schaltkreise $\mathcal{C}_{n}^{\varphi_{i}}$
ist $\hat{\pi}$ per Annahme bereits korrekt.

\begin{enumerate}
\item Für jede neue Kante $\left(\Omega_{i}\left(\rho_{i}\bar{t}\right),g_{\bar{t}}\right)\in W_{\mathtt{AND}}$
gilt nach Voraussetzung: 
\begin{eqnarray*}
\left(\hat{\pi}\Omega_{i}\left(\rho_{i}\bar{t}\right),\hat{\pi}g_{\bar{t}}\right) & = & \left(\hat{\pi}_{i}\Omega_{i}\left(\rho_{i}\bar{t}\right),\hat{\pi}g_{\bar{t}}\right)\\
 & = & \left(\Omega_{i}\left(\rho_{i}\pi\bar{t}\right),g_{\pi\bar{t}}\right)\\
 & \in & W_{\mathtt{AND}}
\end{eqnarray*}
(Die Reduktion $\rho_{i}:U^{k}\rightarrow U^{k_{i}}$ ist ein Homomorphismus
und kommutiert mit der Permutation $\pi$.)
\item Es gilt $\Sigma\left(\hat{\pi}g_{\bar{t}}\right)=\Sigma\left(g_{\bar{t}}\right)=\mathtt{AND}$.
\item Es gilt $\hat{\pi}\Omega\left(\bar{t}\right)=\hat{\pi}g_{\bar{t}}=g_{\pi\bar{t}}=\Omega\left(\pi\bar{t}\right)$.
\end{enumerate}
\item [{Größe:}] Der Schaltkreis hat die Tiefe $T\left(\mathcal{C}_{n}^{\varphi}\right)$
und die Größe $\left|\mathcal{C}_{n}^{\varphi}\right|$: 
\begin{eqnarray*}
T\left(\mathcal{C}_{n}^{\varphi}\right) & = & 1+\max_{i=1}^{m}T\left(\mathcal{C}_{n}^{\psi_{i}}\right)\\
 & \overset{\mathrm{Ann.}}{\leqslant} & 1+\max_{i=1}^{m}\left\Vert \psi_{i}\right\Vert \\
 & \leqslant & 1+\sum_{i=1}^{m}\left\Vert \psi_{i}\right\Vert \leqslant\left\Vert \varphi\right\Vert 
\end{eqnarray*}
 
\begin{eqnarray*}
\left|\mathcal{C}_{n}^{\varphi}\right| & = & n^{k}+\sum_{i=1}^{m}\left|\mathcal{C}_{n}^{\psi_{i}}\right|\\
 & \overset{\mathrm{Ann.}}{\leqslant} & n^{k}+\sum_{i=1}^{m}\left\Vert \psi_{i}\right\Vert n^{\mathtt{MF}\left(\psi_{i}\right)}\\
 & \leqslant & n^{\mathtt{MF}\left(\varphi\right)}+\sum_{i=1}^{m}\left\Vert \psi_{i}\right\Vert n^{\mathtt{MF}\left(\varphi\right)}\\
 & \leqslant & n^{\mathtt{MF}\left(\varphi\right)}\left(1+\sum_{i=1}^{m}\left\Vert \psi_{i}\right\Vert \right)\leqslant n^{\mathtt{MF}\left(\varphi\right)}\left\Vert \varphi\right\Vert 
\end{eqnarray*}
\end{description}
\item Falls $\varphi\left(\bar{x}\right)=\varphi_{1}\vee\cdots\vee\varphi_{\ell}$,
so ist der Schaltkreis analog zu Fall 3 mit $\Sigma\left(g_{\bar{t}}\right)=\mathtt{OR}$.
\item Falls $\varphi\left(\bar{x}\right)=\neg\psi$, so wird der Schaltkreis
$\mathcal{C}_{n}^{\psi}$ wie folgt erweitert:

\begin{description}
\item [{Schaltkreis:}] 
\begin{eqnarray*}
\mathcal{C}_{n}^{\psi} & = & \left(G',W',\Sigma',\Omega',U\right)\\
\mathcal{C}_{n}^{\varphi} & \coloneqq & \left(G,W,\Sigma,\Omega,U\right)
\end{eqnarray*}
Für jedes Tupel $\bar{t}\in U^{k}$ wird ein neues Gate $g_{\bar{t}}$
eingefügt. Die Gates werden mit den Outputs von $\mathcal{C}_{n}^{\varphi'}$
verknüpft. 
\begin{eqnarray*}
G & \coloneqq & G'\uplus\left\{ g_{\bar{t}}\mid\bar{t}\in U^{k}\right\} \\
W & \coloneqq & W'\cup W_{\mathtt{NOT}}\\
W_{\mathtt{NOT}} & \coloneqq & \left\{ \left(\Omega'\left(\bar{t}\right),g_{\bar{t}}\right)\mid\bar{t}\in U^{k}\right\} \\
\Sigma\left(g\right) & \coloneqq & \begin{cases}
\Sigma'\left(g\right) & \mathrm{falls}\,\,g\in G'\\
\mathtt{NOT} & \mathrm{sonst}
\end{cases}\\
\Omega\left(\bar{t}\right) & \coloneqq & g_{\bar{t}}
\end{eqnarray*}
\item [{Korrektheit:}] 
\begin{eqnarray*}
\left\llbracket \neg\psi\right\rrbracket \left(\mathfrak{A},\bar{t}\right) & = & 1-\left\llbracket \psi\right\rrbracket \left(\mathfrak{A},\bar{t}\right)\\
 & = & 1-\mathcal{C}_{n}^{\psi}\left[\mathfrak{A}\right]\left(\Omega'\left(\bar{t}\right)\right)\\
 & = & \mathcal{C}_{n}^{\varphi}\left[\mathfrak{A}\right]\left(g_{\bar{t}}\right)
\end{eqnarray*}
\item [{Symmetrie:}] Es existiert bereits der Automorphismus $\hat{\pi'}$.
Dieser wird wie folgt erweitert:
\[
\hat{\pi}\left(g\right)\coloneqq\begin{cases}
\hat{\pi'}\left(g\right) & \mathrm{falls}\,\,g\in G'\\
g_{\pi\bar{t}} & \mathrm{falls}\,\,g=g_{\bar{t}}
\end{cases}
\]
Dann gilt:
\begin{eqnarray*}
\hat{\pi}W_{\mathtt{NOT}} & = & \left\{ \left(\hat{\pi}\Omega'\left(\bar{t}\right),\hat{\pi}g_{\bar{t}}\right)\mid\bar{t}\in U^{k}\right\} \\
 & = & \left\{ \left(\Omega'\left(\pi\bar{t}\right),g_{\pi\bar{t}}\right)\mid\bar{t}\in U^{k}\right\} =W_{\mathtt{NOT}}\\
\Sigma\left(\hat{\pi}g_{\bar{t}}\right) & = & \Sigma\left(g_{\bar{t}}\right)=\mathtt{NOT}
\end{eqnarray*}
\item [{Größe:}] Der Schaltkreis hat die Tiefe $T\left(\mathcal{C}_{n}^{\varphi}\right)$
und die Größe $\left|\mathcal{C}_{n}^{\varphi}\right|$: 
\begin{eqnarray*}
T\left(\mathcal{C}_{n}^{\varphi}\right) & = & 1+T\left(\mathcal{C}_{n}^{\psi}\right)\\
 & \leqslant & 1+\left\Vert \psi\right\Vert \leqslant\left\Vert \varphi\right\Vert 
\end{eqnarray*}
\begin{eqnarray*}
\left|\mathcal{C}_{n}^{\varphi}\right| & = & n^{k}+\left|\mathcal{C}_{n}^{\psi}\right|\\
 & \overset{\mathrm{Ann.}}{\leqslant} & n^{k}+\left\Vert \psi\right\Vert n^{\mathtt{MF}\left(\psi\right)}\\
 & \leqslant & n^{\mathtt{MF}\left(\varphi\right)}+\left\Vert \psi\right\Vert n^{\mathtt{MF}\left(\varphi\right)}\\
 & \leqslant & \left\Vert \varphi\right\Vert n^{\mathtt{MF}\left(\varphi\right)}
\end{eqnarray*}
\end{description}
\item \label{case:fo-ex}Falls $\varphi\left(\bar{x}\right)=\exists y_{1}\cdots\exists y_{m}\psi\left(z_{1},\cdots,z_{k+m}\right)$,
so wird der Schaltkreis $\mathcal{C}_{n}^{\varphi'}$ wie folgt erweitert:

\begin{description}
\item [{Schaltkreis:}] 
\begin{eqnarray*}
\mathcal{C}_{n}^{\varphi} & \coloneqq & \left(G,W,\Sigma,\Omega,U\right)\\
\mathcal{C}_{n}^{\varphi'} & \coloneqq & \left(G',W',\Sigma',\Omega',U\right)
\end{eqnarray*}
Sei $\rho:U^{k+m}\rightarrow U^{k}$ die Abbildung, die aus $\bar{z}$
die gebundenen Variablen $\bar{y}$ entferne:
\begin{eqnarray*}
\mathrm{Sei}\,\,\,\bar{i} & \in & \left[1,k\right]^{k}\\
\mathrm{so}\,\mathrm{dass}\,\,\,\bar{x} & = & \left(z_{\left(i_{1}\right)},\cdots,z_{\left(i_{k}\right)}\right)\\
\mathrm{dann}\,\,\,\rho\left(t_{1},\cdots,t_{k}\right) & \coloneqq & \left(t_{\left(i_{1}\right)},\cdots,t_{\left(i_{k}\right)}\right)
\end{eqnarray*}
\end{description}
Es werden neue Outputs eingefügt.
\begin{eqnarray*}
G & \coloneqq & G'\uplus\left\{ g_{\bar{t}}\mid\bar{t}\in U^{k}\right\} 
\end{eqnarray*}
Jedes Gate $\Omega'\left(\bar{u}\right)$ mit $\bar{u}\in U^{k+m}$
wird mit dem Gate $g_{\rho\bar{u}}$ verknüpft.
\begin{eqnarray*}
W & \coloneqq & W'\cup W_{\exists}\\
W_{\exists} & \coloneqq & \left\{ \left(\Omega'\left(\bar{u}\right),g_{\rho\bar{u}}\right)\mid\bar{u}\in U^{k+m}\right\} 
\end{eqnarray*}

Die neuen Outputs werden mit $\mathtt{OR}$ markiert.
\begin{eqnarray*}
\Sigma\left(g\right) & \coloneqq & \begin{cases}
\Sigma'\left(g\right) & \mathrm{f\ddot{u}r}\,g\in G'\\
\mathtt{OR} & \mathrm{sonst}
\end{cases}
\end{eqnarray*}
\[
\Omega\left(\bar{t}\right)\coloneqq g_{\bar{t}}
\]

\begin{description}
\item [{Korrektheit:}] Für $\bar{t}\in U^{k}$ gilt: 
\begin{eqnarray*}
\left\llbracket \varphi\right\rrbracket \left(\mathfrak{A},\bar{t}\right) & = & \max_{\begin{subarray}{c}
\bar{u}\in U^{k+m}\\
\rho\bar{u}=\bar{t}
\end{subarray}}\left\llbracket \psi\right\rrbracket \left(\mathfrak{A},\bar{u}\right)=1\\
 & = & \max_{\begin{subarray}{c}
\bar{u}\in U^{k+m}\\
\rho\bar{u}=\bar{t}
\end{subarray}}\left(\mathcal{C}_{n}^{\varphi}\left[\mathfrak{A}\right]\left(\Omega'\left(\bar{u}\right)\right)\right)\\
 & = & \mathcal{C}_{n}^{\varphi}\left[\mathfrak{A}\right]\left(g_{\bar{t}}\right)
\end{eqnarray*}
\item [{Symmetrie:}] Es existiert bereits der Automorphismus $\hat{\pi'}$.
Dieser wird wie folgt erweitert:
\[
\hat{\pi}\left(g\right)\coloneqq\begin{cases}
\hat{\pi'}\left(g\right) & \mathrm{f\ddot{u}r}\,\,g\in G'\\
g_{\pi\bar{t}} & \mathrm{f\ddot{u}r}\,\,g=g_{\bar{t}}
\end{cases}
\]
Auf den Gates von $\mathcal{C}_{n}^{\varphi'}$ ist $\hat{\pi}$ per
Annahme treu zu $\pi$.

\begin{enumerate}
\item Für die neuen Kanten $\left(\Omega'\left(\bar{u}\right),g_{\rho\bar{u}}\right)\in W$
gilt: 
\[
\left(\hat{\pi}\Omega'\left(\bar{u}\right),\hat{\pi}g_{\rho\bar{u}}\right)=\left(\Omega'\left(\pi\bar{u}\right),g_{\rho\pi\bar{u}}\right)\in W
\]
\item $\hat{\pi}\Sigma\left(g_{\bar{t}}\right)=\Sigma\left(g_{\bar{t}}\right)=\mathtt{OR}$.
\item $\Omega\left(\pi\bar{t}\right)=g_{\pi\bar{t}}=\hat{\pi}g_{\bar{t}}=\hat{\pi}\Omega\left(\bar{t}\right)$.
\end{enumerate}
\item [{Größe:}] Der Schaltkreis hat die Tiefe $T\left(\mathcal{C}_{n}^{\varphi}\right)$
und die Größe $\left|\mathcal{C}_{n}^{\varphi}\right|$:
\begin{eqnarray*}
T\left(\mathcal{C}_{n}^{\varphi}\right) & = & 1+T\left(\mathcal{C}_{n}^{\psi}\right)\\
 & \leqslant & 1+\left\Vert \psi\right\Vert \leqslant\left\Vert \varphi\right\Vert 
\end{eqnarray*}
\begin{eqnarray*}
\left|\mathcal{C}_{n}^{\varphi}\right| & = & n^{k}+\left|\mathcal{C}_{n}^{\psi}\right|\\
 & \overset{\mathrm{Ann.}}{\leqslant} & n^{k}+\left\Vert \psi\right\Vert n^{\mathtt{MF}\left(\psi\right)}\\
 & \leqslant & n^{\mathtt{MF}\left(\varphi\right)}+\left\Vert \psi\right\Vert n^{\mathtt{MF}\left(\varphi\right)}\\
 & \leqslant & \left\Vert \varphi\right\Vert n^{\mathtt{MF}\left(\varphi\right)}
\end{eqnarray*}
\end{description}
\item \label{case:fo-forall}Falls $\varphi\left(\bar{x}\right)=\forall y_{1}\cdots\forall y_{m}\psi\left(\bar{z}\right)$,
so sei der Schaltkreis analog zu Fall 8 mit $\Sigma\left(g_{\bar{t}}\right)\coloneqq\mathtt{AND}$.
\end{casenv}
\begin{description}
\item [{Speicherplatz}] Die beschriebene Konstruktion wird von dem $\left\Vert \varphi\right\Vert \left|\mathrm{var}\left(\varphi\right)\right|\log n$-platzbeschränkten
Algorithmus \ref{alg:fo-circ} berechnet.

Jeder Aufruf von $\mathtt{Berechne}\left(\varphi\left(\bar{x}\right)\right)$
iteriert über Variablen $\bar{t}\in U^{\mathrm{var}\left(\varphi\right)}$,
die $\mathrm{var}\left(\varphi\right)\log n$ Bits belegen. Da die
maximale Tiefe der Rekursion durch $\left\Vert \varphi\right\Vert $
beschränkt ist, werden insgesamt $\left\Vert \varphi\right\Vert \left|\mathrm{var}\left(\varphi\right)\right|\log n$
Bits benötigt.

(Hierbei werden ohne Details eine passende Kodierung der Formel $\varphi$
und ein Algorithmus zur platz-effizienten Berechnung von $\mathrm{frei}\left(\varphi\right)$
vorausgesetzt.)
\end{description}
\end{proof}
\begin{algorithm}
\begin{lyxcode}
Berechne($\varphi\left(\bar{x}\right)$):~

\begin{lyxcode}
Für~jedes~$\bar{t}\in\left[1,n\right]^{\mathrm{ar}\left(\bar{x}\right)}$:~

\begin{lyxcode}
Gib~Gate~$g_{\varphi\left(\bar{t}\right)}$~und~$\Omega_{\varphi}\left(\bar{t}\right)\coloneqq g_{\varphi\left(\bar{t}\right)}$~aus.

Gib~$\Sigma_{\varphi}\left(g_{\varphi\left(\bar{t}\right)}\right)$~entsprechend~der~Konstruktion~aus.
\end{lyxcode}
Für~jede~direkte~Teilformel~$\psi\left(\bar{y}\right)$:~

\begin{lyxcode}
Berechne($\psi\left(\bar{y}\right)$).

Für~jedes~$\bar{t}\in\left[1,n\right]^{\mathrm{ar}\left(\bar{x}\right)}$:~

\begin{lyxcode}
Für~jedes~$\bar{t}'\in\left[1,n\right]^{\mathrm{ar}\left(\bar{y}\right)}$:~

\begin{lyxcode}
Falls~$t_{i}=t'_{j}$~für~alle~$i\in\left[1,\mathrm{ar}\left(\bar{x}\right)\right],j\in\left[1,\mathrm{ar}\left(\bar{y}\right)\right]$~mit~$x_{i}=y_{j}$:~

\begin{lyxcode}
Verknüpfe~$\Omega_{\psi}\left(\bar{t}'\right)$~mit~$g_{\varphi\left(\bar{t}\right)}$.
\end{lyxcode}
\end{lyxcode}
\end{lyxcode}
\end{lyxcode}
\end{lyxcode}
\end{lyxcode}
\caption{\label{alg:fo-circ}Berechnung von $\mathcal{C}_{n}^{\varphi}$ für
$\mathrm{FO}\left[\sigma\right]$-Formeln.}
\end{algorithm}


\section{Disjunkte numerische Erweiterungen}

Wir betrachten eine beliebige numerische Erweiterung $\mathcal{L}+\Upsilon$,
und weisen nach, dass eine Schaltkreis-Konstruktion für $\mathcal{L}$
angepasst werden kann, ohne die asymptotische Tiefe und Größe zu verändern.

Die neue Schaltkreisfamilie ist $\mathcal{K}$-uniform (mit $\mathcal{K}\supseteq\mathrm{LOGSPACE}$),
wenn sowohl die Konstruktion für $\mathcal{L}$ als auch das Orakel
$\mathcal{K}$-uniform sind.
\begin{lem}
\label{lem:oracle-circuit}Sei $\mathcal{L}$ eine Logik, $\mathbb{B}$
eine boolesche Basis und $\mathcal{K}\supseteq\mathrm{LOGSPACE}$
eine Komplexitätsklasse, so dass für jede $\mathcal{L}\left[\sigma\right]$-Formel
eine symmetrische $\mathcal{K}$-uniforme $\left(\sigma,\mathbb{B}\right)$-Schaltkreisfamilie
mit $t\left(n\right)$-Tiefe und $s\left(n\right)$-Größe (mit $s\mathrm{\left(n\right)\in\mathrm{poly}\left(n\right)}$)
existiert, die die gleiche Anfrage beschreibt.

Sei $\eta$ eine von $\sigma$ disjunkte Signatur und $\Upsilon$
ein $\mathcal{K}$-uniformes $\eta$-Orakel.

Dann existiert auch für jede $\left(\mathcal{L}+\Upsilon\right)\left[\sigma\right]$-Formel
eine ebensolche Schaltkreisfamilie mit der Tiefe $t'\left(n\right)\coloneqq t\left(2n+1\right)$
und der Größe $s'\left(n\right)\coloneqq s\left(2n+1\right)$.
\end{lem}
Insbesondere folgt dann aus der Kombination mit Lemma \ref{lem:fo-circuit}:
\begin{eqnarray*}
\mathrm{FO}+\mathbf{BIT} & \subseteq & \left(\mathrm{SAC}^{0}\right)^{\mathrm{LOGSPACE}}\\
\mathrm{FO}+\mathbf{ARB} & \subseteq & \left(\mathrm{SAC}^{0}\right)^{P/\mathrm{poly}}
\end{eqnarray*}

\begin{proof}
Sei $\varphi$ eine $\left(\mathcal{L}+\Upsilon\right)\left[\sigma\right]$-Formel.
Für $n\in\mathbb{N}$ und $\mathfrak{A}\in\mathbf{FIN}^{\left(n\right)}\left(\sigma\right)$
wird $\varphi$ als $\mathcal{L}\left[\sigma\uplus\eta\uplus\left\{ \leqslant\right\} \right]$-Formel
auf der disjunkt vereinigten $\left(\sigma\uplus\eta\uplus\left\{ \leqslant\right\} \right)$-Struktur
$\mathfrak{A}\uplus\Upsilon\left(n\right)$ ausgewertet, wobei $\Upsilon\left(n\right)\in\mathbf{FIN}_{<}^{\left[0,n\right]}\left(\eta\right)$
das Universum $\left[0,n\right]$ hat.

Der Schaltkreis $\mathcal{C}_{n}$ wird aber auf einer umbenannten
$\sigma$-Struktur $\mathfrak{A}'\coloneqq\pi\mathfrak{A}\in\mathbf{FIN}^{\left[1,n\right]}\left(\sigma\right)$
mit für $\pi:A\rightleftarrows\left[1,n\right]$ ausgewertet. Um $\mathfrak{A}'$
disjunkt mit $\Upsilon\left(n\right)$ zu vereinigen, werden wir zuerst
das Universum $\left[0,n\right]$ durch eine Umbenennung nach $\left[n+1,2n+1\right]$
,,verschieben``:
\begin{eqnarray*}
\Upsilon'\left(n\right) & \coloneqq & \rho\Upsilon\left(n\right)\\
\rho & : & \left[0,n\right]\rightarrow\left[n+1,2n+1\right]\\
\rho\left(i\right) & \coloneqq & i+n+1
\end{eqnarray*}

Nun betrachten wir die disjunkte Vereinigung (wobei die Prädikate
aus $\eta\uplus\left\{ \leqslant\right\} $ nur auf $\left[n+1,2n+1\right]$
definiert sind): 
\[
\mathfrak{A}'\uplus\Upsilon'\left(n\right)\in\mathbf{FIN}^{\left[1,2n+1\right]}\left(\sigma\uplus\eta\uplus\left\{ \leqslant\right\} \right)
\]

Weil $\Upsilon\left(n\right)\cong\Upsilon'\left(n\right)$ und $\mathfrak{A}\cong\mathfrak{A}'$,
gilt auch $\mathfrak{A}\uplus\Upsilon\left(n\right)\cong\mathfrak{A}'\uplus\Upsilon'\left(n\right)$
(denn der Isomorphismus ist unter disjunkter Vereinigung abgeschlossen):
\[
\left\llbracket \varphi\right\rrbracket \left(\mathfrak{A}\uplus\Upsilon\left(n\right),\beta\right)=\left\llbracket \varphi\right\rrbracket \left(\mathfrak{A}'\uplus\Upsilon'\left(n\right),\beta\right)
\]

Per Voraussetzung können wir in $\mathcal{K}$ einen $\left(\sigma\uplus\eta\uplus\left\{ \leqslant\right\} ,\mathbb{B}\right)$-Schaltkreis
$\mathcal{C}_{2n+1}^{\varphi}$ über $U'=\left[1,2n+1\right]$ berechnen.
Dieser arbeitet korrekt auf $\mathfrak{A}'\uplus\Upsilon'\left(n\right)$
und hat offensichtlich die geforderte Größe $s'\left(n\right)=s\left(2n+1\right)$
und Tiefe $t'\left(n\right)=t\left(2n+1\right)$.

Anschließend konstruieren wir daraus den $\left(\sigma,\mathbb{B}\right)$-Schaltkreis
$\dot{\mathcal{C}_{n}^{\varphi}}\coloneqq\left(G,W,\dot{\Sigma},\dot{\Omega},U\right)$
über $U\coloneqq\left[1,n\right]$, indem alle Inputs $\Sigma\left(g\right)=R\bar{t}$
mit $\bar{t}\notin U^{*}$ oder $R\notin\sigma$ durch Konstanten
$\dot{\Sigma}\left(g\right)\in\left\{ \mathbf{0},\mathbf{1}\right\} $
ersetzt werden, und die Ausgangsfunktion auf $\dot{\Omega}=\Omega_{\mid U}$
reduziert wird. Das ist eine einfache Iteration über die Gates $G$
des Schaltkreises, die nur einen Zähler der Größe $\log\left(s'\left(n\right)\right)\in\log\left(\mathrm{poly}\left(n\right)\right)=\mathcal{O}\left(\log n\right)$
benötigt, und somit in $\mathrm{LOGSPACE}$ bleibt.

\begin{casenv}
\item Für $\Sigma\left(g\right)=R\bar{t}$ mit $R/k\in\sigma$ werden die
,,überschüssigen`` Inputs einfach auf $\mathbf{0}$ gesetzt:
\begin{eqnarray*}
\dot{\Sigma}\left(g\right) & \coloneqq & \begin{cases}
R\bar{t} & \mathrm{falls}\,\,\bar{t}\in U^{k}\\
\mathbf{0} & \mathrm{sonst}
\end{cases}
\end{eqnarray*}
\item Für $\Sigma\left(g\right)=R\bar{t}$ mit $\sigma/k\in\eta\uplus\left\{ \leqslant\right\} $
wird die numerische Relation $R^{\Upsilon'\left(n\right)}$ fest in
den Schaltkreis eingebaut:
\begin{eqnarray*}
\dot{\Sigma}\left(g\right) & \coloneqq & \begin{cases}
\mathbf{1} & \mathrm{falls}\,\,\bar{t}\in R^{\Upsilon'\left(n\right)}\\
\mathbf{0} & \mathrm{sonst}
\end{cases}
\end{eqnarray*}
\end{casenv}
Sei nun $\pi\in\mathrm{Sym}_{U}$ eine beliebige Permutation.

Wir betrachten die Erweiterung $\pi'\coloneqq\pi\cup\mathbf{id}_{\left[n+1,2n+1\right]}$,
die die Elemente von $\Upsilon'\left(n\right)$ auf sich selbst abbildet.
Offensichtlich ist $\pi'\in\mathrm{Sym}_{U'}$ eine Permutation von
$U'=\left[1,2n+1\right]$.

Aus der Symmetrie des Schaltkreises $\mathcal{C}_{2n+1}^{\varphi}$
bezüglich $\mathrm{Sym}_{U'}$ (siehe Lemma \ref{lem:fo-circuit})
folgt die Existenz eines von $\pi'$ induzierten Automorphismus $\hat{\pi}$.

Es wird nun nachgewiesen, dass $\hat{\pi}$ auch ein von $\pi$ induzierter
Automorphismus in $\dot{\mathcal{C}_{n}^{\varphi}}$ ist. Dazu müssen
nur $\dot{\Sigma}$ und $\dot{\Omega}$ betrachtet werden, da der
Graph $\left(G,W\right)$ unverändert bleibt. Ferner betrachten wir
nur die nicht-konstanten Inputs von $\mathcal{C}_{2n+1}^{\varphi}$,
denn ansonsten bleibt $\Sigma$ unverändert.

\begin{casenv}
\item Für $\Sigma\left(g\right)=R\bar{t}$ mit $R/k\in\sigma$ gilt:

\begin{casenv}
\item Falls $\bar{t}\in U^{k}$:
\[
\dot{\Sigma}\left(\hat{\pi}g\right)=\Sigma\left(\hat{\pi}g\right)=R\pi\bar{t}
\]
\item Sonst:
\[
\dot{\Sigma}\left(\hat{\pi}g\right)=\mathbf{0}=\dot{\Sigma}\left(g\right)
\]
\end{casenv}
\item Für $\Sigma\left(g\right)=R\bar{t}$ mit $R/k\in\eta\uplus\left\{ \leqslant\right\} $
gilt:

\begin{casenv}
\item Falls $\bar{t}\in R^{\Upsilon'\left(n\right)}\subseteq\left[n+1,2n+1\right]^{k}$,
dann ist $\pi'\bar{t}=\bar{t}$:
\[
\dot{\Sigma}\left(\hat{\pi}g\right)=\mathbf{1}=\dot{\Sigma}\left(g\right)
\]
\item Sonst:
\[
\dot{\Sigma}\left(\hat{\pi}g\right)=\mathbf{0}=\dot{\Sigma}\left(g\right)
\]
\end{casenv}
\end{casenv}
Außerdem gilt ist $U^{k}$ bezüglich $\pi'$ abgeschlossen: 
\[
\hat{\pi}\dot{\Omega}\left(\bar{t}\right)=\hat{\pi}\Omega\left(\bar{t}\right)=\Omega\left(\pi'\bar{t}\right)=\dot{\Omega}\left(\pi'\bar{t}\right)
\]
 Damit ist $\bar{\dot{\mathcal{C}^{\varphi}}}$ eine symmetrische,
$\mathcal{K}$-uniforme Schaltkreisfamilie mit Tiefe $t'\left(n\right)=t\left(2n+1\right)$
und Größe $s'\left(n\right)=s\left(2n+1\right)$, die $q$ berechnet.
\end{proof}

\section{Logiken mit Zählquantoren}

Wir betrachten die Logik $\mathcal{L}+\Upsilon+C$, und weisen nach,
dass die Konstruktion von Lemma \ref{lem:oracle-circuit} angepasst
werden kann, indem Majority-Gates hinzugefügt werden.
\begin{lem}
\label{lem:counting-circuit}Sei $\mathcal{L}$ eine Logik, $c\in\mathbb{N}$
eine Konstante, $\mathbb{B}$ eine boolesche Basis und $\mathcal{K}\supseteq\mathrm{LOGSPACE}$
eine Komplexitätsklasse.

Es gelte die Voraussetzung dass für jede $\mathcal{L}\left[\sigma\right]$-Formel
eine symmetrische, $\mathcal{K}$-uniforme $\left(\sigma,\mathbb{B}\right)$-Schaltkreisfamilie
mit $\left\Vert \varphi\right\Vert n^{c\left\Vert \varphi\right\Vert }$-Tiefe
und $\left\Vert \varphi\right\Vert n^{\left\Vert \varphi\right\Vert }$-Größe
existiert, die die gleiche Anfrage beschreibt.

Sei $\eta$ eine von $\sigma$ disjunkte Signatur und $\Upsilon$
ein $\mathcal{K}$-uniformes $\eta$-Orakel.

Dann existiert für jede $\left(\mathcal{L}+\Upsilon+C\right)\left[\sigma\right]$
eine ebensolche $\left(\sigma,\mathbb{B}\cup\left\{ \mathtt{MAJ}\right\} \right)$-Schaltkreisfamilie
mit der Tiefe $\left\Vert \varphi\right\Vert \left(2n+1\right){}^{t\left\Vert \varphi\right\Vert }+\left\Vert \varphi\right\Vert $
und der Größe $\left\Vert \varphi\right\Vert \left(2n+1\right)^{\left\Vert \varphi\right\Vert }+2n\left\Vert \varphi\right\Vert $. 
\end{lem}
Insbesondere folgt aus der Kombination mit Lemma \ref{lem:fo-circuit}
und Lemma \ref{lem:oracle-circuit} (mit $c=0$ und der Tiefenbeschränkung
$\mathcal{O}\left(1\right)$):

\begin{eqnarray*}
\mathrm{FO}+\mathbf{BIT}+C & \subseteq & \left(\mathrm{SAC}^{0}+\mathbf{MAJ}\right)^{\mathrm{LOGSPACE}}\\
\mathrm{FO}+\mathbf{ARB}+C & \subseteq & \left(\mathrm{SAC}^{0}+\mathbf{MAJ}\right)^{P/\mathrm{poly}}
\end{eqnarray*}
Wir werden in diesem Fall konkret mit $\exists^{\geqslant}$-Quantoren
arbeiten, denn per Satz \ref{prop:counting-equal} sind diese gleich
ausdrucksstark mit $\exists^{=}$-Quantoren und $\#$-Termen.
\begin{proof}
Sei $\varphi$ eine $\left(\mathcal{L}+\Upsilon+\exists^{\geqslant}\right)\left[\sigma\right]$-Formel,
und sei $n\in\mathbb{N}$ die Eingabegröße. Wir gehen analog zu Lemma
\ref{lem:oracle-circuit} für $\mathcal{L}+\Upsilon$ vor und erzeugen
einen $\left(\sigma\uplus\eta\uplus\left\{ \leqslant\right\} ,\mathbb{B}\cup\left\{ \mathtt{MAJ}\right\} \right)$-Schaltkreis
über $U'=\left[1,2n+1\right]$, wobei wir die folgende induktive Konstruktion
verwenden:

\begin{casenv}
\item Falls $\varphi\left(\bar{x}\right)$ eine $\left(\mathcal{L}+\Upsilon\right)\left[\sigma\right]$-Formel
ohne $\exists^{\geqslant}$-Quantor ist, so existiert der $\left(\sigma\uplus\eta\uplus\left\{ \leqslant\right\} ,\mathbb{B}\right)$-Schaltkreis
$\mathcal{C}_{2n+1}^{\varphi}$ schon per Voraussetzung.
\item Falls $\varphi\left(\bar{x}\right)=\exists^{\geqslant x_{i}}y_{j}\psi\left(\bar{y}\right)$
mit $\bar{x}\in\mathbf{var}^{k}$, so nehmen wir an, dass die Behauptung
für $\psi$ gilt. 

Sei $\mathcal{C}_{2n+1}^{\psi}$ der symmetrische Schaltkreis für
die Formel $\psi$ mit der Tiefe $t\leqslant\left\Vert \psi\right\Vert \left(2n+1\right)^{c\left\Vert \psi\right\Vert }+\left\Vert \psi\right\Vert $
und der Größe $s\leqslant\left\Vert \psi\right\Vert \left(2n+1\right)^{\left\Vert \psi\right\Vert }+2n\left\Vert \psi\right\Vert $.
\[
\mathcal{C}_{2n+1}^{\psi}=\left(G_{\psi},W_{\psi},\Sigma_{\psi},\Omega_{\psi},U'\right)
\]
Wir erzeugen den folgenden Schaltkreis $\mathcal{C}_{2n+1}^{\varphi}$:
\begin{eqnarray*}
\mathcal{C}_{2n+1}^{\varphi} & \coloneqq & \left(G,W,\Sigma,\Omega,U'\right)
\end{eqnarray*}
Es wird für jedes Tupel $\bar{t}\in\left[1,2n+1\right]^{k}$ ein neuer
Output $g_{\bar{t}}$ eingefügt; ferner werden $2n$ Konstanten eingefügt:
\begin{eqnarray*}
G & = & G_{\psi}\uplus\left\{ g_{\bar{t}}\mid\bar{t}\in U'^{k}\right\} \uplus\left\{ 0_{j},1_{j}\mid1\leqslant j\leqslant n\right\} \\
W & = & W_{\psi}\uplus W_{\mathrm{maj}}\uplus W_{\mathrm{pad}}
\end{eqnarray*}
Die neuen Outputs werden mit $\mathtt{MAJ}$ markiert, falls $t_{i}$
(der Wert der Variable $x_{i}$) einer der numerischen Werte (im Bereich
$\rho\left[0,n\right]=\left[n+1,2n+1\right]$) ist, und sonst mit
$\mathbf{0}$. 
\[
\begin{array}{ccc}
\Sigma\left(0_{j}\right)=\mathbf{0}, & \Sigma\left(0_{j}\right)=\mathbf{1}, & \Sigma\left(g_{\bar{t}}\right)=\begin{cases}
\mathtt{MAJ} & \mathrm{falls}\,\,t_{i}\in\rho\left[0,n\right]\\
\mathbf{0} & \mathrm{sonst}
\end{cases}\end{array}
\]
\[
\Omega\left(\bar{t}\right)=g_{\bar{t}}
\]
Seien $\tau_{1},\tau_{2}:U^{k}\rightarrow U^{k-1}$ die folgenden
Abbildungen, die den Wert $t_{i}$ aus $\bar{t}$ beziehungsweise
$u_{j}$ aus $\bar{u}$ entfernen:
\begin{eqnarray*}
\tau_{1}\left(t_{1},\cdots,t_{k}\right) & = & \left(t_{1},\cdots t_{i-1},t_{i+1},\cdots,t_{k}\right)\\
\tau_{2}\left(t_{1},\cdots,t_{k}\right) & = & \left(t_{1},\cdots t_{j-1},t_{j+1},\cdots,t_{k}\right)
\end{eqnarray*}
Es gilt also $\tau_{1}\left(\bar{t}\right)=\tau_{2}\left(\bar{u}\right)$
genau dann wenn $\bar{t}$ und $\bar{u}$ in den gemeinsamen freien
Variablen von $\varphi$ und $\psi$ übereinstimmen. Das Majority-Gate
$g_{\bar{t}}$ soll prüfen, ob $t_{i}\leqslant\rho f\left(\tau_{1}\left(\bar{t}\right)\right)$,
wobei $f:U^{k-1}\rightarrow\left[0,n\right]$ die Anzahl der $\psi$
erfüllenden Belegungen von $y_{j}$ bei der Belegung der übrigen freien
Variablen mit $\tau_{1}\left(\bar{t}\right)$ sei: 
\begin{eqnarray*}
f\left(\tau_{1}\left(\bar{t}\right)\right) & \coloneqq & \left|\left\{ u_{j}\mid\bar{u}\in q_{\mathcal{C}_{2n+1}^{\psi}}\left(\mathfrak{A}\uplus\rho\Upsilon\left(n\right)\right),\,\tau_{2}\left(\bar{u}\right)=\tau_{1}\left(\bar{t}\right)\right\} \right|
\end{eqnarray*}
Dazu wird jedes Gate $g_{\bar{t}}$ mit $t_{i}\in\left[n+1,2n+1\right]$
zunächst mit den entsprechenden Outputs von $\mathcal{C}_{2n+1}^{\psi}$
verknüpft: 
\[
W_{\mathrm{maj}}=\left\{ \left(\Omega_{\psi}\left(\bar{u}\right),g_{\bar{t}}\right)\mid\bar{t},\bar{u}\in U'^{k},\,\tau_{1}\left(\bar{t}\right)=\tau_{2}\left(\bar{u}\right),\,u_{j}\in\left[1,n\right]\right\} 
\]
Momentan hat es genau $n$ Eingänge (für jeden Wert $u_{j}\in\left[1,n\right]$).
Daher gibt es $1$ aus, wenn $f\left(\bar{t}\right)\geqslant\frac{n}{2}$.
Es soll aber berechnen, ob $f\left(\bar{t}\right)\geqslant\rho^{-1}t_{i}$.
Dazu fügen wir die Kanten $W_{\mathrm{pad}}$ ein, um die Eingänge
der Majority-Gates mit Konstanten aufzufüllen. Ein Majority-Gate mit
$k$ zusätzlichen $\mathbf{0}$-Eingängen entscheidet $f\left(\bar{t}\right)\geqslant\frac{n+k}{2}$,
eines mit $k'$ zusätzlichen $\mathbf{1}$-Eingängen berechnet $f\left(\bar{t}\right)\geqslant\frac{n-k'}{2}$.
Es folgt:
\begin{eqnarray*}
\rho^{-1}t_{i} & \in & \left\{ \frac{n+k}{2},\frac{n-k'}{2}\right\} \\
k=2\rho^{-1}t_{i}-n & \mbox{oder} & k'=n-2\rho^{-1}t_{i}
\end{eqnarray*}
Jedes Majority-Gate $g_{\bar{t}}$ muss entweder $k=2\rho^{-1}t_{i}-n$
$\mathbf{0}$-Eingänge oder $k'=n-2\rho^{-1}t_{i}$ $\mathbf{1}$-Eingänge
erhalten:
\begin{eqnarray*}
W_{\mathrm{pad}} & = & \left\{ \left(0_{j},g_{\bar{t}}\right)\mid\bar{t}\in U'^{k},\,t_{i}\in\rho\left[0,n\right],\,1\leqslant j\leqslant2\rho^{-1}t_{i}-n\right\} \\
 & \cup & \left\{ \left(1_{j},g_{\bar{t}}\right)\mid\bar{t}\in U'^{k},\,t_{i}\in\rho\left[0,n\right],\,1\leqslant j\leqslant n-2\rho^{-1}t_{i}\right\} 
\end{eqnarray*}
\begin{eqnarray*}
\overset{n+k=2\rho^{-1}t_{i}}{\overbrace{x_{1}\cdots x_{n}\underset{k=2\rho^{-1}t_{i}-n}{\underbrace{\mathbf{0}\mathbf{0}\cdots\mathbf{0}}}}} &  & \overset{n+k'=2n-2\rho^{-1}t_{i}}{\overbrace{x_{1}\cdots x_{n}\underset{k'=n-2\rho^{-1}t_{i}}{\underbrace{\mathbf{1}\mathbf{1}\cdots\mathbf{1}}}}}
\end{eqnarray*}
Der neue Schaltkreis hat die Tiefe 
\[
t+1\leqslant\left\Vert \psi\right\Vert n^{c\left\Vert \psi\right\Vert }+\left\Vert \psi\right\Vert +1<\left\Vert \varphi\right\Vert n^{c\left\Vert \varphi\right\Vert }+\left\Vert \varphi\right\Vert 
\]
 und die Größe 
\begin{eqnarray*}
s+n^{k}+2n & \leqslant & \left(\left\Vert \psi\right\Vert +1\right)n^{\left\Vert \varphi\right\Vert }+2n\left(\left\Vert \psi\right\Vert +1\right)\\
 & \leqslant & \left\Vert \varphi\right\Vert n^{\left\Vert \varphi\right\Vert }+2n\left\Vert \varphi\right\Vert 
\end{eqnarray*}

\item \label{case:counting3}Falls $\varphi\left(\bar{x}\right)$ eine andere
$\left(\mathcal{L}+\Upsilon+\exists^{\geqslant}\right)\left[\sigma\right]$-Formel
ist, so seien $\psi_{1}\left(\bar{y}_{1}\right),\cdots,\psi_{m}\left(\bar{y}_{m}\right)$
alle größten\footnote{d.h. solche, die nicht selbst in einer Teilformel der gleichen Form
enthalten sind.} Teilformeln von $\varphi$ der Form $\psi_{i}=\exists^{\geqslant u_{i}}v_{i}\chi_{i}$.
Wir ersetzen diese Teilformeln durch neue Atome $R_{1}\bar{y}_{1},\cdots,R_{m}\bar{y}_{m}$
und erhalten eine $\left(\mathcal{L}+\Upsilon\right)\left[\sigma\uplus\left\{ R_{1},\cdots,R_{m}\right\} \right]$-Formel
$\varphi'$, die per Voraussetzung zu einer $\left(\sigma\uplus\eta\uplus\left\{ \leqslant\right\} \uplus\left\{ R_{1},\cdots,R_{m}\right\} ,\mathbb{B}\right)$-Schaltkreisfamilie
$\mathcal{C}_{2n+1}^{\varphi'}$ wird. Per Annahme existiert für jede
der Teilformeln $\psi_{1},\cdots,\psi_{m}$ ein Schaltkreis $\mathcal{C}_{2n+1}^{\psi_{i}}$.
(In der Implementierung fügen wir den Zählquantor effektiv als Subroutine
in die bestehende Konstruktion ein.)

Wir erzeugen den Schaltkreis $\mathcal{C}_{2n+1}^{\varphi}$ aus der
disjunkten Vereinigung wie folgt:
\begin{eqnarray*}
\mathcal{C}_{2n+1}^{\psi_{i}} & = & \left(G_{i},W_{i},\Sigma_{i},\Omega_{i},\left[1,2n+1\right]\right)\\
\mathcal{C}_{2n+1}^{\varphi'} & = & \left(G',W',\Sigma',\Omega',\left[1,2n+1\right]\right)\\
\mathcal{C}_{2n+1}^{\varphi} & \coloneqq & \left(G,W,\Sigma,\Omega',\left[1,2n+1\right]\right)
\end{eqnarray*}
\begin{eqnarray*}
G & \coloneqq & G'\uplus\biguplus_{i=1}^{m}G_{i}\\
W & \coloneqq & W'\uplus\biguplus_{i=1}^{m}W_{i}\uplus\\
 &  & \uplus\left\{ \left(\Omega_{i}\left(\bar{t}\right),g\right)\mid g\in G',\,\Sigma'\left(g\right)=R_{i}\bar{t}\right\} 
\end{eqnarray*}
Die Output-Gates der Schaltkreise $\mathcal{C}_{2n+1}^{\psi_{i}}$
werden mit den $R_{i}\bar{t}$-Inputs des Schaltkreises $\mathcal{C}_{2n+1}^{\varphi'}$
verbunden, und deren Beschriftung zu $\mathtt{AND}$ geändert. 
\begin{eqnarray*}
\Sigma\left(g\right) & \coloneqq & \begin{cases}
\mathtt{AND} & \mathrm{falls}\,g\in G',\,\Sigma'\left(g\right)=R_{i}\bar{y}_{i}\\
\Sigma'\left(g\right) & \mathrm{sonst,\,falls}\,g\in G'\\
\Sigma_{i}\left(g\right) & \mathrm{sonst,\,falls}\,g\in G_{i}
\end{cases}
\end{eqnarray*}
Die Tiefe von $\mathcal{C}_{2n+1}^{\psi_{i}}$ ist $\left\Vert \psi_{i}\right\Vert \left(2n+1\right)^{c\left\Vert \psi_{i}\right\Vert }+\left\Vert \psi_{i}\right\Vert $
und die Größe ist $\left\Vert \psi_{i}\right\Vert \left(2n+1\right)^{\left\Vert \psi_{i}\right\Vert }+2n\left\Vert \psi_{i}\right\Vert $.
Der Schaltkreis $\mathcal{C}_{2n+1}^{\varphi'}$ hat per Voraussetzung
die Tiefe $\left\Vert \varphi'\right\Vert \left(2n+1\right)^{c\left\Vert \varphi'\right\Vert }$
und die Größe $\left\Vert \varphi'\right\Vert \left(2n+1\right)^{\left\Vert \varphi'\right\Vert }$.
Zusätzlich ist $\left\Vert \varphi\right\Vert \geqslant\left\Vert \varphi'\right\Vert ,\sum_{i=1}^{m}\left\Vert \psi_{i}\right\Vert $,
da die 

Daher hat der neue Schaltkreis die Tiefe
\begin{eqnarray*}
t' & \leqslant & \left\Vert \varphi'\right\Vert \left(2n+1\right)^{c\left\Vert \varphi'\right\Vert }+\max\left\Vert \psi_{i}\right\Vert \left(2n+1\right)^{c\left\Vert \psi_{i}\right\Vert }+\left\Vert \psi_{i}\right\Vert +1\\
 & \leqslant & \left\Vert \varphi\right\Vert \left(2n+1\right)^{c\left\Vert \varphi\right\Vert }+\left\Vert \varphi\right\Vert 
\end{eqnarray*}
\begin{eqnarray*}
s' & \leqslant & \left\Vert \varphi'\right\Vert \left(2n+1\right)^{\left\Vert \varphi'\right\Vert }+\sum_{i=1}^{m}\left(\left\Vert \psi_{i}\right\Vert \left(2n+1\right)^{\left\Vert \psi_{i}\right\Vert }+2n\left\Vert \psi_{i}\right\Vert \right)\\
 & \leqslant & \left\Vert \varphi\right\Vert \left(2n+1\right)^{\left\Vert \varphi\right\Vert }+2n\left\Vert \varphi\right\Vert 
\end{eqnarray*}

\end{casenv}
\begin{description}
\item [{Symmetrie:}] Der neue Schaltkreis ist offensichtlich nicht symmetrisch
bezüglich $\mathrm{Sym}_{\left[1,2n+1\right]}$, denn die Anzahl der
konstanten Vorgänger jedes Majority-Gates $g_{\bar{t}}$ hängt von
dem Wert $t_{i}$ ab. Allerdings ist er symmetrisch bezüglich der
Permutationen, die die Werte in $\left[n+1,2n+1\right]$ fixieren,
denn für alle Werte $t_{i}\in\left[1,n\right]$ ist das Gate $g_{\bar{t}}$
eine $\mathbf{0}$-Konstante. 

Sei $\pi\in\mathrm{Sym}_{U}$ beliebig, und $\pi'\coloneqq\pi\cup\mathbf{id}_{\left[n+1,2n+1\right]}$
deren Erweiterung auf $U'$. Nun erweitern wir den Automorphismus
$\hat{\pi}$ von $\mathcal{C}_{2n+1}^{\psi}$ wie folgt: 
\begin{eqnarray*}
\hat{\pi}\left(0_{i}\right) & \coloneqq & 0_{i}\\
\hat{\pi}\left(1_{i}\right) & \coloneqq & 1_{i}\\
\hat{\pi}\left(g_{\bar{t}}\right) & \coloneqq & g_{\pi'\bar{t}}
\end{eqnarray*}

\begin{enumerate}
\item Nun gilt für jede Kante $\left(\Omega_{\psi}\left(\bar{u}\right),g_{\bar{t}}\right)\in W_{\mathrm{maj}}$
mit $\bar{t},\bar{u}\in U'^{k}$ und $\tau_{1}\left(\bar{t}\right)=\tau_{2}\left(\bar{u}\right)$:
\begin{eqnarray*}
\left(\hat{\pi}\Omega_{\psi}\left(\bar{u}\right),\hat{\pi}g_{\bar{t}}\right) & \overset{\mathrm{Ann.}}{=} & \left(\Omega_{\psi}\left(\pi'\bar{u}\right),\hat{\pi}g_{\bar{t}}\right)\\
 & = & \left(\Omega_{\psi}\left(\pi'\bar{u}\right),g_{\pi'\bar{t}}\right)
\end{eqnarray*}
Und weil $\pi'\bar{t},\pi'\bar{u}\in U'^{k}$ und $\tau_{1}\left(\pi'\bar{t}\right)=\tau_{2}\left(\pi'\bar{u}\right)$
gilt, folgt:
\[
\left(\Omega_{\psi}\left(\pi'\bar{u}\right),g_{\pi'\bar{t}}\right)\in W_{\mathtt{MAJ}}
\]
\item Für jede Kante $\left(0_{j},g_{\bar{t}}\right)\in W_{\mathrm{pad}}$
(mit $\bar{t}\in U'^{k}$ und $t_{i}\in\left[n+1,2n+1\right]$, und
$j\in\left[1,2t_{i}-3n-2\right]$):
\begin{eqnarray*}
\left(\hat{\pi}0_{j},\hat{\pi}g_{\bar{t}}\right) & = & \left(0_{j},g_{\pi'\bar{t}}\right)
\end{eqnarray*}
Und weil $\pi't_{i}=t_{i}$, gilt immer noch $j\in\left[1,2\pi't_{i}-3n-2\right]$,
und daher $\left(\hat{\pi}0_{j},\hat{\pi}g_{\bar{t}}\right)\in W_{\mathrm{pad}}$.
Das gleiche gilt analog für die Kanten $\left(\hat{\pi}1_{j},\hat{\pi}g_{\bar{t}}\right)$.
\item Für $\Sigma$:
\begin{eqnarray*}
\Sigma\left(\hat{\pi}0_{j}\right) & = & \Sigma\left(0_{j}\right)\\
\Sigma\left(\hat{\pi}1_{j}\right) & = & \Sigma\left(1_{j}\right)\\
\Sigma\left(\hat{\pi}g_{\bar{t}}\right) & = & \Sigma\left(g_{\pi'\bar{t}}\right)\\
 & = & \begin{cases}
\mathtt{MAJ} & \mathrm{falls}\,\,\pi't_{i}\in\left[n+1,2n+1\right]\\
\mathbf{0} & \mathrm{sonst}
\end{cases}\\
 & = & \Sigma\left(g_{\bar{t}}\right)
\end{eqnarray*}
\item Für $\Omega$ ist $\hat{\pi}\Omega\left(\bar{t}\right)=\hat{\pi}g_{\bar{t}}=g_{\pi'\bar{t}}=\Omega\left(\pi'\bar{t}\right)$
per Definition gegeben.
\end{enumerate}
\end{description}
Nach der Erzeugung von $\mathcal{C}_{2n+1}^{\varphi}$ wird der Schaltkreis
auf die bereits beschriebene Weise zu $\dot{\mathcal{C}_{n}^{\varphi}}$
umgewandelt, in dem die nicht-konstanten Inputs mit Elementen aus
$\left[n+1,2n+1\right]$ zu Konstanten werden. Der neue Schritt fügt
keine nicht-konstanten Inputs hinzu, so dass sich hierbei nichts ändert.

Da der Schaltkreis $\mathcal{C}_{2n+1}^{\varphi}$ für jede Permutation
$\pi\cup\mathbf{id}_{\left[n+1,2n+1\right]}$ mit $\pi\in\mathrm{Sym}_{\left[1,n\right]}$
einen Automorphismus besitzt, ist der Schaltkreis $\dot{\mathcal{C}_{n}^{\varphi}}$
symmetrisch (siehe Lemma \ref{lem:oracle-circuit}).
\end{proof}

\section{Fixpunktlogik}

Die Fixpunktlogik $\mathrm{LFP}\left[\sigma\right]$ ist echt ausdrucksstärker
als die Logik $\mathrm{FO}\left[\sigma\right]$. Beispielsweise ist
die Erreichbarkeit über einen Pfad beliebiger Länge nicht erststufig
definierbar \cite{Immerman1982,Libkin2012}. In $\mathrm{LFP}\left[E\right]$
wird diese Klasse durch die folgende Formel definiert:
\[
\varphi\left(u,v\right)\coloneqq\left[\mathrm{lfp}_{R,\left(x,y\right)}\,\left(\exists z\,\left(E\left(x,z\right)\wedge R\left(z,y\right)\right)\vee x=y\right)\right]\left(u,v\right)
\]
Jede $\mathrm{LFP}\left[\sigma\right]$-Formel $\varphi$ ist aber
äquivalent zu einer Familie von $\mathrm{FO}\left[\sigma\right]$-Formeln
$\left(\varphi_{n}\right)_{n\in\mathbb{N}}$ mit einer konstanten
Anzahl Variablen, so dass $\varphi_{n}$ auf allen Strukturen der
Größe $n$ äquivalent zu $\varphi$ ist. (Dies ist vergleichbar zum
$k$-Variablen-Fragment der infinitären Logik $L_{\infty\omega}^{k}$,
welches die Fixpunktlogik einschließt\cite{Dawar1995c,Kolaitis1992,Kolaitis1992b}.
In unserer Definition hat jedoch jede einzelne Formel $\varphi_{n}$
eine endliche Länge.)

Leider sind diese Formeln noch zu lang: Wenn das Symbol $R$ mehr
als einmal in $\psi$ vorkommt, ist $\left\Vert \psi_{i+1}\right\Vert \geqslant2\left\Vert \psi_{i}\right\Vert $,
und daher wächst die Formellänge mit $\left\Vert \varphi\right\Vert \geqslant2^{n^{k}}$
exponentiell für $k=\mathrm{ar}\left(R\right)$. Mit $\mathcal{C}_{n}^{\varphi}=\mathcal{C}_{n}^{\varphi_{n}}$
könnten wir nicht die gewünschte $\mathrm{poly}\left(n\right)$-Größe
ableiten.

Die booleschen Schaltkreise haben jedoch nicht dieselbe Einschränkung
wie $\mathrm{FO}\left[\sigma\right]$: Die Berechnung einer Teilformel
$\psi_{i}\left(\bar{z}\right)$ kann mehrfach verwendet werden, ohne
den Schaltkreis $\mathcal{C}_{n}^{\psi_{i}}$ zu vervielfältigen.
\begin{lem}
\label{lem:lfp-circ}Sei $\sigma$ eine beliebige relationale Signatur,
und sei $\varphi$ eine parameter-freie $\mathrm{LFP}\left[\sigma\right]$-Formel.

Dann existiert eine $P$-uniforme symmetrische $\left(\sigma,\mathbb{B}_{\mathrm{std}}\right)$-Schaltkreisfamilie
$\bar{\mathcal{C}^{\varphi}}$ mit $\mathrm{poly}\left(n\right)$-Tiefe
und $\mathrm{poly}\left(n\right)$-Größe, so dass $q_{\bar{\mathcal{C}^{\varphi}}}\left(\mathfrak{A}\right)=q_{\varphi}\left(\mathfrak{A}\right)$
für alle $\mathfrak{A}\in\mathbf{FIN}\left(\sigma\right)$ gilt.
\end{lem}
In Kombination mit dem Rest dieses Kapitels folgt dann:
\begin{eqnarray*}
\mathrm{LFP}+\mathbf{BIT} & \subseteq & \mathrm{SBC}^{P}\\
\mathrm{LFP}+\mathbf{ARB} & \subseteq & \mathrm{SBC}^{P/\mathrm{poly}}\\
\mathrm{LFP}+\mathbf{BIT}+C & \subseteq & \left(\mathrm{SBC}+\mathbf{MAJ}\right)^{P}\\
\mathrm{LFP}+\mathbf{ARB}+C & \subseteq & \left(\mathrm{SBC}+\mathbf{MAJ}\right)^{P/\mathrm{poly}}
\end{eqnarray*}

\begin{proof}
Sei $\varphi$ eine $\mathrm{LFP}\left[\sigma\right]$-Formel, und
sei $n\in\mathbb{N}$ die Eingabegröße. Die Behauptung wird induktiv
über den Aufbau der Formel nachgewiesen.

\begin{casenv}
\item Falls $\varphi$ eine $\mathrm{FO}\left[\sigma\right]$-Formel ohne
Fixpunkt-Operator ist, dann existiert die Schaltkreisfamilie $\mathcal{C}_{n}^{\varphi}$
schon per Voraussetzung.
\item Falls $\varphi=\left[\mathrm{lfp}_{X,\bar{x}}\psi\right]\left(\bar{y}\right)$
für eine Relationsvariable $X/k\in\mathbf{var}_{2}$, eine $k$-stellige
parameterfreie $\mathrm{LFP}\left[\sigma\right]$-Formel $\psi\left(X,\bar{x}\right)$
und ein Tupel von $\mathrm{LFP}\left[\sigma\right]$-Termen $\bar{y}\in\mathbf{var}^{k}$,
so seien $\psi_{1}\left(\bar{y}_{1}\right),\cdots,\psi_{m}\left(\bar{y}_{m}\right)$
alle größten Teilformeln von $\psi$ der Form $\psi_{i}=\left[\mathrm{lfp}_{X_{i}\bar{x}_{i}}\varphi_{i}\right]\left(\bar{y}_{i}\right)$.
Wir ersetzen diese Teilformeln durch neue Atome $R_{1}\bar{y}_{1},\cdots,R_{m}\bar{y}_{m}$
und erhalten eine $\mathrm{FO}\left[\sigma\uplus\left\{ R_{1},\cdots,R_{m}\right\} \right]$-Formel
$\psi'$, die per Voraussetzung zu einer $\left(\sigma\uplus\left\{ R_{1},\cdots,R_{m}\right\} ,\mathbb{B}_{\mathrm{std}}\right)$-Schaltkreisfamilie
$\mathcal{C}_{n}^{\varphi'}$ mit konstanter Tiefe $t\in\mathbb{N}$
wird. Per Annahme existiert für jede der Teilformeln $\psi_{1},\cdots,\psi_{m}$
ein Schaltkreis $\mathcal{C}_{n}^{\psi_{i}}$. 
\[
\mathcal{C}_{n}^{\psi'}=\left(G_{\psi},W_{\psi},\Sigma_{\psi},\Omega_{\psi},U\right)
\]
Wir bauen einen neuen Schaltkreis für $\mathcal{C}_{n}^{\varphi}$
bauen, der den Schaltkreis $\mathcal{C}_{n}^{\psi}$ genau $n^{k}$-mal
iteriert.

\begin{description}
\item [{Schaltkreis:}] Es wird eine Sequenz $\left(\mathcal{D}^{i}\right)_{i\in\mathbb{N}}$
von $k$-stelligen $\left(\sigma,\mathbb{B}_{\mathrm{std}}\right)$-Schaltkreisen
definiert.
\[
\mathcal{D}^{i}\coloneqq\left(G_{i},W_{i},\Sigma_{i},\Omega_{i},U\right)
\]
 Für $i=0$ nehmen wir nur eine $\mathbf{0}$-Konstante:
\begin{eqnarray*}
\mathcal{D}^{0} & \coloneqq & \left(\left\{ g_{0}\right\} ,\emptyset,\Sigma_{0},\Omega_{0},U\right)\\
\Sigma_{0}\left(g_{0}\right)=\mathbf{0} &  & \Omega\left(\bar{t}\right)=g_{0}
\end{eqnarray*}
Für $i\in\mathbb{N}_{\geqslant1}$ fügen wir eine neue Kopie von $\mathcal{C}_{n}^{\psi}$
in $\mathcal{D}^{i}$ ein:
\begin{eqnarray*}
G_{i+1} & \coloneqq & G_{i}\uplus G_{\psi}\\
W_{i+1} & \coloneqq & W_{i}\uplus W_{\psi}\uplus W'
\end{eqnarray*}
$\mathcal{D}^{i}$ verbindet alle mit $X\bar{u}$ markierten Inputs
mit dem Output $\Omega_{i}\left(\bar{u}\right)$, und markiert sie
mit ,,$\mathtt{AND}$``. (Hier wäre ,,$\mathtt{OR}$`` äquivalent,
da jedes dieser Gates nur einen Vorgänger bekommt.) 
\[
\Sigma_{i+1}\left(g\right)\coloneqq\begin{cases}
\Sigma_{i}\left(g\right) & \mathrm{falls}\,\,g\in G_{i}\\
\mathtt{AND} & \mathrm{falls}\,\,g\in G_{\psi},\,\Sigma_{\psi}\left(g\right)=X\bar{u}\\
\Sigma_{\psi}\left(g\right) & \mathrm{sonst}
\end{cases}
\]
\[
W'\coloneqq\left\{ \left(\Omega_{i}\left(\bar{u}\right),g\right)\mid g\in G_{\psi},\,\Sigma_{\psi}\left(g\right)=X\bar{u}\right\} 
\]
Die Outputs von $\mathcal{C}_{n}^{\psi'}$ werden zu den Outputs von
$\mathcal{D}^{i+1}$: 
\[
\Omega_{i+1}\coloneqq\Omega_{\psi}
\]
Seien nun $\mathcal{C}_{n}^{\psi_{i}}=\left(G_{i},W_{i},\Sigma_{i},\Omega_{i},U\right)$
die Schaltkreise für jede $\mathrm{LFP}\left[\sigma\right]$-Formel
$\psi_{i}$. Wir vereinigen diese mit $\mathcal{C}_{n}^{\psi'}$,
benennen die mit $R_{i}\bar{t}$ beschrifteten Gates zu $\mathtt{AND}$
um und verbinden sie mit dem Output $\Omega_{i}\left(\bar{t}\right)$
aus dem Schaltkreis $\mathcal{C}_{n}^{\psi'}$, um den Schaltkreis
$\mathcal{C}_{n}^{\psi}$ zu erhalten.
\item [{Korrektheit:}] Es wird induktiv bewiesen, dass $\mathcal{D}^{i}$
auf $\mathfrak{A}\in\mathbf{FIN}\left(\sigma\right)$ die $\sigma$-Anfrage
$q_{\mathcal{D}^{i}}\left(\mathfrak{A}\right)=F^{i}\left(\emptyset\right)$
berechnet.
\[
q_{\mathcal{D}^{0}}\left(\mathfrak{A}\right)=F^{0}\left(\emptyset\right)=\emptyset
\]
\begin{eqnarray*}
q_{\mathcal{D}^{i+1}}\left(\mathfrak{A}\right) & = & q_{\mathcal{C}_{n}^{\psi}}\left(\mathfrak{A}\cup\left(A,\left(X\mapsto q_{\mathcal{D}^{i}}\left(\mathfrak{A}\right)\right)\right)\right)\\
 & \overset{\mathrm{Ann.}}{=} & q_{\mathcal{C}_{n}^{\psi}}\left(\mathfrak{A}\cup\left(A,\left(X\mapsto F^{i}\left(\emptyset\right)\right)\right)\right)\\
 & = & q_{\psi^{\left(X\mapsto F^{i}\left(\emptyset\right)\right)}}\left(\mathfrak{A}\right)\\
 & \overset{\mathrm{Def.\ref{def:lfp}}}{=} & F^{i+1}\left(\emptyset\right)
\end{eqnarray*}
Daher berechnet $\mathcal{C}_{n}^{\varphi}=\mathcal{D}^{n^{k}}$ die
Anfrage $q_{\mathcal{C}_{n}^{\varphi}}\left(\mathfrak{A}\right)=F^{n^{k}}\left(\emptyset\right)=F^{\infty}\left(\emptyset\right)$.
\item [{Symmetrie:}] Sei $\pi\in\mathrm{Sym}_{U}$ beliebig. Per Annahme
ist $\mathcal{C}_{n}^{\psi}$ symmetrisch, also existiert der Automorphismus
$\hat{\pi}$.
\end{description}
Sei $\hat{\pi}_{0}\coloneqq\mathbf{id}_{\left\{ g_{0}\right\} }$.
Per Annahme existiere für $i\in\mathbb{N}$ ein von $\pi$ induzierter
Automorphismus $\hat{\pi}_{i}$ in $\mathcal{D}^{i}$. Sei dann $\hat{\pi}_{i+1}$
die folgende Abbildung in $\mathcal{D}^{i+1}$:
\begin{eqnarray*}
\\
\hat{\pi}_{i+1}\left(g\right) & \coloneqq & \begin{cases}
\hat{\pi}_{i}\left(g\right) & \mathrm{falls}\,g\in G_{i}\\
\hat{\pi}\left(g\right) & \mathrm{sonst}
\end{cases}
\end{eqnarray*}

\begin{enumerate}
\item Es gilt $\left(\hat{\pi}_{i+1}\right)_{\mid G_{i}}=\hat{\pi}_{i}$,
und $\left(\hat{\pi}_{i+1}\right)_{\mid G_{\psi}}=\hat{\pi}$; damit
sind die Bedingungen für $W_{i}\cup W_{\psi}$, $\Sigma_{i}$, $\Omega_{i+1}=\Omega_{\psi}$
und alle internen Gates $g\in G_{\psi}$ mit $\Sigma_{\psi}\notin\left\{ X\bar{u}\mid\bar{u}\in U^{m}\right\} $
bereits erfüllt.
\item Für die ehemaligen Inputs mit $\Sigma\left(g\right)=X\bar{u}$ gilt
$\Sigma_{i+1}\left(\hat{\pi}g\right)=\Sigma_{i+1}\left(g\right)=\mathtt{AND}$.
\item Für jede neue Kante $\left(\Omega_{i}\left(\bar{u}\right),g\right)\in W'$
mit $g\in G_{\psi}$ und $\Sigma_{\psi}\left(g\right)=X\bar{u}$:
\begin{eqnarray*}
\left(\hat{\pi}_{i+1}\Omega_{i}\left(\bar{u}\right),\hat{\pi}_{i+1}g\right) & = & \left(\hat{\pi}_{i}\Omega_{i}\left(\bar{u}\right),\hat{\pi}g\right)\\
 & = & \left(\Omega_{i}\left(\pi\bar{u}\right),\hat{\pi}g\right)
\end{eqnarray*}
Per Definition ist $\Sigma_{\psi}\left(\hat{\pi}g\right)=X\pi\bar{u}$,
und daher:
\[
\left(\Omega_{i}\left(\pi\bar{u}\right),\hat{\pi}g\right)\in W'
\]
\end{enumerate}
Daher ist $\mathcal{D}^{i+1}$ ebenfalls symmetrisch, und es folgt
die Symmetrie von $\mathcal{C}_{n}^{\varphi}=\mathcal{D}^{n^{k}}$.
\begin{description}
\item [{Größe:}] Es wird induktiv bewiesen, dass $T\left(\mathcal{D}^{i}\right)\leqslant i\left(t+1\right)$,
und $\left|\mathcal{D}^{i}\right|\leqslant1+i\cdot s\left(n\right)$
\begin{eqnarray*}
T\left(\mathcal{D}^{0}\right) & = & 0\\
T\left(\mathcal{D}^{i+1}\right) & = & T\left(\mathcal{D}^{i}\right)+T\left(\mathcal{C}_{n}^{\psi}\right)+1\\
 & \leqslant & T\left(\mathcal{D}^{i}\right)+t+1\\
 & \overset{\mathrm{Ann.}}{\leqslant} & i\left(t+1\right)+t+1=\left(i+1\right)\left(t+1\right)
\end{eqnarray*}
\begin{eqnarray*}
\left|\mathcal{D}^{0}\right| & = & 1\\
\left|\mathcal{D}^{i+1}\right| & = & \left|\mathcal{D}^{i}\right|+\left|\mathcal{C}_{n}^{\psi}\right|\\
 & \leqslant & \left|\mathcal{D}^{i}\right|+s\left(n\right)\\
 & \overset{\mathrm{Ann.}}{\leqslant} & 1+i\cdot s\left(n\right)+s\left(n\right)=1+\left(i+1\right)s\left(n\right)
\end{eqnarray*}
Damit gilt $T\left(\mathcal{C}_{n}^{\varphi}\right)=T\left(\mathcal{D}^{n^{k}}\right)+\max_{i=1}^{m}T\left(\mathcal{C}_{n}^{\psi_{i}}\right)\in\mathrm{poly}\left(n\right)$
und $\left|\mathcal{C}_{n}^{\varphi}\right|=\left|\mathcal{D}^{n^{k}}\right|+\sum_{i=1}^{m}\left|\mathcal{C}_{n}^{\psi_{i}}\right|\in\mathrm{poly}\left(n\right)$.
\end{description}
\item Falls $\varphi$ eine andere Form hat, dann gehe analog zu Fall 2
vor: Ersetze die Fixpunkt-Operatoren durch atomare Formeln, erzeuge
den Schaltkreis der $\mathrm{FO}\left[\sigma\uplus\left\{ R_{1},\cdots,R_{m}\right\} \right]$-Formel
$\varphi'$ und die Schaltkreise der Fixpunkt-Formeln $\psi_{1},\cdots,\psi_{m}$,
und füge diese zusammen. Es gelten die gleichen Schranken für die
Größe und Tiefe.
\end{casenv}
\end{proof}


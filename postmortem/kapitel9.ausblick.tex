
\chapter{Ausblick}

Wir haben in Kapitel 5 gesehen, dass logische Formeln der ersten Stufe
mit beliebigen $\mathrm{LOGSPACE}$-berechenbaren disjunkten numerischen
Erweiterungen (insbesondere $\mathbf{BIT}$) sich durch symmetrische
$\mathrm{LOGSPACE}$-uniforme Schaltkreisfamilien beschreiben lassen. 

Für die Fixpunkt-Logik $\mathrm{LFP}$ mit $P$-berechenbaren numerischen
Prädikaten haben wir die gleiche Reduktion auf symmetrische $P$-uniforme
Schaltkreisfamilien gezeigt. In beiden Fällen konnte ein Zählquantor
durch Hinzunahme von Majority-Gates ausgedrückt werden, und beliebige
Erweiterungen (insbesondere $\mathbf{ARB}$) durch Ausweitung auf
$P/\mathrm{poly}$-Uniformität.

In Kapitel 7 haben wir gesehen, dass sich symmetrische, $P$-uniforme
Schaltkreisfamilien umgekehrt durch Fixpunkt-Logik mit disjunkter
Ordnungserweiterung definieren lassen. Auch hier wurde das Majority-Gate
auf einen Zählquantor abgebildet, und die $P/\mathrm{poly}$-uniforme
Schaltkreisklasse durch die $\mathbf{ARB}$-Erweiterung definiert.

Für die Logik der ersten Stufe haben wir dann den Kreis geschlossen
und Schaltkreise konstanter Tiefe auf $\mathrm{FO}+\mathbf{BIT}$
reduziert. Die Erweiterung auf $\mathrm{P/\mathrm{poly}}$ und $\mathrm{FO}+\mathbf{ARB}$
funktionierte problemlos analog zu der Fixpunkt-Logik, die Zähl-Erweiterung
$\mathrm{FO}+C$ jedoch nicht.

Zum Schluss haben wir in Kapitel 8 nachgewiesen, dass die symmetrischen
Schaltkreisfamilien (zumindest bei konstanter Tiefe) echt weniger
ausdrücken können als die allgemein invarianten Schaltkreisfamilien.
Dazu wurde der Satz von Hanf verwendet, und gezeigt, dass die Logik
$\mathrm{FO}+\mathbf{ARB}$ modulo der Struktur-Größe Hanf-lokal ist,
während dies für die arb-invariante Logik $\mathrm{inv}\left(\mathrm{FO}\oplus\mathbf{ARB}\right)$
nicht allgemein gilt \cite{AnMeScSe11}.

Vermutlich gelten ähnliche Beziehungen auch für die symmetrischen
Teile der übrigen Schaltkreisklassen $\mathrm{SBC}$, $\mathrm{AC}^{0}+\mathbf{MAJ}$
und $\mathrm{SBC}+\mathbf{MAJ}$. Ein Lokalitäts-Beweis bietet sich
hier aber vermutlich nicht an, denn insbesondere wäre die im letzten
Kapitel definierte Anfrage bereits in $\mathrm{FO}+\mathbf{BIT}+C$
definierbar:
\begin{eqnarray*}
\varphi & \coloneqq & \exists x\left(\exists^{=x}y\exists z\,\left(Eyz\vee Ezy\right)\wedge\exists x_{\max}\left(\log_{\leqslant}\left(x,x_{\max}\right)\wedge\forall y\left(x_{\max}\leqslant y\rightarrow x_{\max}=y\right)\right)\right)\\
\log_{\leqslant} & \coloneqq & \left\{ \left(m,n\right)\in\mathbb{N}^{2}\mid2^{m}\leqslant n\right\} 
\end{eqnarray*}



\chapter{Grenzen der symmetrischen Schaltkreisklassen}

Die symmetrischen $\mathrm{AC}^{0}$-Schaltkreisfamilien beschreiben
eine echte Teilmenge der Anfragen, die durch $\mathrm{AC}^{0}$ definierbar
sind.

Der Beweis benutzt die Charakterisierung aus Theorem \ref{thm:fo}
von symmetrischem $\mathrm{AC}^{0}$ durch die Logik $\mathrm{FO}+\mathbf{ARB}$,
die Charakterisierung von $\mathrm{AC}^{0}$ durch die Logik $\mathrm{inv}\left(\mathrm{FO}\oplus\mathbf{ARB}\right)$,
und ein Problem, dass die Ausdrucksstärke dieser beiden Logiken voneinander
trennt.
\begin{thm}
\textbf{\label{lem:fo-arb-ac0}Immerman (1987)}\cite{Immerman1987}\textbf{,
Makowsky (1997)\cite{Makowsky1997-FO}}

Die Klasse der $\mathrm{AC}^{0}$-definierbaren Anfragen ist äquivalent
zu der Klasse der $\mathrm{inv}\left(\mathrm{FO}\oplus\mathbf{ARB}\right)$-definierbaren
Anfragen.
\end{thm}
Für die Konstruktion der Formel (hier nur skizziert) wird jeder Schaltkreis
in eine alternierende Normalform der Tiefe $d$ gebracht, in der jedes
Gate $n^{k}$ Vorgänger hat. Die $n^{kd}$ Wege von einem Input zu
einem Output werden durch numerische Tupel kodiert (die Relation dieser
Wege ist ein numerisches Prädikat und daher in $\mathrm{inv}\left(\mathrm{FO}\oplus\mathbf{ARB}\right)$-Logik
verwendbar). Die Formel muss dann über jede der alternierenden Ebenen
des Schaltkreises quantifizieren (mit $\exists$ für $\mathtt{OR}$,
und $\forall$ für $\mathtt{AND}$), und berechnet so die Auswertung
von $\mathcal{C}$.

Da per Definition offensichtlich $\mathrm{SAC}^{0}\subseteq\mathrm{AC}^{0}$
gilt, können wir mit $\mathrm{FO}+\mathbf{ARB}\subseteq\mathrm{SAC}^{0}$
direkt ablesen, dass $\mathrm{FO}+\mathbf{ARB}$ in $\mathrm{inv}\left(\mathrm{FO}\oplus\mathbf{ARB}\right)$
enthalten ist (und beweisen somit Teil 1 des Theorems \ref{thm:fo-arb}):
Die Logik erster Stufe mit disjunktem $\mathbf{ARB}$-Orakel ist nicht
stärker als die mit nicht-disjunktem Orakel.

Für den echten Einschluss (Teil 2 des Theorems \ref{thm:fo-arb})
benötigen wir eine Anfrage, die in $\mathrm{inv}\left(\mathrm{FO}\oplus\mathbf{ARB}\right)$
beschreibbar ist, jedoch nicht in $\mathrm{FO}+\mathbf{ARB}$ beschreibbar
ist. Per Theorem \ref{thm:fo} ist diese Anfrage dann in $\mathrm{AC}^{0}$,
aber per Theorem \ref{thm:fo} nicht in $\mathrm{SAC}^{0}$ beschreibbar.

Dafür verwenden wir den Lokalitäts-Satz von Hanf.

\bigskip{}

\begin{thm}
\textbf{Der Satz von Hanf} (Theorem 2.4.1 in \cite{EbbinghausFlum})

Seien $\mathfrak{A},\mathfrak{B}\in\mathbf{FIN}\left(\sigma\right)$,
$m\in\mathbb{N}$ und $r\coloneqq3^{m}$, und sei $\#_{r}^{\mathfrak{A}}\left(a\right)\coloneqq\left|\left\{ b\in A\mid a\sim_{r}b\right\} \right|$
die Anzahl der $r$-ähnlichen Elemente zu $a$ in $\mathfrak{B}$
(und $\#_{r}^{\mathfrak{B}}\left(a\right)$ analog in $\mathfrak{B}$).
Sei $e\in\mathbb{N}$ so gewählt, dass $\max_{a\in A}\left(\left|N_{r}^{\mathfrak{A}}\left(a\right)\right|\right),\max_{b\in B}\left(\left|N_{r}^{\mathfrak{B}}\left(b\right)\right|\right)\leqslant e$.

Nun gelte für jedes Element $a\in A$ mindestens eine der folgenden
Aussagen:

\begin{enumerate}
\item Die Anzahl der $r$-ähnlichen Elemente (siehe Definition \ref{def:neighborhoods})
zu $a$ in $\mathfrak{A}$ und $\mathfrak{B}$ ist gleich.
\[
\#_{r}^{\mathfrak{A}}\left(a\right)=\#_{r}^{\mathfrak{B}}\left(a\right)
\]
\item Die Anzahl der $r$-ähnlichen Elemente zu $a$ in $\mathfrak{A}$
und in $\mathfrak{B}$ ist jeweils größer als $m\cdot e$.
\[
\#_{r}^{\mathfrak{A}}\left(a\right),\#_{r}^{\mathfrak{B}}\left(a\right)\geqslant m\cdot e
\]
\end{enumerate}
Dann gilt: $\mathfrak{A}\equiv_{m}\mathfrak{B}$ (siehe Definition
\ref{def:m-equiv}).
\end{thm}
\begin{lem}
Es existiert eine Graph-Anfrage $q$, die in $\mathrm{inv}\left(\mathrm{FO}\oplus\mathbf{ARB}\right)$-definierbar
ist, aber nicht in $\mathrm{FO}+\mathbf{ARB}$ beschreibbar ist.
\end{lem}
\begin{proof}
Sei $C\subseteq\mathbf{FIN}\left(E\right)$ die Klasse der Graphen
$\mathfrak{A}=\left(A,E^{\mathfrak{A}}\right)$ für die gilt: $A$
enthält weniger als $\left\lceil \mathrm{log}\left|A\right|\right\rceil $
viele nicht-isolierte Knoten.

Diese Anfrage ist durch einen $\mathrm{inv}\left(\mathrm{FO}\oplus\mathbf{ARB}\right)\left[E\right]$-Satz
definierbar (siehe dazu Abschnitt 4.4 von \cite{AMSS2012-locality}).

\begin{assumption*}
Es existiere ein beliebiger $\left(\mathrm{FO}+\mathbf{ARB}\right)\left[E\right]$-Satz
$\varphi_{C}$, der die Anfrage $q_{C}$ definiere.
\end{assumption*}
Sei $m\coloneqq\mathrm{qr}\left(\varphi_{C}\right)$ der Quantorenrang
von $\varphi_{C}$, und o.B.d.A. sei $m\geqslant9$. Wir definieren
die folgenden Graphen $\mathfrak{A},\mathfrak{B}\in\mathbf{FIN}^{\left(2^{m^{m}}\right)}\left(E\right)$
(siehe Abbildung \ref{fig:m-equiv}):
\begin{eqnarray*}
A,B & \coloneqq & \left[1,2^{m^{m}}\right]\\
E^{\mathfrak{A}} & \coloneqq & \left\{ \left(i,i+1\right)\mid i\in\left[1,m^{m}-2\right]\right\} \cup\left\{ \left(m^{m}-1,1\right)\right\} \\
E^{\mathfrak{B}} & \coloneqq & \left\{ \left(i,i+1\right)\mid i\in\left[1,m^{m}-1\right]\right\} \cup\left\{ \left(m^{m},1\right)\right\} 
\end{eqnarray*}
Beide Strukturen bestehen aus $2^{m^{m}}$ Knoten, $\mathfrak{A}$
enthält einen Kreis der Länge $m^{m}-2$, und $\mathfrak{B}$ einen
Kreis der Länge $m^{m}$. Damit gilt:
\[
\mathfrak{A}\in C,\,\mathfrak{B}\notin C
\]
\[
\mathfrak{A}\models\varphi_{C},\,\mathfrak{B}\not\models\varphi_{C}
\]
Offensichtlich besteht die $3^{m}$-Nachbarschaft jedes Knotens im
Kreis aus einer $3^{m}+1$ langen Kette, und die übrigen Knoten sind
isoliert. Damit sind alle $3^{m}$-Nachbarschaften kleiner als $e\coloneqq3^{m}+2$.
Ferner hat jeder Knoten im Kreis mindestens $m^{m}-1\geqslant m\cdot3^{m+7}-1>m\cdot e$
viele $3^{m}$-ähnliche Knoten in jedem Graphen, und jeder isolierte
Knoten ohnehin $2^{m^{m}}-m^{m}>m^{m}$ viele $3^{m}$-ähnliche Knoten.
Somit sind die Graphen $\mathfrak{A}$ und $\mathfrak{B}$ nach dem
Satz von Hanf $m$-äquivalent.

\begin{figure}
\begin{centering}
\tikzstyle{g}=[circle, draw, fill=black,
                        inner sep=0pt, minimum width=2pt]
%  Tutte's 8-cage
\begin{tikzpicture}[thick,scale=0.8]
    \draw  \foreach \x in {0,18,...,342}
    {
        (\x:1.5) node[g] {} edge[dotted] (\x+18:1.5)
    };
	\node (0:0) {$m^{m-1}$};
	\begin{scope}[shift={(0,-4)},xscale=2]
		\draw [dotted] (0,0) ellipse (2 and 2);
		\draw plot [only marks, mark=*, mark size=0.1, domain=0:5, samples=75] ({rnd*360}:{1+rnd});
	\end{scope}
	\node at (0,-4) {$2^{m^m}-m^m+1$};
\end{tikzpicture}\tikzstyle{g}=[circle, draw, fill=black,
                        inner sep=0pt, minimum width=2pt]
%  Tutte's 8-cage
\begin{tikzpicture}[thick,scale=0.8]
    \draw  \foreach \x in {0,18,...,342}
    {
        (\x:1.5) node[g] {} edge[dotted] (\x+18:1.5)
    };
	\node (0:0) {$m^m$};
	\begin{scope}[shift={(0,-4)},xscale=2]
		\draw [dotted] (0,0) ellipse (2 and 2);
		\draw plot [only marks, mark=*, mark size=0.1, domain=0:5, samples=75] ({rnd*360}:{1+rnd});
	\end{scope}
	\node at (0,-4) {$2^{m^m}-m^m$};
\end{tikzpicture}
\par\end{centering}
\caption{\label{fig:m-equiv}Zwei $m$-äquivalente Graphen $\mathfrak{A},\mathfrak{B}$}
\end{figure}

Der $\mathrm{FO}+\mathbf{ARB}\left[\sigma\right]$-Satz $\varphi$
wird nicht auf $\mathfrak{A}$ und $\mathfrak{B}$, sondern $\mathfrak{A}\uplus\mathbf{ARB}\left(n\right)$
und $\mathfrak{B}\uplus\mathbf{ARB}\left(n\right)$ ausgewertet. Wir
werden im folgenden sehen, warum das nicht hilft, denn nach Satz \ref{prop:m-equiv-closed}
ist die $m$-Äquivalenz unter disjunkter Vereinigung abgeschlossen.

Da $\mathfrak{A}\equiv_{m}\mathfrak{B}$, und offensichtlich $\mathbf{ARB}\left(n\right)\equiv_{m}\mathbf{ARB}\left(n\right)$,
folgt somit $\mathfrak{A}\uplus\mathbf{ARB}\left(n\right)\equiv_{m}\mathfrak{B}\uplus\mathbf{ARB}\left(n\right)$,
und es gilt, im Widerspruch zu der Annahme:
\[
\mathfrak{A}\models\varphi_{C}\Leftrightarrow\mathfrak{B}\models\varphi_{C}
\]
\end{proof}
\begin{prop}
\label{prop:m-equiv-closed}Seien $\sigma_{1},\sigma_{2}$ beliebige
Signaturen, $\mathfrak{A}\equiv_{m}\mathfrak{B}$ zwei beliebige $\sigma_{1}$-Strukturen
und $\mathfrak{A}'\equiv_{m}\mathfrak{B}'$ zwei beliebige $\sigma_{2}$-Strukturen.

Dann gilt $\mathfrak{A}\uplus\mathfrak{A}'\equiv_{m}\mathfrak{B}\uplus\mathfrak{B}'$.
\end{prop}
Um dies zu beweisen, verwenden wir den Satz von Ehrenfeucht, nach
dem die $m$-Äquivalenz von Strukturen gleichbedeutend ist mit der
Existenz einer Gewinnstrategie für Duplicator im Ehrenfeucht-Fraïssé-Spiel
mit $m$ Runden. Wir verwenden die Definition des EF-Spiels von Ebbinghaus
und Flum (Abschnitt 2.2 von \cite{EbbinghausFlum}).
\begin{proof}
Es wird per $\mathfrak{A}\equiv_{m}\mathfrak{B}$ und $\mathfrak{A}'\equiv_{m}\mathfrak{B}'$
vorausgesetzt, dass Duplicator eine Gewinnstrategie für jedes $m'\leqslant m$-Runden-Spiel
auf $\left(\mathfrak{A},\mathfrak{B}\right)$ und auf $\left(\mathfrak{A}',\mathfrak{B}'\right)$
hat. Im Spiel auf $\left(\mathfrak{A}\uplus\mathfrak{A}',\mathfrak{B}\uplus\mathfrak{B}'\right)$
geht Duplicator wie folgt vor:

\begin{itemize}
\item Falls Spoiler einen Knoten $a_{i}\in A$ oder $b_{i}\in B$ wählt,
so antworte mit dem Gewinnzug aus dem Spiel auf $\left(\mathfrak{A},\mathfrak{B}\right)$.
\item Falls Spoiler $a_{i}\in A'$ oder $b_{i}\in B'$ wählt, so antworte
mit dem Gewinnzug aus dem Spiel auf $\left(\mathfrak{A}',\mathfrak{B}'\right)$.
\end{itemize}
Nun betrachten wir die gespielten Tupel $\bar{a}\in\left(A\cup A'\right)^{m}$
und $\bar{b}\in\left(B\cup B'\right)^{m}$, und die induzierten Strukturen
$\left(\mathfrak{A}\uplus\mathfrak{A}'\right)_{\mid\bar{a}}$ und
$\left(\mathfrak{B}\uplus\mathfrak{B}'\right)_{\mid\bar{b}}$.

Wegen der Annahme induzieren $\bar{a}$ und $\bar{b}$ zwei partielle
Isomorphismen mit $\left(\mathfrak{A}_{\mid\bar{a}},\bar{a}_{\mid A}\right)\cong\left(\mathfrak{B}_{\mid\bar{b}},\bar{b}_{\mid B}\right)$
und $\left(\mathfrak{A}'_{\mid\bar{a}},\bar{a}_{\mid A'}\right)\cong\left(\mathfrak{B}'_{\mid\bar{b}},\bar{b}_{\mid B'}\right)$.
Nach Satz \ref{prop:iso-closed-disjoint} ist der Isomorphismus abgeschlossen,
so dass gilt:
\[
\left(\mathfrak{A}\uplus\mathfrak{A}'_{\mid\bar{a}},\bar{a}\right)\cong\left(\mathfrak{B}\uplus\mathfrak{B}'_{\mid\bar{b}},\bar{b}\right)
\]
Damit hat Duplicator das Spiel auf $\left(\mathfrak{A}\uplus\mathfrak{A}',\mathfrak{B}\uplus\mathfrak{B}'\right)$
gewonnen, und es folgt $\mathfrak{A}\uplus\mathfrak{A}'\equiv_{m}\mathfrak{B}\uplus\mathfrak{B}'$.
\end{proof}
Die hier dargestellte Folgerung lässt sich allgemein auf jedes Paar
von $m$-äquivalenten Strukturen erweitern, solange die Strukturen
die gleiche Größe haben.
\begin{cor}
Die Logik $\mathrm{FO}+\mathbf{ARB}$ mit disjunktem Orakel kann nur
Eigenschaften ausdrücken, die bezüglich gleich großer Strukturen Hanf-lokal
sind.
\end{cor}
Die Bedingung der gleichen Größe ist notwendig, denn offensichtlich
ist zum Beispiel die $\mathrm{FO}+\mathbf{ARB}$-definierbare Klasse
$\textsc{Even}$ nicht Hanf-lokal.


\chapter{Einleitung}

Wir modellieren Graphen und Datenbanken als Strukturen, die ein Schema
von relationalen Prädikaten über einem endlichen Universum interpretieren.
Eine Datenbank-Anfrage in einem solchen Schema wird als Funktion modelliert,
die jede endliche Struktur auf eine Relation abbildet.

Von besonderem Interesse ist die Datenkomplexität solcher Anfragen:
Die Zeit- und Platzkomplexität der Auswertung einer festen Anfrage
in Abhängigkeit von der Größe der eingegebenen Struktur. Wir betrachten
zwei Modelle, in denen alle beschreibbaren Anfragen eine beschränkte
Datenkomplexität besitzen:
\begin{enumerate}
\item Klassen von booleschen Schaltkreisfamilien $\left(\mathcal{C}_{n}\right)_{n\in\mathbb{N}}$
(mit $\mathtt{AND}$-, $\mathtt{OR}$-, $\mathtt{NOT}$-, und gegebenenfalls
Majority-Gates).
\item Formeln der Prädikatenlogik erster Stufe (beziehungsweise deren Erweiterungen).
\end{enumerate}
Boolesche Schaltkreise arbeiten per Definition auf einer geordneten
Struktur. Für ungeordnete Strukturen wird deshalb eine beliebige Ordnung
gewählt und gefordert, dass das Ergebnis bezüglich der Ordnung invariant
ist. Wir führen eine strukturelle Einschränkung (Symmetrie) der Schaltkreise
ein, die diese Invarianz garantiert.

Die Schaltkreisfamilie besteht aus einer unendlichen Sequenz von Schaltkreisen
für jede Eingabegröße $n\in\mathbb{N}$ ist. Ohne Einschränkung kann
eine solche Sequenz auch unentscheidbare Klassen von natürlichen Zahlen
kodieren. In der Praxis ist es daher erwünscht, dass die Schaltkreise
von einem effizienten Algorithmus berechnet werden: Ist zum Beispiel
jeder Schaltkreis in Polynomialzeit oder mit logarithmischem Platz
(in Abhängigkeit von $n$) berechenbar, dann nennen wir die Schaltkreisfamilie
$P$- beziehungsweise $\mathrm{LOGSPACE}$-uniform.

Die Logik erster Stufe ist in der Praxis zu eingeschränkt: Selbst
einfache Probleme wie die Frage, ob eine Struktur eine gerade Anzahl
von Elementen enthält, oder zwei Knoten durch einen Weg beliebiger
Länge verbunden sind, können nicht ausgedrückt werden.\cite{Libkin2012}
Daher erweitern wir die Logik um Fixpunkt-Operatoren, numerische Prädikate
und Zähler. Insbesondere betrachten wir Logiken, in denen die numerischen
Prädikate \emph{disjunkt} von dem Universum der eigentlichen Struktur
interpretiert werden, und weisen nach, dass solche Logiken gerade
die verschiedenen Klassen symmetrischer Schaltkreisfamilien charakterisieren.

Konkret ist die Klasse der symmetrischen $P$-uniformen booleschen
Schaltkreisfamilien $\mathrm{SBC}^{P}$ äquivalent zu der Logik erster
Stufe mit Fixpunkt-Operator und disjunktem Ordnungsprädikat $\mathrm{FP}+\mathbf{ORD}$.
Das Ergebnis wurde 2014 von Matthew Anderson und Anuj Dawar\cite{AD2014}
veröffentlicht (nach einem verwandten Ergebnis für infinitäre Logik
und lokal polynomiell beschränkte Schaltkreise von Martin Otto in
1997\cite{Otto1997}), und bildet die Grundlage dieser Arbeit.

Die logische Zählerweiterung $\mathrm{FP}+C$ charakterisiert die
symmetrische $P$-uniforme Schaltkreisklasse $\left(\mathrm{SBC}+\mathbf{MAJ}\right)^{P}$
mit einem Majority-Gate, und die Erweiterung der Logik um beliebige
numerische Prädikate $\mathrm{FP}+\mathbf{ARB}$ charakterisiert die
$P/\mathrm{poly}$-uniforme Klasse $\mathrm{SBC}^{P/\mathrm{poly}}$.
Diese beiden Erweiterungen sind miteinander kombinierbar.
\begin{thm}
\textbf{\label{thm:fp}Anderson und Dawar (2014)}

Die folgenden Paare von Anfragenklassen sind auf endlichen Strukturen
äquivalent:

\begin{enumerate}
\item $\mathrm{FP}+\mathbf{ORD}$ und $\mathrm{SBC}^{P}$
\item $\mathrm{FP}+\mathbf{ORD}+C$ und $\left(\mathrm{SBC}+\mathbf{MAJ}\right)^{P}$
\item $\mathrm{FP}+\mathbf{ARB}$ und $\mathrm{SBC}^{P/\mathrm{poly}}$
\end{enumerate}
\end{thm}
Das obige Ergebnis wird für Schaltkreise konstanter Tiefe angepasst.
\begin{thm}
\label{thm:fo}Die folgenden Paare von Anfragenklassen sind auf endlichen
Strukturen äquivalent:

\begin{enumerate}
\item $\mathrm{FO}+\mathbf{BIT}$ und $\left(\mathrm{SAC}^{0}\right)^{\mathrm{LOGSPACE}}$
\item $\mathrm{FO}+\mathbf{ARB}$ und $\left(\mathrm{SAC}^{0}\right)^{P/\mathrm{poly}}$
\end{enumerate}
\end{thm}
Die Einschränkung auf symmetrische Schaltkreisfamilien bringt im Allgemeinen
eine Reduktion der Ausdrucksstärke mit sich. So wird die Klasse $\mathrm{AC}^{0}$
durch die invariante Logik $\mathrm{inv}\left(\mathrm{FO}\oplus\mathbf{ARB}\right)$
(\cite{Immerman1987,Makowsky1997-FO}), und $P/\mathrm{poly}$ durch
$\mathrm{inv}\left(\mathrm{FP}\oplus\mathbf{ARB}\right)$ (\cite{Makowsky1997-FO,Makowsky1998-LFP})
charakterisiert.
\begin{thm}
\label{thm:fo-arb}Für die $\mathrm{FO}+\mathbf{ARB}$-Logik und die
arb-invariante $\mathrm{FO}\oplus\mathbf{ARB}$-Logik gilt:

\begin{enumerate}
\item Jede $\mathrm{FO}+\mathbf{ARB}$-Formel beschreibt eine Anfrage, die
durch eine arb-invariante $\mathrm{FO}\oplus\mathbf{ARB}$-Formel
definierbar ist.
\item Es existiert eine Anfrage, die in der arb-invarianten $\mathrm{FO}\oplus\mathbf{ARB}$-Logik,
aber nicht $\mathrm{FO}+\mathbf{ARB}$ definierbar ist.
\end{enumerate}
\end{thm}
Ähnliche Bezüge bestehen vermutlich auch zwischen den symmetrischen
und nicht-symmetrischen Teilen der übrigen betrachteten Schaltkreisklassen.


\chapter{Logik}

\section{Grundlagen der relationalen Logik}

Wir betrachten logische Sprachen auf relationalen Signaturen $\sigma$,
deren Ausdrücke auf endlichen $\sigma$-Strukturen ausgewertet werden.

Zunächst definieren wir $\mathbf{var}$ als die Menge aller erststufigen
Variablen. Für einen Ausdruck $\omega$ sei $\mathrm{var}\left(\omega\right)$
die Menge der darin vorkommenden Variablen, und $\mathrm{frei}\left(\omega\right)\subseteq\mathrm{var}\left(\omega\right)$
die Menge der freien Variablen.

In einer Struktur $\mathfrak{A}$ sei eine \textbf{Belegung} $\beta$
eine partielle Abbildung $\beta:\mathbf{var}\rightarrow A$ von Variablen
auf Elemente des Universums.

Eine Auswertungsfunktion für eine Struktur $\mathfrak{A}$ und eine
Belegung $\beta$ wird durch $\left\llbracket \cdot\right\rrbracket \left(\mathfrak{A},\beta\right)$
notiert.
\begin{defn}
Eine Logik $\mathcal{L}\left[\sigma\right]$ besteht aus der Sprache
der $\mathcal{L}\left[\sigma\right]$-Terme, der Sprache der $\mathcal{L}\left[\sigma\right]$-Formeln,
und einer Auswertungsfunktion $\left\llbracket \omega\right\rrbracket \left(\mathfrak{A},\beta\right)$
für jede Struktur $\mathfrak{A}$ und Belegung $\beta:X\rightarrow A$
und jeden Ausdruck $\omega$ mit $\mathrm{frei}\left(\omega\right)\subseteq X$.

Für einen $\mathcal{L}\left[\sigma\right]$-Term $t$ ist $\left\llbracket t\right\rrbracket \left(\mathfrak{A},\beta\right)\in A$
ein Element des Universums. Für eine $\mathcal{L}\left[\sigma\right]$-Formel
$\varphi$ ist $\left\llbracket \varphi\right\rrbracket \left(\mathfrak{A},\beta\right)\in\left\{ 0,1\right\} $
ein Wahrheitswert.
\end{defn}
\begin{notation}
Wir verwenden den Begriff ,,Ausdruck`` als Oberbegriff der Formeln
und Terme einer Logik, und bezeichnen Ausdrücke mit dem Buchstaben
$\omega$. Terme werden mit kleinen Buchstaben benannt, und Formeln
mit den Buchstaben $\varphi$, $\psi$ oder $\chi$. Für einstellige
Signaturen wie $\left\{ E\right\} $ oder $\left\{ \leqslant\right\} $
kürzen wir $\mathcal{L}\left[\left\{ R\right\} \right]$ durch $\mathcal{L}\left[R\right]$
ab.
\end{notation}
\begin{defn}
Für eine Formel $\varphi$ und ein Tupel $\bar{x}\in\mathbf{var}^{k}$
mit $\mathrm{frei}\left(\varphi\right)=\left\{ x_{1},\cdots,x_{k}\right\} $
und $\left|\left\{ x_{1},\cdots,x_{k}\right\} \right|=k$ nennen wir
das Tupel $\bar{x}=\left(x_{1}\cdots x_{k}\right)$ ein \textbf{Argument
}von $\varphi$. Verschiedene Argumente von $\varphi$ unterscheiden
sich nur in der Reihenfolge der Variablen. Durch die Notation $\varphi\left(\bar{x}\right)$
legen wir ein beliebiges Argument $\bar{x}$ für $\varphi$ fest.

Die \textbf{Stelligkeit }einer logischen Formel (beziehungsweise ihrer
Argumente) bezeichne die Anzahl der frei vorkommenden Variablen: Für
$\varphi\left(\bar{x}\right)$ gelte: 
\[
\mathrm{ar}\left(\varphi\right)=\mathrm{ar}\left(\bar{x}\right)=\left|\mathrm{frei}\left(\varphi\right)\right|
\]

Ein \textbf{Satz} sei eine Formel ohne freie Variablen.
\end{defn}
Mit $\mathtt{MF}\left(\omega\right)$ bezeichnen wir die maximale
Anzahl freier Variablen jedes Teilausdrucks von $\omega$.
\begin{defn}
Für eine $k$-stellige Formel $\varphi$ und eine Belegung $\beta:\mathrm{frei}\left(\varphi\right)\rightarrow A$
schreiben wir $\mathfrak{A}\models\varphi^{\beta}$ genau dann wenn
$\left\llbracket \varphi\right\rrbracket \left(\mathfrak{A},\beta\right)=1$.

Für $\varphi\left(\bar{x}\right)$ und $\bar{a}\in A^{\mathrm{ar}\left(\bar{x}\right)}$
schreiben wir $\left\llbracket \varphi\right\rrbracket \left(\mathfrak{A},\bar{a}\right)$
anstelle von $\left\llbracket \varphi\right\rrbracket \left(\mathfrak{A},\left(\bar{x}\mapsto\bar{b}\right)\right)$,
und $\mathfrak{A}\models\varphi\left[\bar{a}\right]$ anstelle von
$\mathfrak{A}\models\varphi^{\bar{x}\mapsto\bar{a}}$.

Entsprechend sei 
\[
q_{\varphi\left(\bar{x}\right)}\left(\mathfrak{A}\right)\coloneqq\left\{ \bar{a}\in A^{\mathrm{ar}\left(\bar{x}\right)}\mid\mathfrak{A}\models\varphi\left[\bar{a}\right]\right\} 
\]
die Relation aller $\varphi$ erfüllenden Tupel.
\end{defn}
Somit beschreibt jede $\mathcal{L}\left[\sigma\right]$-Formel $\varphi\left(\bar{x}\right)$
eine $\sigma$-Anfrage $q_{\varphi\left(\bar{x}\right)}$, und jeder
Satz eine $\sigma$-Eigenschaft. Da die Reihenfolge der Spalten der
Relation von der Wahl des Arguments $\bar{x}$ abhängt, wird es durch
$q_{\varphi\left(\bar{x}\right)}$ mit angegeben.

In manchen Fällen möchten wir einige Variablen einer Formel belegen
und sie erst dann als Anfrage auswerten.
\begin{defn}
Für eine Formel $\varphi$ und eine Belegung $\beta:X\rightarrow A$
mit $\mathrm{frei}\left(\varphi\right)\nsubseteq X$ sei $\varphi^{\beta}$
ein \textbf{partiell belegter} Ausdruck; es sei $\mathrm{frei}\left(\varphi^{\beta}\right)=\mathrm{frei}\left(\varphi\right)\backslash X$.

Für $\beta':\mathrm{frei}\left(\varphi^{\beta}\right)\rightarrow A$
bezeichne $\left\llbracket \varphi^{\beta}\right\rrbracket \left(\mathfrak{A},\beta'\right)$
die Auswertung $\left\llbracket \varphi\right\rrbracket \left(\mathfrak{A},\beta\cup\beta'\right)$.
Für ein Argument $\bar{x}'$ von $\varphi^{\beta}$ definieren wir
die folgende Anfrage: 
\[
q_{\varphi^{\beta}\left(\bar{x}'\right)}\coloneqq\left\{ \bar{a}\in A^{\ell}\mid\mathfrak{A}\models\varphi^{\beta}\left[\bar{a}\right]\right\} 
\]
\end{defn}
%
\begin{defn}
Für eine $\mathcal{L}\left[\sigma\right]$-Formel $\varphi$ und $n\in\mathbb{N}$
drücke $\models_{n}\varphi$ aus, dass diese von allen $\sigma$-Strukturen
$\mathfrak{A}\in\mathbf{FIN}^{\left(n\right)}\left(\sigma\right)$
der Größe $n$ unter allen Belegungen erfüllt wird. Die Notation $\models_{\mathrm{fin}}\varphi$
drücke aus, dass $\varphi$ von allen endlichen $\sigma$-Strukturen
$\mathfrak{A}\in\mathbf{FIN}\left(\sigma\right)$ unter allen Belegungen
erfüllt wird.

Falls $\models_{n}\left(\varphi\leftrightarrow\psi\right)$, so heißen
$\varphi$ und $\psi$ $n$-\textbf{äquivalent}. Insbesondere bedeutet
dies, dass $\varphi$ und $\psi$ die gleiche Anfrage $q_{\varphi\left(\bar{x}\right)}=q_{\psi\left(\bar{x}\right)}$
auf Strukturen der Größe $n$ definieren.
\end{defn}

\section{Die Logik erster Stufe}
\begin{defn}
\label{def:fo}Für eine relationale Signatur $\sigma$ sind die Syntax
und Semantik der Logik erster Stufe $\mathrm{FO}\left[\sigma\right]$
wie folgt definiert.
\end{defn}
\begin{labeling}{00.00.0000}
\item [{(TV)}] Für jede Variable $x\in\mathbf{var}$ ist $x$ ein $\mathrm{FO}\left[\sigma\right]$-Term.
\[
\mathrm{frei}\left(x\right)=\mathrm{var}\left(x\right)\coloneqq\left\{ x\right\} 
\]
\[
\left\llbracket x\right\rrbracket \left(\mathfrak{A},\beta\right)\coloneqq\beta x
\]
\item [{(AR)}] Für jedes Relationssymbol $R/k\in\sigma$ und jedes $k$-Tupel
von $\mathrm{FO}\left[\sigma\right]$-Termen $\bar{x}$ ist $R\bar{x}$
eine $\mathrm{FO}\left[\sigma\right]$-Formel.
\begin{eqnarray*}
\mathrm{frei}\left(R\bar{x}\right)\coloneqq\bigcup_{i=1}^{k}\mathrm{frei}\left(x_{i}\right) &  & \mathrm{var}\left(R\bar{x}\right)\coloneqq\bigcup_{i=1}^{k}\mathrm{var}\left(x_{i}\right)
\end{eqnarray*}
\begin{eqnarray*}
\left\llbracket R\bar{x}\right\rrbracket \left(\mathfrak{A},\beta\right) & \coloneqq & \left[R^{\mathfrak{A}}\right]\left(\left\llbracket x_{1}\right\rrbracket \left(\mathfrak{A},\beta\right),\cdots,\left\llbracket x_{k}\right\rrbracket \left(\mathfrak{A},\beta\right)\right)
\end{eqnarray*}
\item [{(AE)}] Für zwei $\mathrm{FO}\left[\sigma\right]$-Terme $x_{1},x_{2}$
ist $x_{1}\dot{=}x_{2}$ eine $\mathrm{FO}\left[\sigma\right]$-Formel.
\begin{eqnarray*}
\mathrm{frei}\left(x_{1}\dot{=}x_{2}\right)\coloneqq\bigcup_{i=1}^{2}\mathrm{frei}\left(x_{i}\right) &  & \mathrm{var}\left(x_{1}\dot{=}x_{2}\right)\coloneqq\bigcup_{i=1}^{2}\mathrm{var}\left(x_{i}\right)
\end{eqnarray*}
\begin{eqnarray*}
\left\llbracket x\dot{=}y\right\rrbracket \left(\mathfrak{A},\beta\right) & \coloneqq & \begin{cases}
1 & \mathrm{falls}\,\left\llbracket x\right\rrbracket \left(\mathfrak{A},\beta\right)=\left\llbracket y\right\rrbracket \left(\mathfrak{A},\beta\right)\\
0 & \mathrm{sonst}
\end{cases}
\end{eqnarray*}
\item [{(N)}] Für eine $\mathrm{FO}\left[\sigma\right]$-Formel $\varphi$
ist $\neg\varphi$ eine $\mathrm{FO}\left[\sigma\right]$-Formel.
\begin{eqnarray*}
\mathrm{frei}\left(\neg\varphi\right)\coloneqq\mathrm{frei}\left(\varphi\right) &  & \mathrm{var}\left(\neg\varphi\right)\coloneqq\mathrm{var}\left(\varphi\right)
\end{eqnarray*}
\[
\left\llbracket \neg\varphi\right\rrbracket \left(\mathfrak{A},\beta\right)\coloneqq1-\left\llbracket \varphi\right\rrbracket \left(\mathfrak{A},\beta\right)
\]
\item [{(J)}] Für $k\geqslant2$ $\mathrm{FO}\left[\sigma\right]$-Formeln
$\varphi_{1},\cdots,\varphi_{k}$ und einen Junktor $\gamma\in\left\{ \wedge,\vee,\rightarrow,\leftrightarrow\right\} $
(mit $k=2$ für $\gamma\in\left\{ \rightarrow,\leftrightarrow\right\} $)
ist auch $\left(\varphi_{1}\gamma\cdots\gamma\varphi_{k}\right)$
eine $\mathrm{FO}\left[\sigma\right]$-Formel. 
\begin{eqnarray*}
\mathrm{frei}\left(\varphi_{1}\gamma\cdots\gamma\varphi_{k}\right)\coloneqq\bigcup_{i=1}^{k}\mathrm{frei}\left(\varphi_{i}\right) &  & \mathrm{var}\left(\varphi_{1}\gamma\cdots\gamma\varphi_{k}\right)\coloneqq\bigcup_{i=1}^{k}\mathrm{var}\left(\varphi_{i}\right)
\end{eqnarray*}
\begin{eqnarray*}
\left\llbracket \varphi_{1}\wedge\cdots\wedge\varphi_{k}\right\rrbracket \left(\mathfrak{A},\beta\right) & \coloneqq & \min_{1\leqslant i\leqslant k}\left\llbracket \varphi_{i}\right\rrbracket \left(\mathfrak{A},\beta\right)\\
\left\llbracket \varphi_{1}\vee\cdots\vee\varphi_{k}\right\rrbracket \left(\mathfrak{A},\beta\right) & \coloneqq & \max_{1\leqslant i\leqslant k}\left\llbracket \varphi_{i}\right\rrbracket \left(\mathfrak{A},\beta\right)\\
\left\llbracket \varphi_{1}\rightarrow\varphi_{2}\right\rrbracket \left(\mathfrak{A},\beta\right) & \coloneqq & \left\llbracket \neg\varphi_{1}\vee\varphi_{2}\right\rrbracket \left(\mathfrak{A},\beta\right)\\
\left\llbracket \varphi_{1}\leftrightarrow\varphi_{2}\right\rrbracket \left(\mathfrak{A},\beta\right) & \coloneqq & \left\llbracket \left(\varphi_{1}\rightarrow\varphi_{2}\right)\wedge\left(\varphi_{2}\rightarrow\varphi_{1}\right)\right\rrbracket \left(\mathfrak{A},\beta\right)
\end{eqnarray*}
\item [{(Q)}] Für einen Quantor $Q\in\left\{ \exists,\forall\right\} $,
eine Variable $x\in\mathbf{var}$ und eine $\mathrm{FO}\left[\sigma\right]$-Formel
$\varphi$ ist $Qx\varphi$ eine $\mathrm{FO}\left[\sigma\right]$-Formel.
\begin{eqnarray*}
\mathrm{frei}\left(Qx\varphi\right)\coloneqq\mathrm{frei}\left(\varphi\right)\backslash\left\{ x\right\}  &  & \mathrm{var}\left(Qx\varphi\right)\coloneqq\mathrm{var}\left(\varphi\right)\cup\left\{ x\right\} 
\end{eqnarray*}
\begin{eqnarray*}
\left\llbracket \exists x\varphi\right\rrbracket \left(\mathfrak{A},\beta\right) & \coloneqq & \max_{a\in A}\left(\left\llbracket \varphi\right\rrbracket \left(\mathfrak{A},\beta_{\backslash\left\{ x\right\} }\cup\left(x\mapsto a\right)\right)\right)\\
\left\llbracket \forall x\varphi\right\rrbracket \left(\mathfrak{A},\beta\right) & \coloneqq & \min_{a\in A}\left(\left\llbracket \varphi\right\rrbracket \left(\mathfrak{A},\beta_{\backslash\left\{ x\right\} }\cup\left(x\mapsto a\right)\right)\right)
\end{eqnarray*}
Ohne Beschränkung der Allgemeinheit gelte für $Qx\varphi$ stets $x\in\mathrm{frei}\left(\varphi\right)$,
denn $Qx\varphi\equiv Qx\left(x\dot{=}x\wedge\varphi\right)$.
\end{labeling}
Wir kürzen $Qx_{1}\cdots Qx_{k}\varphi$ durch $Q\bar{x}\varphi$
und $\bigwedge_{i=1}^{k}\left(x_{i}=y_{i}\right)$ durch $\bar{x}=\bar{y}$
ab. Im folgenden werden wir im Allgemeinen Formeln ohne Implikations-Pfeile
betrachten.
\begin{defn}
Eine $\mathrm{FO}\left[\sigma\right]$-Formel sei \textbf{implikationsfrei
}wenn sie keine Teilformel der Form $\left(\varphi\rightarrow\psi\right)$
oder $\left(\varphi\leftrightarrow\psi\right)$ enthält.
\end{defn}
Weil $\left(\varphi\rightarrow\psi\right)\equiv\left(\neg\varphi\vee\psi\right)$
und $\left(\varphi\leftrightarrow\psi\right)\equiv\left(\varphi\wedge\psi\right)\vee\left(\neg\varphi\wedge\neg\psi\right)$,
sind alle $\mathrm{FO}\left[\sigma\right]$-Formeln äquivalent zu
implikationsfreien $\mathrm{FO}\left[\sigma\right]$-Formeln. Hierbei
entsteht ein fester Zuwachs in der Länge der Formel $\left\Vert \varphi\right\Vert $,
der die Datenkomplexität unberührt lässt.

\begin{defn}
\textbf{\label{def:qr}Quantorenrang}

Der Quantorenrang einer Formel sei die maximale Anzahl der geschachtelten
Quantoren:
\[
\begin{array}{cc}
\mathrm{qr}\left(R\bar{x}\right)=\mathrm{qr}\left(x_{1}\dot{=}x_{2}\right)=0 & \mathrm{qr}\left(\psi_{1}\gamma\cdots\gamma\psi_{k}\right)=\max_{i=1}^{k}\mathrm{qr}\left(\psi_{i}\right)\\
\mathrm{qr}\left(\exists x\psi\right)=\mathrm{qr}\left(\forall x\psi\right)=\mathrm{qr}\left(\psi\right)+1
\end{array}
\]
\end{defn}
%
\begin{defn}
\label{def:m-equiv}Zwei $\sigma$-Strukturen $\mathfrak{A}$ und
$\mathfrak{B}$ heißen $m$-äquivalent ($\mathfrak{A}\equiv_{m}\mathfrak{B}$,
wenn für alle $\mathrm{FO}\left[\sigma\right]$-Sätze $\varphi$ mit
$\mathrm{qr}\left(\varphi\right)\leqslant m$ gilt:
\[
\mathfrak{A}\models\varphi\Longleftrightarrow\mathfrak{B}\models\varphi
\]
\end{defn}

\section{Logiken mit Fixpunkt-Erweiterung}

Wir führen eine Erweiterung ein, die es erlaubt, den iterativen Fixpunkt
einer selbstreferenziellen logischen Formel zu definieren.
\begin{defn}
Eine Logik $\mathcal{L}$ erweitert die Logik $\mathcal{L}'$, wenn
sie die Syntax und Semantik von $\mathcal{L}'$ übernimmt, und zusätzliche
Produktionen einführt.
\end{defn}
Als erstes definieren wir einen neuen Typ von Variablen.
\begin{defn}
Sei $\mathbf{var}_{2}$ die Menge aller Relationsvariablen. Jede solche
Variable $X\in\mathbf{var}_{2}$ besitzt eine Stelligkeit $\mathrm{ar}\left(X\right)=k\in\mathbb{N}_{\geqslant1}$;
diese wird auch durch $X/k\in\mathbf{var}_{2}$ notiert. Die Funktionen
$\mathrm{frei}$ und $\mathrm{var}$, so wie die Belegungen $\beta$,
werden auf $\mathbf{var}\uplus\mathbf{var}_{2}$ erweitert.
\begin{eqnarray*}
\mathrm{frei}\left(\varphi\right),\mathrm{var}\left(\varphi\right) & \subseteq & \mathbf{var}\uplus\mathbf{var}_{2}\\
\beta:\mathrm{frei}\left(\varphi\right) & \rightarrow & A\cup\bigcup_{k\in\mathbb{N}}A^{k}\\
\beta X & \subseteq & A^{\mathrm{ar}\left(X\right)}\,\mathrm{f\ddot{u}r}\,X\in\mathbf{var}_{2}
\end{eqnarray*}
\end{defn}
Wir möchten syntaktisch garantieren, dass die Iteration der Formel
eine monoton wachsende Relation berechnet. Dafür wird verlangt, dass
die selbstreferenzielle Relationsvariable nicht negiert vorkommt.
\begin{defn}
\textbf{Positivität}

Wir definieren die (nicht disjunkten) Mengen der $X$-positiven und
$X$-negativen $\mathcal{L}\left[\sigma\right]$-Formeln für $X\in\mathbf{var}_{2}$
wie folgt:

\begin{itemize}
\item Jeder $\mathcal{L}\left[\sigma\right]$-Ausdruck $\omega$ mit $X\notin\mathrm{frei}\left(\omega\right)$
ist sowohl $X$-positiv als auch $X$-negativ.
\item Für jede $X$-positive Formel $\varphi$ ist $\neg\varphi$ $X$-negativ,
und umgekehrt.
\item Für einen Junktor $\gamma\in\left\{ \wedge,\vee\right\} $ und $k\geqslant2$
$X$-positive (beziehungsweise $X$-negative) Formeln $\varphi_{1},\cdots,\varphi_{k}$
ist $\left(\varphi_{1}\gamma\cdots\gamma\varphi_{k}\right)$ ebenfalls
$X$-positiv (beziehungsweise $X$-negativ).
\item Für einen Quantor $Q\in\left\{ \exists,\forall\right\} $, eine Variable
$x\in\mathbf{var}$ und eine $X$-positive (beziehungsweise $X$-negative)
Formel $\varphi$ ist $Qx\varphi$ ebenfalls $X$-positiv (beziehungsweise
$X$-negativ).
\end{itemize}
\end{defn}
Nun definieren wir die Fixpunkt-Erweiterung.
\begin{defn}
\label{def:lfp}Für eine Logik $\mathcal{L}$ und eine relationale
Signatur $\sigma$ erweitert $\mathrm{LFP}\left(\mathcal{L}\right)\left[\sigma\right]$
die Logik $\mathcal{L}\left[\sigma\right]$ wie folgt:
\end{defn}
\begin{labeling}{00.00.0000}
\item [{(AV)}] Für eine Relationsvariable $X/k\in\mathbf{var}_{2}$ und
ein $k$-Tupel $\bar{x}$ von $X$-positiven $\mathrm{LFP}\left(\mathcal{L}\right)\left[\sigma\right]$-Termen
ist $X\bar{x}$ eine $X$-positive $\mathrm{LFP\left(\mathcal{L}\right)}\left[\sigma\right]$-Formel.
\begin{eqnarray*}
\mathrm{frei}\left(X\bar{x}\right) & \coloneqq & \left\{ X\right\} \cup\bigcup_{i=1}^{k}\mathrm{frei}\left(x_{i}\right)\\
\mathrm{var}\left(X\bar{x}\right) & \coloneqq & \left\{ X\right\} \cup\bigcup_{i=1}^{k}\mathrm{var}\left(x_{i}\right)\\
\left\llbracket X\bar{x}\right\rrbracket \left(\mathfrak{A},\beta\right) & \coloneqq & \left[\beta X\right]\left(\beta\bar{x}\right)\\
 & \mathrm{mit} & \beta X\subseteq A^{k},\\
 &  & \beta\bar{x}\in A^{k}
\end{eqnarray*}
\item [{(LFP)}] Für eine Relationsvariable $X/k\in\mathbf{var}_{2}$, eine
$X$-positive $\mathrm{LFP}\left(\mathcal{L}\right)\left[\sigma\right]$-Formel
$\psi$, ein Tupel $\bar{x}\in\mathbf{var}^{k}$ und ein $k$-Tupel
$\bar{y}$ von $X$-positiven $\mathrm{\mathrm{LFP}\left(\mathcal{L}\right)}\left[\sigma\right]$-Termen
ist $\varphi=\left[\mathrm{lfp}_{X,\bar{x}}\psi\right]\left(\bar{y}\right)$
eine $X$-positive $\mathrm{LFP}\left(\mathrm{\mathcal{L}}\right)\left[\sigma\right]$-Formel.
\begin{eqnarray*}
\mathrm{frei}\left(\varphi\right) & = & \mathrm{frei}\left(\psi\right)\backslash\left\{ X,x_{1},\cdots,x_{k}\right\} \cup\bigcup_{i=1}^{k}\mathrm{frei}\left(y_{i}\right)\\
\mathrm{var}\left(\varphi\right) & = & \mathrm{var}\left(\psi\right)\cup\left\{ X,x_{1},\cdots,x_{k}\right\} \cup\bigcup_{i=1}^{k}\mathrm{var}\left(y_{i}\right)
\end{eqnarray*}
Für eine Belegung $\beta:\mathrm{frei}\left(\varphi\right)\rightarrow A$
prüft $\varphi=\left[\mathrm{lfp}_{X,\bar{x}}\psi\right]\left(\bar{y}\right)$,
ob die Belegung des Tupels $\bar{y}$ im kleinsten Fixpunkt des Relationssymbols
$X$ liegt. 
\begin{eqnarray*}
\beta' & \coloneqq & \beta_{\mid\mathrm{frei}\left(\psi\right)\backslash\left\{ X,x_{1},\cdots,x_{k}\right\} }\\
\left\llbracket \left[\mathrm{lfp}_{X,\bar{x}}\psi\right]\left(\bar{y}\right)\right\rrbracket \left(\mathfrak{A},\beta\right) & \coloneqq & \left\llbracket X\bar{y}\right\rrbracket \left(\mathfrak{A},\beta\cup\left(X\mapsto\mathrm{lfp}_{X,\bar{x}}\left(\psi\right)\right)\right)
\end{eqnarray*}
Im folgenden wird eine Berechnung definiert, die den kleinsten Fixpunkt
$\mathrm{lfp}_{X,\bar{x}}\left(\psi\right)$ iterativ bestimmt.
\end{labeling}
Ohne Beschränkung der Allgemeinheit sei $\left\{ x_{1},\cdots,x_{k}\right\} \subseteq\mathrm{frei}\left(\psi\right)$,
denn analog zu Definition \ref{def:fo} ist $\psi\equiv\left(\bar{x}=\bar{x}\wedge\psi\right)$.
Die nicht durch den Operator gebundenen Variablen $P\coloneqq\mathrm{frei}\left(\psi\right)\backslash\left\{ x_{1},\cdots,x_{k}\right\} \subseteq\mathrm{frei}\left(\varphi\right)$
heißen \textbf{Parameter} der Fixpunkt-Operation.

Für eine Parameter-Belegung $\beta:P\rightarrow A$ sei $F_{\beta}$
die folgende Abbildung: 
\begin{eqnarray*}
F_{\beta} & : & 2^{A^{k}}\rightarrow2^{A^{k}}\\
F_{\beta}\left(Y\right) & \coloneqq & \left\{ \bar{a}\in A^{k}\mid\mathfrak{A}\models\psi^{\beta\cup\left(X\mapsto Y\right)}\left[\bar{a}\right]\right\} 
\end{eqnarray*}
Das heißt: $F_{\beta}\left(Y\right)$ ist das Anfrageergebnis von
$\varphi$ auf $\mathfrak{A}$ unter der Belegung der Parameter mit
$\beta$ und der Variable $X$ mit der Relation $Y$.

Aus der $X$-Positivität folgt nach \cite{Gurevich1986,Libkin2012}
die Monotonie von $F_{\beta}$
\[
A\subseteq B\Rightarrow F_{\beta}\left(A\right)\subseteq F_{\beta}\left(B\right)
\]
und daher induktiv die Existenz eines Fixpunkts $F^{\infty}\left(\emptyset\right)$,
der nach höchstens $\left|A^{k}\right|$ Schritten erreicht wird:
\[
\emptyset\subseteq F_{\beta}\left(\emptyset\right)\subseteq\cdots\subseteq F_{\beta}^{\infty}\left(\emptyset\right)\subseteq A^{k}
\]

Für jeden anderen Fixpunkt $Y'$ gilt $\emptyset\subseteq Y'$, und
per Induktion auch $F^{n}\left(\emptyset\right)\subseteq F^{n}\left(Y'\right)=Y'$.
Daher ist $F^{\infty}\left(\emptyset\right)=\mathrm{lfp}_{X,\bar{x}}\left(\psi\right)$
der kleinste Fixpunkt.

Im folgenden wird die Logik $\mathrm{LFP}\left(\mathrm{FO}\right)$
durch $\mathrm{LFP}$ abgekürzt. Ferner werden wir uns auf das \emph{parameterfreie}
Fragment der Logik beschränken, was (bis auf einen Zuwachs in der
Anzahl der Variablen und Länge der Formel) die Allgemeinheit nicht
einschränkt.
\begin{defn}
\label{def:fp-paramfree}Eine $\mathrm{LFP}\left[\sigma\right]$-Formel
$\varphi$ ist \textbf{parameterfrei}, falls der Fixpunkt-Operator
stets alle Variablen bindet - das heißt, für jede Teilformel der Form
$\left[\mathrm{lfp}_{X,\bar{x}}\psi\right]\left(\bar{y}\right)$ gilt:
\[
\mathrm{frei}\left(\psi\right)\subseteq\left\{ X,x_{1},\cdots,x_{\mathrm{ar}\left(X\right)}\right\} 
\]
\end{defn}
\begin{prop}
\label{prop:fp-paramfree}Jede $\mathrm{LFP}\left[\sigma\right]$-Formel
$\varphi$ kann (unter Zuwachs der Länge $\left\Vert \varphi\right\Vert $
und Größe $\left|\mathrm{var}\left(\varphi\right)\right|$) in eine
parameterfreie $\mathrm{LFP}\left[\sigma\right]$-Formel $\varphi'$
übersetzt werden.\cite{Zaid,Grohe2005,Dziembowski96bounded-variablefixpoint}
\end{prop}
\begin{example}
Die folgende $\mathrm{LFP}\left[E\right]$-Formel $\varphi\left(u,v\right)$
ist erfüllt, wenn $u$ und $v$ durch einen Weg beliebiger Länge verbunden
sind: 
\[
\varphi\left(u,v\right)\coloneqq\left[\mathrm{lfp}_{T,\left(x,y\right)}\,\left(\exists z\,\left(E\left(x,z\right)\wedge T\left(z,y\right)\right)\vee x\dot{=}y\right)\right]\left(u,v\right)
\]
\end{example}
\begin{defn}
Eine etwas robustere, aber äquivalente, Definition der Fixpunktlogik
verwendet den inflationären Fixpunkt-Operator. Anstelle der Positivität
einer Formel in einer Relationsvariable betrachten wir mit $\left[\mathrm{ifp}_{X,\bar{x}}\psi\right]\left(\bar{y}\right)$
implizit die Formel $\left[\mathrm{lfp}_{X,\bar{x}}\left(X\bar{x}\vee\psi\right)\right]\left(\bar{y}\right)$,
die uns eine inflationäre Iteration garantiert:
\[
F_{\beta}\left(Y\right)=Y\cup q_{\psi\left(\bar{x}\right)}^{\beta\cup\left(X\rightarrow Y\right)}\left(\mathfrak{A}\right)\supseteq Y
\]

In diesem Fall muss $\psi$ nicht mehr $X$-positiv sein.
\end{defn}

\section{Logiken mit numerischen Erweiterungen}

\subsection{Disjunkte Orakel}

Eine Erweiterung um ein numerisches Orakel fügt einer Logik eine Anzahl
von Relationssymbolen hinzu, die für $\mathfrak{A}\in\mathbf{FIN}\left(\sigma\right)$
eine feste Interpretation über einem von $A$ disjunkten numerischen
Universum $\left[0,\left|A\right|\right]$ erhalten.
\begin{defn}
\textbf{$\eta$-Orakel, $\mathcal{L}+\Upsilon$-Logik}

Sei $\eta$ eine relationale Signatur. Ein $\eta$-Orakel $\Upsilon:\mathbb{N}\rightarrow\mathbf{FIN}_{\leqslant}^{0}\left(\eta\right)$
ist eine Funktion, die jeder natürlichen Zahl $n$ eine geordnete
$\left(\eta\uplus\left\{ \leqslant\right\} \right)$-Struktur $\Upsilon\left(n\right)$
über $\left[0,n\right]$ zuweist.

Sei $\sigma$ eine von $\eta\uplus\left\{ \leqslant\right\} $ disjunkte
relationale Signatur und $\mathcal{L}$ eine Logik (zum Beispiel $\mathrm{FO}$
oder $\mathrm{LFP}$). Für ein $\eta$-Orakel $\Upsilon$ ist die
Syntax der $\left(\mathcal{L}+\Upsilon\right)\left[\sigma\right]$-Logik
die der Logik $\mathcal{L}\left[\sigma\uplus\eta\uplus\left\{ \leqslant\right\} \right]$.

Für eine endliche $\sigma$-Struktur $\mathfrak{A}$ mit $A\cap\left[0,\left|A\right|\right]=\emptyset$
und eine Belegung $\beta:\mathbf{var}\rightarrow A\uplus\left[0,\left|A\right|\right]$
werden $\left(\mathcal{L}+\Upsilon\right)$-Ausdrücke auf der disjunkten
Vereinigung von $\mathfrak{A}$ mit der entsprechenden Orakelstruktur
$\Upsilon\left(\left|A\right|\right)$ ausgewertet:
\[
\left\llbracket \varphi\right\rrbracket \left(\mathfrak{A},\beta\right)\coloneqq\left\llbracket \varphi\right\rrbracket \left(\mathfrak{A}\uplus\Upsilon\left(\left|A\right|\right),\beta\right)
\]
\end{defn}
\begin{notation}
Nach unserem Begriff der logischen Erweiterung ist die Bezeichnung
$\mathrm{LFP}\left(\mathcal{L}+\Upsilon\right)$ gleichbedeutend mit
$\mathrm{LFP}\left(\mathcal{L}\right)+\Upsilon$. Im folgenden werden
wir allgemein die erste Bezeichnung verwenden, aber weiterhin $\mathrm{LFP}\left(\mathrm{FO}\right)+\Upsilon$
durch $\mathrm{LFP}+\Upsilon$ abkürzen.
\end{notation}
\begin{defn}
\textbf{Uniformität}

Wir nennen ein $\eta$-Orakel \textbf{$\mathcal{K}$-uniform} (für
eine Komplexitätsklasse $\mathcal{K}$) wenn die Berechnung der Repräsentation
von $\Upsilon\left(n\right)$ bei Eingabe von $\overset{n}{\overbrace{1\cdots1}}$
in $\mathcal{K}$ ist.

Insbesondere sei $P$ die Klasse der $\mathrm{poly}\left(n\right)$-zeitbeschränkten,
und $\mathrm{LOGSPACE}$ die Klasse der $\mathcal{O}\left(\log n\right)$-platzbeschränkten
Turingmaschinen. $P/\mathrm{poly}$ sei die Klasse der $\mathrm{poly}\left(n\right)$-zeitbeschränkten
Turingmaschinen, die eine $\mathrm{poly}\left(n\right)$-beschränkte
Orakel-Eingabe erhalten.
\end{defn}
Wir definieren die folgenden drei numerischen Orakel für die reine
Ordnung, für Arithmetik, und für nicht berechenbare Prädikate.
\begin{defn}
\textbf{\label{def:ord}}Sei $\mathbf{ORD}:\mathbb{N}\rightarrow\mathbf{FIN}_{<}^{0}\left(\emptyset\right)$
ein $\emptyset$-Orakel, so dass das für $n\in\mathbb{N}$ die geordnete
$\emptyset$-Struktur gilt: 
\[
\mathbf{ORD}\left(n\right)\coloneqq\left(\left[0,n\right],\leqslant_{\mid\left[0,n\right]}\right)\in\mathbf{FIN}_{<}^{\left[0,n\right]}\left(\emptyset\right)
\]
\end{defn}
%
\begin{defn}
\textbf{\label{def:bit}}Sei $\mathbf{BIT}:\mathbb{N}\rightarrow\mathbf{FIN}_{<}^{0}\left(\left\{ \mathtt{BIT}\right\} \right)$
ein $\left\{ \mathtt{BIT}\right\} $-Orakel, wobei das Prädikat $\mathtt{BIT}\left(a,b\right)$
ausdrückt, dass das $b$te Bit der Binärdarstellung von $a$ den Wert
$1$ hat. 
\begin{eqnarray*}
\mathcal{N}_{\mathrm{bit}} & \coloneqq & \left(\mathbb{N},\leqslant,\mathtt{BIT}^{\mathcal{N}}\right)\\
\mathtt{BIT}^{\mathcal{N}} & \coloneqq & \left\{ \left(a,b\right)\in\mathbb{N}^{2}\mid a=\Sigma_{i=0}^{\left\lceil \log a\right\rceil }x_{i}2^{i},\,\bar{x}\in\left\{ 0,1\right\} ^{\left\lceil \log a\right\rceil },\,x_{b}=1\right\} \\
\mathbf{BIT}\left(n\right) & \coloneqq & \left(\mathcal{N}_{\mathrm{bit}}\right)_{\mid\left[0,n\right]}\in\mathbf{FIN}_{<}^{\left[0,n\right]}\left(\left\{ \mathtt{BIT}\right\} \right)
\end{eqnarray*}
\end{defn}
%
\begin{defn}
\label{def:arb}Sei $\eta_{\mathrm{arb}}=\left\{ R_{X}\mid X\in\bigcup_{k\in\mathbb{N}}2^{\mathbb{N}^{k}}\right\} $
eine unendliche Signatur, die für jede beliebige Relation $X\subseteq\mathbb{N}^{k}$
mit $k\in\mathbb{N}$ auf den natürlichen Zahlen ein Symbol $R_{X}/k$
enthält, und sei $\mathcal{N}_{\mathrm{arb}}$ die $\eta_{\mathrm{arb}}$-Struktur
über $\mathbb{N}$, die diese Symbole interpretiert. Dann sei $\mathbf{ARB}:\mathbb{N}\rightarrow\mathbf{FIN}_{<}^{0}\left(\eta_{\mathrm{arb}}\right)$
ein $\eta_{\mathrm{arb}}$-Orakel, das die endlichen Anfangsstücke
dieser Relationen ausgibt:
\begin{eqnarray*}
\mathcal{N}_{\mathrm{arb}} & \coloneqq & \left(\mathbb{N},\leqslant,\left(R_{X}\mapsto X\right)_{R_{X}\in\eta_{\mathrm{arb}}}\right)\\
\mathbf{ARB}\left(n\right) & \coloneqq & \left(\mathcal{N}_{\mathrm{arb}}\right)_{\mid\left[0,n\right]}\in\mathbf{FIN}_{<}^{\left[0,n\right]}\left(\eta_{\mathrm{arb}}\right)
\end{eqnarray*}

Hierbei ist leicht nachweisbar: $\mathbf{ORD}$ und $\mathbf{BIT}$
sind $\mathrm{LOGSPACE}$-uniform, und $\mathbf{ARB}$ ist $P/\mathrm{poly}$-uniform.
\end{defn}

\subsection{Zähl-Erweiterungen}

Wir führen mehrere syntaktische Erweiterungen der Logik $\mathrm{FO}+\Upsilon$
ein, die es erlauben, erfüllende Belegungen einer Variable zu zählen:
Den Zählterm $\#$, den Zählquantor $\exists^{=}$, und den Majority-Quantor
$\exists^{\geqslant}$. Es wird nachgewiesen, dass alle drei Erweiterungen
die gleiche Ausdrucksstärke modulo einem festen Zuwachs der Formellänge
haben.
\begin{defn}
\textbf{Zählterm $\#$} (wie in \cite{AD2014}, und Abschnitt 8.4.2
von \cite{EbbinghausFlum} für Fixpunktlogik)

Der Zählterm ist eine zusätzliche Termproduktion, die die erfüllenden
Belegungen einer Formel zählt.

Sei $\mathcal{L}$ eine beliebige Logik, $\eta$ eine relationale
Signatur, und $\Upsilon:\mathbb{N}\rightarrow\mathbf{FIN}_{<}\left(\eta\right)$
ein $\eta$-Orakel.

Die $\left(\mathcal{L}+\Upsilon+\#\right)$-Logik erweitert die $\left(\mathcal{L}+\Upsilon\right)$-Logik
um die folgende Regel:

\begin{labeling}{00.00.0000}
\item [{(TC)}] Für eine $\left(\mathcal{L}+\Upsilon+\#\right)\left[\sigma\right]$-Formel
$\varphi$ und eine Variable $x\in\mathbf{var}$ ist $\#x\varphi$
ein $\left(\mathcal{L}+\Upsilon+\#\right)\left[\sigma\right]$-Term.
\begin{eqnarray*}
\mathrm{frei}\left(\#x\varphi\right) & \coloneqq & \mathrm{frei}\left(\varphi\right)\backslash\left\{ x\right\} \\
\mathrm{var}\left(\#x\varphi\right) & \coloneqq & \mathrm{var}\left(\varphi\right)\cup\left\{ x\right\} 
\end{eqnarray*}
Auf einer endlichen Struktur $\mathfrak{A}\in\mathbf{FIN}^{\left(n\right)}\left(\sigma\right)$
mit $n\in\mathbb{N}$ und einer Belegung 
\[
\beta:\mathrm{frei}\left(\varphi\right)\backslash\left\{ x\right\} \rightarrow A\uplus\left[0,n\right]
\]
 sei 
\[
\left\llbracket \#x\varphi\right\rrbracket \left(\mathfrak{A},\beta\right)\coloneqq\left|\left\{ a\in A\mid\mathfrak{A}\models\varphi^{\beta\cup\binom{x}{a}}\right\} \right|
\]
die Anzahl der unterschiedlichen Werte $a\in A$, für die $\mathfrak{A}\models\varphi^{\beta\cup\binom{x}{a}}$
gilt.
\end{labeling}
\end{defn}
\begin{example*}
Diese Erweiterung erlaubt die Definition vieler arithmetischer Operatoren
durch Terme, wie zum Beispiel die positive Differenz:
\begin{eqnarray*}
t_{\mathrm{DIFF}}\left(x,y\right) & \coloneqq & \#_{z}\left(\neg z\leqslant x\wedge z\leqslant y\right)\\
\left\llbracket t_{\mathrm{DIFF}}\right\rrbracket \left(\mathfrak{A},\left(a,b\right)\right) & = & \max\left(b-a,0\right)
\end{eqnarray*}
\end{example*}
\begin{defn}
\textbf{Zählquantor }$\exists^{=}$ (wie in \cite{Schweikardt:2005:AFL:1071596.1071602},
hier aber mit einem disjunkten numerischen Universum)

Der Zählquantor ist eine zusätzlicher Quantor, der die Zahl der erfüllenden
Belegungen einer Formel mit einer Variable vergleicht.

Sei $\mathcal{L}$ eine beliebige Logik, $\eta$ eine relationale
Signatur, und $\Upsilon:\mathbb{N}\rightarrow\mathbf{FIN}_{<}^{\left[0,n\right]}\left(\eta\right)$
ein $\eta$-Orakel.

Die $\mathcal{L}+\Upsilon+\exists^{=}$-Logik erweitert die $\mathcal{L}+\Upsilon$-Logik
um die folgende Regel:

\begin{labeling}{00.00.0000}
\item [{(QC)}] Für eine $\left(\mathcal{L}+\Upsilon+\exists^{=}\right)\left[\sigma\right]$-Formel
$\varphi$ und zwei Variablen $x,y\in\mathbf{var}$ ist $\exists^{=y}x\varphi$
eine $\left(\mathcal{L}+\Upsilon+\exists^{=}\right)\left[\sigma\right]$-Formel.
\begin{eqnarray*}
\mathrm{frei}\left(\exists^{=y}x\varphi\right) & \coloneqq & \left\{ y\right\} \cup\left(\mathrm{frei}\left(\varphi\right)\backslash\left\{ x\right\} \right)\\
\mathrm{var}\left(\exists^{=y}x\varphi\right) & \coloneqq & \mathrm{var}\left(\varphi\right)\cup\left\{ x,y\right\} 
\end{eqnarray*}
Auf einer endlichen Struktur $\mathfrak{A}\in\mathbf{FIN}^{\left(n\right)}\left(\sigma\right)$
mit einer Belegung 
\[
\beta:\left(\mathrm{frei}\left(\varphi\right)\backslash\left\{ x\right\} \right)\cup\left\{ y\right\} \rightarrow A\cup\left[0,n\right]
\]
gelte: 
\[
\left\llbracket \exists^{=y}x\varphi\right\rrbracket \left(\mathfrak{A},\beta\right)\coloneqq\begin{cases}
1 & \mathrm{falls}\,\,\beta y=\left\{ a\in A\mid\mathfrak{A}\models\varphi^{\beta_{\backslash\left\{ x\right\} }\cup\binom{x}{a}}\right\} \\
0 & \mathrm{sonst}
\end{cases}
\]
\end{labeling}
\end{defn}
%
\begin{defn}
\textbf{Majority-Quantor }$\exists^{\geqslant}$ (wie in \cite{EbbinghausFlum}
Abschnitt 3.4, hier aber mit $\exists^{\geqslant x}$ für $x\in\mathbf{var}$
anstelle von unendlich vielen Quantoren $\left(\exists^{\geqslant n}\right)_{n\in\mathbb{N}}$)

Der Majority-Quantor funktioniert wie der Zählquantor und prüft, ob
die Zahl der erfüllenden Belegungen mindestens den Wert einer Variable
erreicht.

Sei $\mathcal{L}$ eine beliebige Logik, $\eta$ eine relationale
Signatur, und $\Upsilon:\mathbb{N}\rightarrow\mathbf{FIN}_{<}^{\left[0,n\right]}\left(\eta\right)$
ein $\eta$-Orakel.

Die $\mathcal{L}+\Upsilon+\exists^{\geqslant}$-Logik erweitert die
$\mathcal{L}+\Upsilon$-Logik um die folgende Regel:

\begin{labeling}{00.00.0000}
\item [{(QM)}] Für eine $\left(\mathcal{L}+\Upsilon+\exists^{\geqslant}\right)\left[\sigma\right]$-Formel
$\varphi$ und zwei Variablen $x,y\in\mathbf{var}$ ist $\exists^{\geqslant y}x\varphi$
eine $\left(\mathcal{L}+\Upsilon+\exists^{\geqslant}\right)\left[\sigma\right]$-Formel.
\begin{eqnarray*}
\mathrm{frei}\left(\exists^{\geqslant y}x\varphi\right)=\left\{ y\right\} \cup\left(\mathrm{frei}\left(\varphi\right)\backslash\left\{ x\right\} \right) &  & \mathrm{var}\left(\exists^{\geqslant y}x\varphi\right)=\mathrm{var}\left(\varphi\right)\cup\left\{ x,y\right\} 
\end{eqnarray*}
Auf einer endlichen Struktur $\mathfrak{A}\in\mathbf{FIN}^{\left(n\right)}\left(\sigma\right)$
mit einer Belegung 
\[
\beta:\left(\mathrm{frei}\left(\varphi\right)\backslash\left\{ x\right\} \right)\cup\left\{ y\right\} \rightarrow A\cup\left[0,n\right]
\]
 gilt: 
\[
\left\llbracket \exists^{\geqslant y}x\varphi\right\rrbracket \left(\mathfrak{A},\beta\right)\coloneqq\begin{cases}
1 & \mathrm{falls}\,\,\beta y\in\left[0,n\right],\,\beta y\leqslant\left\{ a\in A\mid\mathfrak{A}\models\varphi^{\beta_{\backslash\left\{ x\right\} }\cup\binom{x}{a}}\right\} \\
0 & \mathrm{sonst}
\end{cases}
\]
\end{labeling}
\end{defn}
\begin{prop}
\label{prop:counting-equal}Die Logiken $\mathcal{L}+\Upsilon+\#$
und $\mathcal{L}+\Upsilon+\exists^{=}$ und $\mathcal{L}+\Upsilon+\exists^{\geqslant}$
sind äquivalent, modulo eines festen Zuwachses in $\left\Vert \varphi\right\Vert $
und $\left|\mathrm{var}\left(\varphi\right)\right|$.
\end{prop}
\begin{proof}
Jede $\left(\mathcal{L}+\Upsilon+\#\right)\left[\sigma\right]$-Formel
$\varphi$ ist äquivalent zu einer $\left(\mathcal{L}+\Upsilon+\exists^{=}\right)\left[\sigma\right]$-Formel
$\varphi'$. Dazu ersetzen wir jede ,,pseudo-atomare`` Teilformel,
die einen Zählterm enthält, wie folgt:

\begin{casenv}
\item Falls $\varphi=y\dot{=}\#x\psi$ oder $\varphi=\#x\psi\dot{=}y$ für
$x,y\in\mathbf{var}$, so sei $\varphi'\coloneqq\exists^{=y}x\psi'$.
\item Falls $\varphi=\#x_{1}\psi_{1}\dot{=}\#x_{2}\psi_{2}$ für $x_{1},x_{2}\in\mathbf{var}$,
so sei $y\in\mathbf{var}\backslash\left(\mathrm{frei}\left(\psi_{1}\right)\cup\mathrm{frei}\left(\psi_{2}\right)\right)$
eine neue Variable, und
\[
\varphi'\coloneqq\exists y\left(\exists^{=y}x_{1}\psi_{1}'\wedge\exists^{=y}x_{2}\psi_{2}'\right)
\]
\item Falls $\varphi=R\bar{x}$ für $R/k\in\sigma\cup\eta\cup\left\{ \leqslant\right\} $
und ein $k$-Tupel von $\left(\mathcal{L}+\Upsilon+\#\right)\left[\sigma\right]$-Termen
$\bar{x}$, so sei $\bar{y}\in\left(\mathbf{var}\backslash\bigcup_{i=1}^{k}\mathrm{frei}\left(x_{i}\right)\right)^{k}$
ein Tupel von neuen Variablen, und:
\begin{eqnarray*}
\chi_{i} & \coloneqq & \begin{cases}
\exists^{=y_{i}}z_{i}\psi_{i}' & \mathrm{falls}\,x_{i}=\#z_{i}\psi_{i}\\
y_{i}=x_{i} & \mathrm{sonst}
\end{cases}\\
\varphi' & \coloneqq & \exists\bar{y}\left(R\bar{y}\wedge\bigwedge_{i=1}^{k}\chi_{i}\right)
\end{eqnarray*}
\end{casenv}
Jede $\left(\mathcal{L}+\Upsilon+\exists^{=}\right)\left[\sigma\right]$-Formel
$\varphi$ ist äquivalent zu einer $\left(\mathcal{L}+\Upsilon+\exists^{\geqslant}\right)\left[\sigma\right]$-Formel
$\varphi'$:

\begin{casenv}
\item Falls $\varphi=\exists^{=y}x\psi$, so sei $\psi'$ eine zu $\psi$
äquivalente $\left(\mathcal{L}+\Upsilon+\exists^{\geqslant}\right)\left[\sigma\right]$-Formel,
$z\in\mathbf{var}\backslash\mathrm{frei}\left(\psi\right)$ eine neue
Variable, und: 
\[
\varphi'\coloneqq\forall z\left(\exists^{\geqslant z}x\psi'\leftrightarrow\left(z\leqslant y\right)\right)
\]
Die Formel $\exists^{=y}x\psi$ ist mit $\beta\left(y\right)\in\left[0,n\right]$
erfüllt, genau dann wenn gilt: Die Formel $\exists^{\geqslant z}x\psi$
ist für alle $\beta\left(z\right)\in\left[0,n\right]$ erfüllt, genau
dann wenn $\beta\left(z\right)\leqslant\beta\left(y\right)$.
\end{casenv}
Schließlich ist jede $\left(\mathcal{L}+\Upsilon+\exists^{\geqslant}\right)\left[\sigma\right]$-Formel
$\varphi$ äquivalent zu einer $\left(\mathcal{L}+\Upsilon+\#\right)\left[\sigma\right]$-Formel
$\varphi'$:

\begin{casenv}
\item Falls $\varphi=\exists^{\geqslant y}x\psi$, so sei $\psi'$ eine
$\left(\mathcal{L}+\Upsilon+\exists^{\geqslant}\right)\left[\sigma\right]$-Formel
mit $\psi\equiv\psi'$, und:
\[
\varphi'\coloneqq y\leqslant\#x\psi
\]
\end{casenv}
\end{proof}
\begin{notation}
Im folgenden wird mit $\mathcal{L}+\Upsilon+C$ stets eine dieser
äquivalenten Zähl-Logiken bezeichnet. Die Logik $\mathcal{L}+\mathbf{ORD}+C$
bezeichnen wir kurz als $\mathcal{L}+C$.
\end{notation}
\begin{example}
Die Logik $\mathrm{FO}+C$ kann ausdrücken, dass die $\sigma$-Struktur
eine gerade Größe hat. Bekanntlich ist diese $\sigma$-Eigenschaft
weder durch $\mathrm{FO}$ auf geordneten, noch $\mathrm{LFP}$ auf
ungeordneten Strukturen definierbar\cite{EbbinghausFlum,Libkin2012}:

\[
\varphi_{\mathrm{EVEN}}\coloneqq\exists y\,\left(y=\#_{z}\left(\neg z\leqslant y\wedge z\leqslant\#_{z}\,z=z\right)\right)
\]
\end{example}

\subsection{Nicht-disjunkte Orakel}

Eine alternative Erweiterung definiert über einer Struktur $\mathfrak{A}$
eine beliebige Bijektion $\pi:\left[1,n\right]\rightleftarrows A$
und interpretiert dann die numerischen Prädikate $R/k\in\eta$ über
dem Universum der Struktur selbst. Diese Erweiterung bezeichnen wir
(in Anlehnung an die disjunkte Erweiterung $\mathcal{L}+\Upsilon$,
und die Konkatenation $\mathfrak{A}\oplus\mathfrak{B}$ von Strukturen
mit dem gleichen Universum) mit $\mathcal{L}\oplus\Upsilon$.
\begin{defn}
\textbf{$\mathcal{L}\oplus\Upsilon$-Logik}

Für eine relationale Signatur $\eta$, ein $\eta$-Orakel $\Upsilon:\mathbb{N}\rightarrow\mathbf{FIN}_{<}^{1}\left(\eta\right)$,
eine von $\eta$ disjunkte relationale Signatur $\sigma$ und eine
Logik $\mathcal{L}\left[\sigma\right]$ sei die Syntax von $\mathcal{L}\oplus\Upsilon\left[\sigma\right]$
gleich der Syntax von $\mathcal{L}\left[\sigma\uplus\eta\uplus\left\{ \leqslant\right\} \right]$.

Für $\mathfrak{A}\in\mathbf{FIN}\left(\sigma\right)$ mit $n=\left|A\right|$
sei $\pi:\left[1,n\right]\rightleftarrows A$ eine beliebige Bijektion.
Sei $\mathfrak{A}_{\pi}\in\mathbf{FIN}\left(\sigma\uplus\eta\right)$
definiert durch:
\begin{eqnarray*}
\mathfrak{A}_{\pi} & \coloneqq & \mathfrak{A}\oplus\pi\Upsilon\left(n\right)\\
 & = & \left(A,\left(R^{\mathfrak{A}}\right)_{R\in\sigma},\left(\pi R^{\Upsilon\left(n\right)}\right)\right)
\end{eqnarray*}

Eine $k$-stellige $\mathcal{L}\oplus\Upsilon\left[\sigma\right]$-Formel
$\varphi$ heiße \textbf{invariant}, wenn für jede Struktur $\mathfrak{A}\in\mathbf{FIN}\left(\sigma\right)$,
jedes Paar von Bijektionen $\pi,\pi'\in\mathrm{Bij}\left(\left[1,\left|A\right|\right],A\right)$
und jedes Tupel $\bar{a}\in A^{k}$ gilt:
\begin{eqnarray*}
\mathfrak{A}_{\pi}\models\varphi\left[\bar{a}\right] & \Longleftrightarrow & \mathfrak{A}_{\pi'}\models\varphi\left[\bar{a}\right]
\end{eqnarray*}

Wir bezeichnen mit $\mathrm{inv}\left(\mathcal{L}\oplus\Upsilon\right)\left[\sigma\right]$
die Sprache der invarianten Formeln der $\left(\mathcal{L}\oplus\Upsilon\right)\left[\sigma\right]$-Logik.
\end{defn}
%
\begin{notation}
Die Logik $\mathrm{inv}\left(\mathrm{FO}\oplus\mathbf{ARB}\right)$
mit dem Orakel $\mathbf{ARB}\left(n\right)\coloneqq\left(\mathcal{N}_{\mathrm{arb}}\right)_{\left[1,n\right]}$
(mit $\mathcal{N}_{\mathrm{arb}}$ und $\eta_{\mathrm{arb}}$ wie
in Definition \ref{def:arb}) bezeichnen wir auch als die ,,arb-invariante
$\mathrm{FO}\left(\mathbf{ARB}\right)$-Logik`` in Anlehnung an \cite{Schweikardt13ashort,AMSS2012-locality}.
\end{notation}
Es ist zu beachten, dass die hier betrachteten Orakel-Strukturen das
Universum $\left[1,n\right]$ haben, und nicht $\left[0,n\right]$.
Dies schließt unter anderem die Zählterm-Erweiterung $\mathcal{L}\oplus\Upsilon+\#$
aus, weil ein Zählterm nicht den Wert $0$ erhalten kann und seine
Auswertung nicht vollständig definiert ist. Die Zählquantor-Erweiterungen
$\mathcal{L}\oplus\Upsilon+\exists^{=}$ und $\mathcal{L}\oplus\Upsilon+\exists^{\geqslant}$
haben dieses Problem nicht.

Nach dem Satz von Trakhtenbrot ist die endlichen Erfüllbarkeit von
$\mathrm{FO}\left[\leqslant\right]$ unentscheidbar, und daher auch
die Invarianz und die Sprache $\mbox{inv}\left(\mathrm{FO}\oplus\mathbf{ARB}\right)\left[\sigma\right]$
in $\mathrm{FO}\oplus\mathbf{ARB}\left[\sigma\right]$.\cite{EbbinghausFlumThomas,Libkin2012,Schweikardt13ashort} 

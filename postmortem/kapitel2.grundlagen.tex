
\chapter{Grundlegende Definitionen}

\section{Notation}

Zunächst legen wir einige Notationen und Abkürzungen fest. Im folgenden
bezeichne $\mathbb{N}$ die Menge der natürlichen Zahlen einschließlich
$0$, $\mathbb{R}$ die Menge der reellen Zahlen, und ,,$\leqslant$``
(wenn nicht anders definiert) die natürliche Ordnung von $\mathbb{R}$
und $\mathbb{N}$. Mit ,,ex.``, ,,s.d.`` und ,,f.a.`` kürzen
wir gegebenenfalls ,,es existiert``, ,,so dass`` und ,,für alle``
ab.
\begin{defn}
\textbf{\label{def:tupel}Mengen und Tupel}

Mengen benennen wir im allgemeinen durch Großbuchstaben wie $A$,
$B$, $U$ oder $X$. Für eine Menge $A$ bezeichne $2^{A}\coloneqq\left\{ A'\mid A'\subseteq A\right\} $
die Menge aller Teilmengen von $A$.

Tupel benennen wir durch Kleinbuchstaben mit Balken wie $\bar{a}$,
$\bar{b}$, $\bar{u}$ oder $\bar{x}$. Die Stelligkeit $\mathrm{ar}\left(\bar{x}\right)$
eines Tupels $\bar{x}$ sei die Anzahl seiner Elemente, und ein $n$-stelliges
Tupel heiße kurz ,,$n$-Tupel``. Implizit gelte stets $\bar{x}=\left(x_{1},\cdots,x_{\left|\mathrm{ar}\left(x\right)\right|}\right)$.
Die Menge aller $k$-Tupel einer Menge $A$ sei $A^{k}$.

Für ein $m$-Tupel $\bar{x}$ und ein $n$-Tupel $\bar{y}$ sei $\bar{x}\bar{y}$
das $\left(m+n\right)$-Tupel $\left(x_{1},\cdots,x_{m},y_{1},\cdots,y_{n}\right)$.
Das $0$-stellige Tupel wird durch $\left\langle \right\rangle $
notiert.

Für $\bar{a}\in A^{k}$ und $A'\subseteq A$ sei $\bar{a}_{\mid A'}$
das Tupel, das aus $\bar{a}$ entsteht, wenn die Elemente $a_{i}\in A\backslash A'$
entfernt werden.
\end{defn}
%
\begin{defn}
\textbf{Intervall}

Ein endliches Intervall von natürlichen Zahlen wird durch $\left[a,b\right]$
abgekürzt:
\[
\left[a,b\right]\coloneqq\left\{ i\in\mathbb{N}\mid a\leqslant i\leqslant b\right\} 
\]

Ein Intervall von reellen Zahlen wird durch $\mathbb{R}_{\left[a,b\right]}$
oder $\mathbb{R}_{\left[a,b\right[}$ abgekürzt: 
\begin{eqnarray*}
\mathbb{R}_{\left[a,b\right]} & \coloneqq & \left\{ i\in\mathbb{R}\mid a\leqslant i\leqslant b\right\} \\
\mathbb{R}_{\left[a,b\right[} & \coloneqq & \left\{ i\in\mathbb{R}\mid a\leqslant i<b\right\} 
\end{eqnarray*}
\end{defn}
Als nächstes definieren wir Relationen als Mengen von Tupeln von Elementen
eines Universums. 
\begin{defn}
\textbf{\label{def:relation}Relation}

Für eine Menge $A$ und $k\in\mathbb{N}$ sei $R\subseteq A^{k}$
eine $k$-stellige Relation über $A$. Für jede Relation $R\subseteq A^{k}$
sei $\left[R\right]:A^{k}\rightarrow\left\{ 0,1\right\} $ die folgende
Funktion: 
\[
\left[R\right]\left(\bar{a}\right)\coloneqq\begin{cases}
1 & \mathrm{falls}\,\,\bar{a}\in R\\
0 & \mathrm{sonst}
\end{cases}
\]
\end{defn}
Für geordnete Mengen definieren wir die Operatoren $\min$ und $\max$,
die das kleinste und größte Element (sofern vorhanden) einer Menge
unter einer bestimmten Ordnung bezeichnen.
\begin{defn}
\textbf{Minimum und Maximum}

Für eine Ordnung $\preceq$ auf einer Menge $X$ seien $\min_{\preceq}\left(X\right)\in X$
und $\max_{\preceq}\left(X\right)\in X$ diejenigen Elemente (sofern
vorhanden), so dass für alle Elemente $z\in X$ gilt:
\[
\min_{\preceq}\left(X\right)\preceq z\preceq\max_{\preceq}\left(X\right)
\]
Für $X\subseteq\mathbb{R}$ wird die natürliche Ordnung $\leqslant$
nicht explizit notiert. Für eine Funktion $f:U\rightarrow\mathbb{R}$
und eine Menge $X\subseteq U$ bezeichnen $\min_{x\in X}f\left(x\right)\in\mathbb{R}$
und $\max_{x\in X}f\left(x\right)\in\mathbb{R}$ die folgende Abkürzungen:
\begin{eqnarray*}
\min_{x\in X}f\left(x\right) & \coloneqq & \min\left\{ f\left(x\right)\mid x\in X\right\} \\
\max_{x\in X}f\left(x\right) & \coloneqq & \max\left\{ f\left(x\right)\mid x\in X\right\} 
\end{eqnarray*}
Umgekehrt bezeichnet $\min_{f}U$ ein beliebiges Element $u\in U$
mit dem minimalen Wert $f\left(u\right)=\min_{x\in U}f\left(x\right)$,
und $\max_{f}U$ analog.
\end{defn}
%
\begin{defn}
\textbf{Asymptotische Klassen}

Für jede Funktion $f:\mathbb{N}\rightarrow\mathbb{R}$ definieren
wir die Funktionsklassen $\mathcal{O}\left(f\right)$ und $\Omega\left(f\right)$:

\end{defn}
\begin{itemize}
\item Es gelte $g\in\mathcal{O}\left(f\right)$ genau dann wenn ein $n_{0},c\in\mathbb{N}$
existieren, so dass für alle $n\geqslant n_{0}$ gilt: $g\left(n\right)\leqslant c\cdot f\left(n\right)$.
\begin{itemize}
\item Es gelte $g\in\Omega\left(f\right)$ genau dann wenn $f\in\mathcal{O}\left(g\right)$.
\end{itemize}
\end{itemize}
\begin{defn}
Für $f:\mathbb{R}\rightarrow\mathbb{R}$ und eine Klasse $\mathcal{F}$
sei $f\left(\mathcal{F}\right)\coloneqq\left\{ f\circ g\mid g\in\mathcal{F}\right\} $.
Zum Beispiel ist $f\in2^{\mathcal{O}\left(n\right)}$ genau dann wenn
$f\in\mathcal{O}\left(2^{kn}\right)$ für ein festes $k\in\mathbb{N}$.
\end{defn}

Wir legen mehrere einfache Operationen für Abbildungen fest, darunter
die Verkettung, Vereinigung disjunkter Definitionsbereiche und Reduktion
auf einen Teilbereich.
\begin{defn}
\textbf{Abbildung}

Für eine Abbildung $\pi:A\rightarrow B$ und $a\in A$ schreiben wir
statt $\pi\left(a\right)$ gegebenenfalls $\pi a$ ohne Klammern.

Jede Abbildung $\pi:A\rightarrow B$ wird auf natürliche Weise auf
Tupel, Teilmengen und Relationen von $A$ erweitert:
\begin{eqnarray*}
\pi\left(x_{1},\cdots,x_{k}\right) & \coloneqq & \left(\pi x_{1},\cdots,\pi x_{n}\right)\\
\pi\left\{ x_{1},\cdots,x_{n}\right\}  & \coloneqq & \left\{ \pi x_{1},\cdots,\pi x_{n}\right\} 
\end{eqnarray*}

Eine Abbildung $\pi:A\rightarrow B$ mit $A=\left\{ a_{1},\cdots,a_{n}\right\} $
schreiben wir gegebenenfalls extensional wie folgt auf. Für ein Tupel
$\bar{a}=\left(a_{1},\cdots,a_{n}\right)$ kürzen wir diese Abbildung
auch durch $\left(\bar{a}\mapsto\pi\bar{a}\right)$ ab. 
\begin{eqnarray*}
\pi & \coloneqq & \left(\begin{array}{c}
a_{1}\\
\pi a_{1}
\end{array}\cdots\begin{array}{c}
a_{n}\\
\pi a_{n}
\end{array}\right)\\
\pi & \coloneqq & \left(\bar{a}\mapsto\pi\bar{a}\right)
\end{eqnarray*}

Für zwei Abbildungen $\pi_{1}:B\rightarrow C$ und $\pi_{2}:A\rightarrow B$
sei $\pi_{1}\circ\pi_{2}:A\rightarrow C$ (kurz $\pi_{1}\pi_{2}$)
die folgende Abbildung: 
\[
\pi_{1}\pi_{2}\coloneqq\left(\bar{a}\mapsto\pi_{1}\left(\pi_{2}\left(\bar{a}\right)\right)\right)
\]

Für zwei Abbildungen $\pi:A\rightarrow B$ und $\pi':A'\rightarrow B'$
mit disjunkten Definitionsbereichen $A\cap A'=\emptyset$ sei $\pi''\coloneqq\pi\cup\pi'$
die folgende Abbildung: 
\begin{eqnarray*}
\pi'' & : & A\uplus A'\rightarrow B\cup B'\\
\pi''x & \coloneqq & \begin{cases}
\pi x & \mathrm{falls}\,\,x\in A\\
\pi'x & \mathrm{falls}\,\,x\in A'
\end{cases}
\end{eqnarray*}

Für $\pi:A\rightarrow B$ und $A'\subseteq A$ sei $\pi_{\mid A'}:A'\rightarrow B$
die Reduktion von $\pi$ auf eine Teilmenge des Definitionsbereichs,
und $\pi_{\backslash A'}:A\backslash A'\rightarrow B$ die Reduktion
auf das Komplement.

Es sei $\mathbf{id}$ die Identität mit $\mathbf{id}\left(a\right)=a$
für alle Elemente $a$. Mit $\mathbf{id}_{X}$ bezeichnen wir die
Identitätsfunktion auf einer Menge $X$.

Die Menge $\mathrm{Abb}\left(A,B\right)$ bezeichne alle Funktionen
$\pi:A\rightarrow B$, und $\mathrm{Bij}\left(A,B\right)$ bezeichne
für endliche $\left|A\right|=\left|B\right|=n$ alle $n!$ bijektiven
Abbildungen $\pi:A\rightleftarrows B$.
\end{defn}
%
\begin{defn}
\textbf{Permutation}

Eine Permutation von $U$ ist eine bijektive Abbildung $\pi:U\rightleftarrows U$.
Die Menge aller Permutationen $\mathrm{Bij}\left(U,U\right)$ bezeichnen
wir auch als $\mathrm{Sym}_{U}$. Diese bilden eine Symmetrie-Gruppe
bezüglich der Verkettung $\circ$ mit dem neutralen Element $\mathbf{id}_{U}$.

Es sei $\pi^{-1}$ die inverse Abbildung mit $\pi^{-1}\pi=\pi\pi^{-1}=\mathbf{id}_{U}$.

Eine Transposition sei eine Permutation, die zwei Elemente $u_{i}$
und $u_{j}$ vertauscht und alle anderen Elemente fixiert. Die Permutation
$\left(\begin{array}{cc}
u_{i} & u_{j}\\
u_{j} & u_{i}
\end{array}\right)\cup\mathbf{id}_{U\backslash\left\{ u_{i},u_{j}\right\} }$ wird kurz durch $\left(u_{i}u_{j}\right)$ notiert.
\end{defn}
%
\begin{defn}
\textbf{\label{def:orbit}Orbit}

In einer Permutationsgruppe $G\subseteq\mathrm{Sym}_{U}$ sei $\mathrm{Orb}_{G}\left(u\right)\coloneqq\left\{ \pi u\mid\pi\in G\right\} $
die Menge aller Elemente, auf die $u$ abgebildet wird.
\end{defn}

\section{Endliche relationale Strukturen}

Wir betrachten Anfragen und Eigenschaften auf Graphen und allgemeinen
endlichen Strukturen über eine beliebige relationale Signatur $\sigma$.
\begin{defn}
\textbf{Relationale Signaturen}

Eine relationale Signatur $\sigma$ ist eine Menge von Relationssymbolen.
Jedes Symbol $R\in\sigma$ hat eine feste Stelligkeit $\mathrm{ar}\left(R\right)=k\in\mathbb{N}_{\geqslant1}$.
Gegebenenfalls wird die Stelligkeit kompakt durch $R/k\in\sigma$
beziehungsweise $\sigma=\left\{ R_{1}/k_{1},\cdots,R_{k}/k_{k}\right\} $
notiert.
\end{defn}
%
\begin{defn}
\textbf{Endliche Strukturen}

Eine endliche $\sigma$-Struktur $\mathfrak{A}=\left(A,\left(R^{\mathfrak{A}}\right)_{R\in\sigma}\right)$
über einer Signatur $\sigma$ und einem endlichen nicht-leeren Universum
$A$ besteht aus einer Interpretation $R^{\mathfrak{A}}\subseteq A^{k}$
für jedes Symbol $R/k\in\sigma$. Strukturen benennen wir im Allgemeinen
durch die Frakturbuchstaben $\mathfrak{A}$ und $\mathfrak{B}$.

\end{defn}
\begin{itemize}
\item Für eine endliche Menge $U$ sei $\mathbf{FIN}^{U}\left(\sigma\right)$
die Menge aller $\sigma$-Strukturen über dem Universum $U$.
\begin{itemize}
\item Für $n\in\mathbb{N}_{\geqslant1}$ seien $\mathbf{FIN}^{\left(n\right)}\left(\sigma\right)$
die $\sigma$-Strukturen über einem beliebigen Universum der Größe
$n$.
\item Seien $\mathbf{FIN}\left(\sigma\right)\coloneqq\bigcup_{n\in\mathbb{N}_{\geqslant1}}\mathbf{FIN}^{\left(n\right)}\left(\sigma\right)$
die endlichen $\sigma$-Strukturen.
\end{itemize}
\end{itemize}
Die Signatur $\sigma$ kann in Ausnahmefällen unendlich sein; da wir
jedoch nur endlich repräsentierbare $\sigma$-Anfragen betrachten,
können diese sich nur auf eine endlichen Menge von Relationssymbolen
$\sigma'\subseteq_{\mathrm{fin}}\sigma$ beziehen.

Wir formalisieren die Interpretation als eine Abbildung $\square^{\mathfrak{A}}:\sigma\rightarrow\bigcup_{k\in\mathbb{N}}2^{\left(A^{k}\right)}$,
die jedem Symbol eine Relation der entsprechenden Stelligkeit zuweist.
Daher kann die Interpretation gegebenenfalls auch explizit durch $\mathfrak{A}=\left(A,\square^{\mathfrak{A}}\right)$
notiert werden, um die Zuordnung von Symbolen und Relationen zu verdeutlichen:
\[
\mathfrak{A}\coloneqq\left(A,\left(\begin{array}{c}
R_{1}\\
R_{1}^{\mathfrak{A}}
\end{array}\cdots\begin{array}{c}
R_{k}\\
R_{k}^{\mathfrak{A}}
\end{array}\right)\right)
\]

\begin{defn}
\textbf{Geordnete Strukturen}

Sei $\sigma$ eine relationale Signatur, die nicht das zweistellige
Symbol $\leqslant$ enthält.

Für $a,b\in\mathbb{N}$ sei 
\[
\mathbf{FIN}_{\leqslant}^{\left[a,b\right]}\left(\sigma\right)\subseteq\mathbf{FIN}^{\left[a,b\right]}\left(\sigma\cup\left\{ \leqslant\right\} \right)
\]
 die Menge der endlichen $\sigma\cup\left\{ \leqslant\right\} $-Strukturen
mit dem Universum $\left[a,b\right]$, wobei $\leqslant$ durch die
natürliche Ordnung von $\left[a,b\right]$ interpretiert wird, und
sei 
\[
\mathbf{FIN}_{\leqslant}^{a}\left(\sigma\right)\coloneqq\bigcup_{b\in\mathbb{N}}\mathbf{FIN}_{\leqslant}^{\left[a,b\right]}\left(\sigma\right)
\]
 die Menge aller endlichen geordneten $\sigma\cup\left\{ \leqslant\right\} $-Strukturen
über Intervallen, die mit $a$ beginnen (normalerweise mit $a\in\left\{ 0,1\right\} $).
\end{defn}
Der Lesbarkeit halber verwenden wir die Infixnotation $a\leqslant b$
anstelle von $\left(a,b\right)\in\leqslant$ oder $\left[\leqslant\right]\left(a,b\right)$.
Die Symbole $\dot{\leqslant}$ und $\dot{=}$ seien gleichbedeutend
mit den Symbolen $\leqslant$ und $=$, und werden gegebenenfalls
in Gleichungen wie $\varphi=x\dot{=}y$ und $\varphi=x\dot{\leqslant}y$
verwendet.
\begin{defn}
\textbf{\label{def:isomorphism}Isomorphismus}

Für zwei $\sigma$-Strukturen $\mathfrak{A}$ und $\mathfrak{B}$
sei eine bijektive Abbildung $\pi:A\rightleftarrows B$ ein Isomorphismus,
falls $\pi R^{\mathfrak{A}}=R^{\mathfrak{B}}$ für alle Symbole $R\in\sigma$
gilt.

Die Abbildung $\pi$ wird auf natürliche Weise auf Strukturen erweitert:
\[
\pi\mathfrak{A}\coloneqq\left(\pi A,\left(\pi R^{\mathfrak{A}}\right)_{R\in\sigma}\right)
\]

Die Menge aller Isomorphismen bezeichnen wir mit $\mathrm{Bij}\left(\mathfrak{A},\mathfrak{B}\right)$.
Zwei Strukturen heißen isomorph (kurz $\mathfrak{A}\cong\mathfrak{B}$),
falls $\mathrm{Bij}\left(\mathfrak{A},\mathfrak{B}\right)$ nicht
leer ist. Wir schreiben $\left(\mathfrak{A},\bar{a}\right)\cong\left(\mathfrak{B},\bar{b}\right)$,
wenn insbesondere ein $\pi\in\mathrm{Bij}\left(\mathfrak{A},\mathfrak{B}\right)$
mit $\pi\bar{a}=\bar{b}$ existiert.

Ein Automorphismus $\pi\in\mathrm{Bij}\left(\mathfrak{A},\mathfrak{A}\right)$
sei ein Isomorphismus von $\mathfrak{A}$ zu sich selbst. Die Menge
der Automorphismen $\mathrm{Bij}\left(\mathfrak{A},\mathfrak{A}\right)$
nennen wir $\mathrm{Aut}_{\mathfrak{A}}$; diese bilden (so wie die
Permutationen einer Menge) eine Gruppe bezüglich der Verkettung $\circ$
und dem neutralen Element $\mathbf{id}_{A}$.

Der Orbit eines Elements $a\in A$ sei analog zu Definition \ref{def:orbit}
die Menge $\mathrm{Orb}_{\mathfrak{A}}\left(a\right)\coloneqq\left\{ \pi a\mid\pi\in\mathrm{Aut}_{\mathfrak{A}}\right\} $
aller Elemente, auf die $a$ von einem Automorphismus abgebildet werden
kann.
\end{defn}
%
\begin{defn}
\textbf{\label{def:disjoint-union}Vereinigung von Strukturen}

Zwei Strukturen können vereinigt werden, wenn sie entweder disjunkte
Signaturen oder die gleiche Signatur besitzen. Für eine $\sigma_{1}$-Struktur
$\mathfrak{A}$ und eine $\sigma_{2}$-Struktur $\mathfrak{B}$ gelte:

\end{defn}
\begin{enumerate}
\item Wenn $\sigma_{1}\cap\sigma_{2}=\emptyset$, so ist $\mathfrak{A}\cup\mathfrak{B}$
die folgende $\left(\sigma_{1}\cup\sigma_{2}\right)$-Struktur:
\[
\mathfrak{A}\cup\mathfrak{B}\coloneqq\left(A\cup B,\left(R^{\mathfrak{A}}\right)_{R\in\sigma_{1}},\left(R^{\mathfrak{B}}\right)_{R\in\sigma_{2}}\right)
\]
\item Wenn $\sigma_{1}=\sigma_{2}=\sigma$, so ist $\mathfrak{A}\cup\mathfrak{B}$
die folgende $\sigma$-Struktur:
\[
\mathfrak{A}\cup\mathfrak{B}\coloneqq\left(A\cup B,\left(R^{\mathfrak{A}}\cup R^{\mathfrak{B}}\right)_{R\in\sigma}\right)
\]
\end{enumerate}
\begin{defn}
Falls die beiden Strukturen disjunkte Universen $A\cap B=\emptyset$
haben, so heiße $\mathfrak{A}\cup\mathfrak{B}=\mathfrak{A}\uplus\mathfrak{B}$
die \textbf{disjunkte Vereinigung} der Strukturen. Wenn sie disjunkte
Signaturen und das gleiche Universum haben, so nennen wir $\mathfrak{A}\cup\mathfrak{B}=\mathfrak{A}\oplus\mathfrak{B}$
die \textbf{Konkatenation} der Strukturen.
\end{defn}
\begin{prop}
\label{prop:iso-closed-disjoint}Isomorphismen sind unter disjunkter
Vereinigung abgeschlossen.
\end{prop}
\begin{proof}
Seien $\mathfrak{A},\mathfrak{A}'\in\mathbf{FIN}\left(\sigma_{1}\right)$
und $\mathfrak{B},\mathfrak{B}'\in\mathbf{FIN}\left(\sigma_{2}\right)$
jeweils zwei Paare isomorpher Strukturen mit $\pi_{1}\in\mathrm{Bij}\left(\mathfrak{A},\mathfrak{A}'\right)$
und $\pi_{2}\in\mathrm{Bij}\left(\mathfrak{B},\mathfrak{B}'\right)$.
Sei ferner $A\cap B=A'\cap B'=\emptyset$, und seien $\bar{a}\in A^{k},\bar{a}'\in A'^{k},\bar{b}\in B^{k'},\bar{b}\in B'^{k'}$
vier Tupel mit $\pi_{1}\bar{a}=\bar{a}'$ und $\pi_{2}\bar{b}=\bar{b}'$.

So gilt für die Abbildung $\pi\coloneqq\pi_{1}\cup\pi_{2}:A\uplus A'\rightarrow B\uplus B'$:

\end{proof}
\begin{casenv}
\item Wenn $\sigma_{1}\cap\sigma_{2}=\emptyset$:
\begin{eqnarray*}
\pi R^{\mathfrak{A}\uplus\mathfrak{B}} & = & \pi_{1}R^{\mathfrak{A}}=R^{\mathfrak{A}'}=R^{\mathfrak{A}'\uplus\mathfrak{B}'}\,\mathrm{f\ddot{u}r}\,R\in\sigma_{1}\\
\pi R^{\mathfrak{A}\uplus\mathfrak{B}} & = & \pi_{2}R^{\mathfrak{B}}=R^{\mathfrak{B}'}=R^{\mathfrak{A}'\uplus\mathfrak{B}'}\,\mathrm{f\ddot{u}r}\,R\in\sigma_{2}
\end{eqnarray*}
\item Wenn $\sigma_{1}=\sigma_{2}$, dann gilt für alle $R\in\sigma_{1}$:
\[
\pi R^{\mathfrak{A}\cup\mathfrak{B}}=\pi_{1}R^{\mathfrak{A}}\uplus\pi_{2}R^{\mathfrak{B}}=R^{\mathfrak{A}'}\uplus R^{\mathfrak{B}'}=R^{\mathfrak{A}'\uplus\mathfrak{B}'}
\]
\end{casenv}
\begin{proof}
Damit ist $\pi\left(\mathfrak{A}\uplus\mathfrak{B}\right)=\mathfrak{A}'\uplus\mathfrak{B}'$
und $\pi\bar{a}\bar{b}=\bar{a}'\bar{b}'$, und $\left(\mathfrak{A}\uplus\mathfrak{B},\bar{a}\bar{b}\right)\cong\left(\mathfrak{A}'\uplus\mathfrak{B}',\bar{a}'\bar{b}'\right)$.
\end{proof}
%
\begin{defn}
\textbf{\label{def:induced-structure}Induzierte Teilstruktur}

Für eine Relation $R\subseteq A^{k}$ und eine Teilmenge $A'\subseteq A$
sei $R_{\mid A'}\coloneqq R\cap\left(A'\right)^{k}$ die von $A'$
induzierte Teilrelation. Für eine $\sigma$-Struktur $\mathfrak{A}$
sei $\mathfrak{A}_{\mid A'}\coloneqq\left(A',\left(R_{\mid A'}^{\mathfrak{A}}\right)_{R\in\sigma}\right)$
die von der Teilmenge $A'$ in $\mathfrak{A}$ induzierte Teilstruktur.
\end{defn}

\begin{defn}
\textbf{\label{def:neighborhoods}$r$-Nachbarschaft} (siehe Abschnitt
2.4 von \cite{EbbinghausFlum})

Für eine Struktur $\mathfrak{A}\in\mathbf{FIN}\left(\sigma\right)$
sei $\mathcal{G}\left(\mathfrak{A}\right)\in\mathbf{FIN}^{A}\left(\left\{ E\right\} \right)$
der \textbf{Gaifman-Graph} von $\mathfrak{A}$, der alle Knoten miteinander
verbindet, die im gleichen Tupel vorkommen. Insbesondere ist für jeden
symmetrischen\footnote{Der Gaifman-Graph ist ungerichtet, wird aber hier als symmetrische,
gerichtete $\left\{ E\right\} $-Struktur modelliert.} Graphen $\mathfrak{A}\in\mathbf{FIN}\left(\left\{ E\right\} \right)$
offensichtlich $\mathcal{G}\left(\mathfrak{A}\right)=\mathfrak{A}$.
\begin{eqnarray*}
\mathcal{G}\left(\mathfrak{A}\right) & = & \left(A,E^{\mathcal{G}\left(\mathfrak{A}\right)}\right)\\
E^{\mathcal{G}\left(\mathfrak{A}\right)} & = & \left\{ \left(a_{i},a_{j}\right)\mid R/k\in\sigma,\,\bar{a}\in R^{\mathfrak{A}},\,i,j\in\left[1,k\right],\,i\neq j\right\} 
\end{eqnarray*}

Sei $\mathrm{dist}_{\mathfrak{A}}:A^{2}\rightarrow\mathbb{N\cup\left\{ \infty\right\} }$
die minimale Distanz $\mathrm{dist}_{\mathfrak{A}}\left(a,b\right)$
in $\mathcal{G}\left(\mathfrak{A}\right)$ zwischen zwei Knoten $a,b\in A$.
Für ein $r\in\mathbb{N}$ sei nun $N_{r}^{\mathfrak{A}}\left(a\right)\coloneqq\left\{ b\in A\mid\mathrm{dist}_{\mathfrak{A}}\left(a,b\right)\leqslant r\right\} $
die \textbf{$r$-Kugel} um $a\in A$. Deren induzierte Teilstruktur
$\mathcal{N}_{r}^{\mathfrak{A}}\left(a\right)=\mathfrak{A}_{\mid N_{r}\left(a\right)}$
in $\mathfrak{A}$ sei die \textbf{$r$-Umgebung} von $a$.

Für zwei Strukturen $\mathfrak{A},\mathfrak{B}\in\mathbf{FIN}\left(\sigma\right)$
und zwei Elemente $a\in A$, $b\in B$ schreiben wir $a\sim_{r}b$
($a$ ist \textbf{$r$-ähnlich} zu $b$), wenn ein Isomorphismus $\pi\in\mathrm{Bij}\left(\mathcal{N}_{r}^{\mathfrak{A}}\left(a\right),\mathcal{N}_{r}^{\mathfrak{B}}\left(b\right)\right)$
mit $\pi a=b$ existiert.
\end{defn}
%
\begin{defn}
\textbf{$\sigma$-Anfragen}

Eine $\sigma$\textbf{-Anfrage} $q$ mit der Stelligkeit $\mathrm{ar}\left(q\right)=k$
sei eine Abbildung jeder endlichen $\sigma$-Struktur $\mathfrak{A}\in\mathbf{FIN}\left(\sigma\right)$
auf eine Relation $q\left(\mathfrak{A}\right)\subseteq A^{k}$. Eine
$\sigma$\textbf{-Eigenschaft} $S\subseteq\mathbf{FIN}\left(\sigma\right)$
sei eine Menge von $\sigma$-Strukturen und entspreche der 0-stelligen
Anfrage $q_{S}$:
\begin{eqnarray*}
q_{S}\left(\mathfrak{A}\right) & \coloneqq & \begin{cases}
\left\{ \left\langle \right\rangle \right\}  & \mathrm{falls}\,\mathfrak{A}\in S\\
\emptyset & \mathrm{sonst}
\end{cases}
\end{eqnarray*}

Per Definition sind alle $\sigma$-Anfragen und $\sigma$-Eigenschaften
unter Isomorphismen abgeschlossen: Für $\mathfrak{A}\cong\mathfrak{B}$
und $\pi\in\mathrm{Bij}\left(\mathfrak{A},\mathfrak{B}\right)$ gilt
$\pi q\left(\mathfrak{A}\right)=q\left(\mathfrak{B}\right)$ und $\mathfrak{A}\in S\Leftrightarrow\mathfrak{B}\in S$.
\end{defn}

